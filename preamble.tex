%%%%%%%%%%%%%%%%%%%%%%%%%%%%%%%%%%%%%%%%%%%%%%%%%%%%%%%%%%%%%%%%%%%%%%%%%%%%%%%% 
%
%   Basic configuration
%
%%%%%%%%%%%%%%%%%%%%%%%%%%%%%%%%%%%%%%%%%%%%%%%%%%%%%%%%%%%%%%%%%%%%%%%%%%%%%%%%

% Use 'KOMA-Script Book' as the document class
\documentclass%
[%
    toc=bibliography,%
    toc=indentunnumbered,%
    listof=totoc,%
    chapterprefix=true,%
    captions=tableheading% Makes the heading of tables have formatting
                          % appropriate to being placed above them.    
]{scrbook}



%%%%%%%%%%%%%%%%%%%%%%%%%%%%%%%%%%%%%%%%%%%%%%%%%%%%%%%%%%%%%%%%%%%%%%%%%%%%%%%%
%
%   Preload essentials
%
%%%%%%%%%%%%%%%%%%%%%%%%%%%%%%%%%%%%%%%%%%%%%%%%%%%%%%%%%%%%%%%%%%%%%%%%%%%%%%%%

% Load some required packages
\usepackage{etoolbox}                   % Modding standard environments
\usepackage[intlimits]{amsmath}   % Common mathematical environments.
                                        % Option fleqn removed for centered eqs.
\usepackage[svgnames]{xcolor}           % Enables coloring of text and pages
\usepackage{hyperref}                   % Enables hyperlink generation



%%%%%%%%%%%%%%%%%%%%%%%%%%%%%%%%%%%%%%%%%%%%%%%%%%%%%%%%%%%%%%%%%%%%%%%%%%%%%%%%
%
%   Page design
%
%%%%%%%%%%%%%%%%%%%%%%%%%%%%%%%%%%%%%%%%%%%%%%%%%%%%%%%%%%%%%%%%%%%%%%%%%%%%%%%%

% Font size
\KOMAoptions{fontsize=11pt}

% Line spacing
\linespread{1.04} 

% Paper format
\KOMAoptions{paper=B5}

% Duplex layout
\KOMAoptions{twoside}

% Page layout [1/sqrt(3)]
\usepackage[text={108.25mm,187.50mm},hmarginratio=1:1,vmarginratio=1:2]{geometry}

% Page layout [1/sqrt(2)]
%\KOMAoptions{BCOR=15mm}
%\KOMAoptions{DIV=11}

% Page layout [Classic circle]
%\KOMAoptions{BCOR=15mm}
%\KOMAoptions{DIV=classic}

% Disable headers
\pagestyle{plain}

% Font used for page numbers
%\addtokomafont{pagenumber}{\lining\scshape}
\addtokomafont{pagenumber}{\scshape}

% Use spacing instead of indentation to separate paragraphs
%\KOMAoptions{parskip=half+} 

% Don't stretch the content to fill entire pages
\raggedbottom

% Don't break paragraphs because of a single line
\PassOptionsToPackage{defaultlines=2,all}{nowidow}

% Permit some hyphenation in ragged-right blocks
\PassOptionsToPackage{newcommands}{ragged2e}



%%%%%%%%%%%%%%%%%%%%%%%%%%%%%%%%%%%%%%%%%%%%%%%%%%%%%%%%%%%%%%%%%%%%%%%%%%%%%%%%
%
%   Document fonts
%
%%%%%%%%%%%%%%%%%%%%%%%%%%%%%%%%%%%%%%%%%%%%%%%%%%%%%%%%%%%%%%%%%%%%%%%%%%%%%%%%

%% Load font management packages
\usepackage[no-math]{fontspec} 
\usepackage{unicode-math}
%\usepackage{realscripts}
%\usepackage{microtype}
%
% Where to look for fonts
%\defaultfontfeatures{Path={Fonts/}}

% Scale all fonts to the same x-height
\defaultfontfeatures{Scale=MatchLowercase}
%
%% Use italics for all math letters
%\unimathsetup%
%{ 
%  math-style=ISO,
%  nabla=upright,
%  partial=upright
%}
%
%% Turn on "contextual alternates"
%\defaultfontfeatures{RawFeature={+calt}}
%
%% Define commands to switch number style
\newcommand{\lining}{\addfontfeature{Numbers={Lining}}}
%\newcommand{\oldstyle}{\addfontfeature{Numbers={OldStyle}}}
%
%% Serif font (used for body text)
%\setmainfont{Libertinus Serif}%
%[
%  UprightFont      = {*-Regular},
%  ItalicFont       = {*-Italic},
%  BoldFont         = {*-Semibold},
%  BoldItalicFont   = {*-SemiboldItalic},
%  Numbers          = {OldStyle},
%  PunctuationSpace = 1.125
%]
\setmainfont[Ligatures=TeX]{TeX Gyre Pagella}
%
%% Sans font (used for titling)
\setsansfont{URW Classico}%
[
  UprightFont    = {*-Regular},
  ItalicFont     = {*-Italic},
  BoldFont       = {*-Bold},
  Numbers        = {Proportional,Lining},
  Scale          = MatchUppercase
]
%
%% Math font (used for equations)
%\setmathfont{Libertinus Math}

% Create a specific font family for heading prefix
\newfontfamily\prefixFont[Ligatures=TeX]{Tex Gyre Pagella}%
\newfontfamily\headerTypeface[Ligatures=TeX]{URW Classico}


%%%%%%%%%%%%%%%%%%%%%%%%%%%%%%%%%%%%%%%%%%%%%%%%%%%%%%%%%%%%%%%%%%%%%%%%%%%%%%%%
%
%   Table of contents
%
%%%%%%%%%%%%%%%%%%%%%%%%%%%%%%%%%%%%%%%%%%%%%%%%%%%%%%%%%%%%%%%%%%%%%%%%%%%%%%%%
 
% Load a package for styling the table of contents
\usepackage{tocstyle}

% Place page numbers right after the section entries
\usetocstyle{nopagecolumn}

% Do not include subsections in the table of contents
\setcounter{tocdepth}{1}

% Fix vertical spacing after table of contents title
\BeforeTOCHead[toc]{\RedeclareSectionCommand[beforeskip=1sp,afterskip=1sp]{chapter}}

%% Use tabular lining figures for the sections, but oldstyle figures for the pages
%\settocstylefeature{entryhook}{\lining}
%\settocstylefeature{pagenumberhook}{\oldstyle}
%\settocstylefeature[0]{entryhook}{\lining\bfseries}
%\settocstylefeature[0]{pagenumberhook}{\oldstyle\bfseries}



%%%%%%%%%%%%%%%%%%%%%%%%%%%%%%%%%%%%%%%%%%%%%%%%%%%%%%%%%%%%%%%%%%%%%%%%%%%%%%%%
%
%   Headings
%
%%%%%%%%%%%%%%%%%%%%%%%%%%%%%%%%%%%%%%%%%%%%%%%%%%%%%%%%%%%%%%%%%%%%%%%%%%%%%%%%

% Change the font used for headings
\addtokomafont{disposition}{\sffamily}

% Change the sizes of chapters and sections
\addtokomafont{chapter}{\LARGE}
\addtokomafont{section}{\large}

%% Change spacing around chapters and sections
%\RedeclareSectionCommand[beforeskip=-0.0\baselineskip,afterskip=0.5\baselineskip]{chapter}
\RedeclareSectionCommand[beforeskip=-1.0\baselineskip,afterskip=0.5\baselineskip]{section}
%
%% Bringhurst-style chapter numbers in the margin
%\makeatletter
%  \newsavebox{\feline@chapter}
%  \newcommand{\feline@chapter@marker}[1][4cm]{\sbox\feline@chapter{\resizebox{!}{#1}{\setlength{\fboxsep}{0pt}\color{gray}\thechapter}}\parbox[b][0.5cm]{1.5cm}{\usebox{\feline@chapter}\vspace*{-1.325cm}}}
%  \renewcommand*{\chapterformat}{\sbox\feline@chapter{\feline@chapter@marker[1.6cm]}\makebox[0pt][l]{\makebox[\dimexpr\textwidth+2.0\marginparsep+\wd\feline@chapter\relax][r]{\usebox\feline@chapter}}}
%\makeatother

% Changing the font of the chapter number prefix
%\newkomafont{customChapterPrefix}{%
%    \bfseries%
%}

% Create Lenny-style chapter titles
% Change spacing around chapters
\newlength{\chapterVSpace}
\setlength{\chapterVSpace}{2.5\baselineskip}
\RedeclareSectionCommand[beforeskip=6\baselineskip,afterskip=\chapterVSpace]{chapter}
% Defining boxes and lengths
\newsavebox{\thechapternumber}
\newsavebox{\chapterPrefix}
\newlength\prefixLength
\newlength\ruleWidth
\setlength{\ruleWidth}{1pt}
% Adjusting prefix typeface.
\addtokomafont{chapterprefix}{\prefixFont\scshape\Large}
% Redefining chapter prefix and number
\renewcommand*{\chapterformat}{%
    \parbox[c]{\textwidth}{%
        \savebox{\thechapternumber}{\,\color{gray}\scalebox{3}{\headerTypeface\thechapter\autodot}} % Definition of box of the chapter number.
        \savebox{\chapterPrefix}{\chapappifchapterprefix{\enskip}} % Definition of box containing "Chapter" text.
        \setlength{\prefixLength}{\dimexpr\wd\chapterPrefix+\wd\thechapternumber\relax}
%
        % Vertical left line.
        \vspace{10pt}\rule[\dimexpr\ht\chapterPrefix+1pt\relax]{\ruleWidth}{\dimexpr\ht\thechapternumber-\ht\chapterPrefix-1pt\relax}\hspace{-\ruleWidth}%
        % Top left horizontal line
        \rule[\ht\thechapternumber]{\dimexpr\wd\chapterPrefix\relax}{\ruleWidth}\hspace{\dimexpr-\wd\chapterPrefix\relax}%
        % Boxes and lower right horizontal line
        \usebox{\chapterPrefix}\raisebox{2pt}{\usebox{\thechapternumber}}\rule{\dimexpr\textwidth-\prefixLength-1pt\relax}{\ruleWidth}%
        % Top right horizontal line.
        \hspace{\dimexpr -\textwidth+\prefixLength + 1pt\relax}%
        \rule[\ht\thechapternumber]{\dimexpr\textwidth -\prefixLength-1pt\relax}{\ruleWidth}%
        % Vertical right line
        \hspace{-\ruleWidth}\rule{\ruleWidth}{\ht\thechapternumber}
    }
}
% Redefining overall chapter header when prefix option is used.
\renewcommand*{\chapterlineswithprefixformat}[3]{%
    \parbox[b]{\textwidth}{#2}\par\vspace{0.5\chapterVSpace}%
    #3\par%
}

% The default koma script definition of \chapterformat

%\renewcommand*{\chapterformat}{%
%    \mbox{\chapappifchapterprefix{\nobreakspace}{\thechapter\autodot}%
%    \IfUsePrefixLine{}{\enskip}}
%}



%%%%%%%%%%%%%%%%%%%%%%%%%%%%%%%%%%%%%%%%%%%%%%%%%%%%%%%%%%%%%%%%%%%%%%%%%%%%%%%%
%
%   Captions
%
%%%%%%%%%%%%%%%%%%%%%%%%%%%%%%%%%%%%%%%%%%%%%%%%%%%%%%%%%%%%%%%%%%%%%%%%%%%%%%%%

%% Change the font used for captions
\addtokomafont{caption}{\small}
%
%% Change the font used for labels
\addtokomafont{captionlabel}{\bfseries}

% Add 2em margins on each side of the caption. (Since the default 
% \parindent is 1em, this implies that the left end of the caption 
% will always look one \parindent indented if it comes right before
% or after a new paragraph, and can thus prevent weird indentation.)
\setcapdynwidth{\dimexpr\textwidth-4em\relax}

% Disable extra indentation of subsequent lines in a multiline caption
\setcapindent{0em}



%%%%%%%%%%%%%%%%%%%%%%%%%%%%%%%%%%%%%%%%%%%%%%%%%%%%%%%%%%%%%%%%%%%%%%%%%%%%%%%%
%
%   Footnotes
%
%%%%%%%%%%%%%%%%%%%%%%%%%%%%%%%%%%%%%%%%%%%%%%%%%%%%%%%%%%%%%%%%%%%%%%%%%%%%%%%%

% Make sure footnote marks are separated by commas and kerned properly
\usepackage[multiple=true,mult-fn-sep=${}^{,\kern-0.07em}$]{fnpct}

% Change the font used for footnotes
%\addtokomafont{footnote}{\sffamily}

% Change the footnote marks to lining numbers
%\renewcommand{\thefootnote}{\lining{\arabic{footnote}}}
\renewcommand{\thefootnote}{\arabic{footnote}}

% Change the footnote marks to Latin letters
%\renewcommand{\thefootnote}{\textit{\alph{footnote}}}

% Change the footnote marks to symbols
%\usepackage[wiley]{footmisc}
%\renewcommand{\thefootnote}{\fnsymbol{footnote}}

% Set the footnote rule length to the text width
\setfootnoterule{\textwidth}

% Remove the footnote rule entirely
%\setfootnoterule{0pt}

% Adjust the footnote formatting and spacing
\deffootnote{2.15em}{2.15em}{\thefootnotemark.\kern0.75em}



%%%%%%%%%%%%%%%%%%%%%%%%%%%%%%%%%%%%%%%%%%%%%%%%%%%%%%%%%%%%%%%%%%%%%%%%%%%%%%%%
%
%   Hyperlinks
%
%%%%%%%%%%%%%%%%%%%%%%%%%%%%%%%%%%%%%%%%%%%%%%%%%%%%%%%%%%%%%%%%%%%%%%%%%%%%%%%%

% Table of contents links
\hypersetup{linktoc=page}

% Color the hyperlinks
\hypersetup{colorlinks}
\hypersetup{allcolors=Navy}

% Font used for hyperlinks
\urlstyle{rm}

% Fix kerning problems for backslashes and redefine underscores in hyperlinks
\makeatletter
  \let\UrlSpecialsOld\UrlSpecials
  \def\UrlSpecials{\UrlSpecialsOld\do\/{\Url@slash}\do\_{\Url@underscore}}%
  \def\Url@slash{\@ifnextchar/{\kern+0.05em\mathchar47\kern-0.10em}%
      {\kern0.08em\mathchar47\penalty\UrlBigBreakPenalty}}
  \def\Url@underscore{\nfss@text{\leavevmode \kern.06em\vbox{\hrule height 0.12ex width 0.4em}}}
\makeatother



%%%%%%%%%%%%%%%%%%%%%%%%%%%%%%%%%%%%%%%%%%%%%%%%%%%%%%%%%%%%%%%%%%%%%%%%%%%%%%%%
%
%   References
%
%%%%%%%%%%%%%%%%%%%%%%%%%%%%%%%%%%%%%%%%%%%%%%%%%%%%%%%%%%%%%%%%%%%%%%%%%%%%%%%%

% Bibliography backend (e.g. 'biber' or 'bibtex')
\PassOptionsToPackage{backend=biber}{biblatex}

% Bibliography style (e.g. 'phys' or 'nature')
\PassOptionsToPackage{style=nicobib}{biblatex}

% Citation style (e.g. 'plain' or 'superscript')
\PassOptionsToPackage{autocite=plain}{biblatex} 

% Enable multiple bibliographies with separate numbering
\PassOptionsToPackage{defernumbers=true}{biblatex}

% Format for cross-references with \cref
\PassOptionsToPackage{noabbrev}{cleveref}
\newcommand{\crefrangeconjunction}{--}



%%%%%%%%%%%%%%%%%%%%%%%%%%%%%%%%%%%%%%%%%%%%%%%%%%%%%%%%%%%%%%%%%%%%%%%%%%%%%%%%
%
%   Postload packages
%
%%%%%%%%%%%%%%%%%%%%%%%%%%%%%%%%%%%%%%%%%%%%%%%%%%%%%%%%%%%%%%%%%%%%%%%%%%%%%%%%

% Load the required packages
\usepackage{biblatex}   % Produces the bibliography
\usepackage{ragged2e}   % Permits ragged-right with hyphenation
\usepackage{nowidow}    % Prevents widows and orphans in text
\usepackage{cleveref}   % Easy and consistent cross-references
\usepackage{graphicx}   % Loads and displays figures
\usepackage{pdfpages}   % Enables embedding of documents
\usepackage{booktabs}   % Proper formatting of tables
\usepackage{siunitx}    % Proper formatting of units
\usepackage{mhchem}     % Proper formatting of chemicals
\usepackage{lipsum}     % Insertion of arbitrary content

% Load custom packages
\usepackage{dirtytalk}
\usepackage{amsthm}     % Customization of theorem environments
\usepackage{tikz-feynman}
\usepackage[acronym]{glossaries}



%%%%%%%%%%%%%%%%%%%%%%%%%%%%%%%%%%%%%%%%%%%%%%%%%%%%%%%%%%%%%%%%%%%%%%%%%%%%%%%%
%
%   Miscellaneous
%
%%%%%%%%%%%%%%%%%%%%%%%%%%%%%%%%%%%%%%%%%%%%%%%%%%%%%%%%%%%%%%%%%%%%%%%%%%%%%%%%

%% Enforce a consistent Greek style
%%\AtBeginDocument{\let\nabla=𝛁}
%\AtBeginDocument%
%{
%  \let\epsilon=\varepsilon
%  \let\phi=\varphi
%}

% Change the font used for tables
\AtBeginEnvironment{tabular}{\small}
\AtBeginEnvironment{tabular*}{\small}

%% Use lining numbers for chemistry and physics
%\mhchemoptions{textfontcommand=\lining}
%\sisetup{text-rm=\lining}
%
%% Format for typesetting physical units
%\sisetup{range-units=single}
%\sisetup{range-phrase=--}
%\sisetup{detect-all=true} 

% Use 2em equation indentation
\makeatletter
  \setlength\@mathmargin{2em}
\makeatother

% Set width of a column
\newlength\colparwidth
%\setlength{\colparwidth}{\dimexpr\textwidth-2\fboxsep}
\setlength{\colparwidth}{\dimexpr\textwidth-6pt}

% Replace \cite with the more flexible \autocite
\let\cite=\autocite

% Define a custom color palette
\definecolor{whiteish}{rgb}{1.000, 0.964, 0.859}
\definecolor{rosewood}{rgb}{0.396, 0.000, 0.043}

% Declare a custom article format for my papers
\DeclareBibliographyAlias{customa}{article}
\DeclareFieldFormat[customa]{labelnumber}{\textsc{\Rn{#1}}}

% Where to look for figure files
\graphicspath{{./Figs/}}


%%%%%%%%%%%%%%%%%%%%%%%%%%%%%%%%%%%%%%%%%%%%%%%%%%%%%%%%%%%%%%%%%%%%%%%%%%%%%%%%
%
%   Custom macros
%
%%%%%%%%%%%%%%%%%%%%%%%%%%%%%%%%%%%%%%%%%%%%%%%%%%%%%%%%%%%%%%%%%%%%%%%%%%%%%%%%

% Abbreviations
\newcommand{\eg}{e.g.\ }
\newcommand{\ie}{i.e.\ }
\newcommand{\cf}{c.f.\ }
\newcommand{\etal}{et al.\ }
\newcommand{\irr}{irrep.\ }
\newcommand{\ham}{\mathcal{H}}
\newcommand{\hil}{\mathscr{H}}
\newcommand{\hilB}{\mathcal{B}}

% Common notation
\renewcommand{\d}[1]{\mathop{\textrm{d}\kern0em#1\kern0.1em}}
%\let\Oldint=\int
%\renewcommand{\int}{ds\Oldint\!\!\!\!}
\newcommand{\sint}{\int\!}
\newcommand{\avg}[1]{\langle #1 \rangle}
\newcommand{\up}{\uparrow}
\newcommand{\dn}{\downarrow}
\newcommand{\trans}{{\symsfup{T}}}
\newcommand{\boltz}{k_\textrm{B}}
\renewcommand{\v}[1]{{\symbfit{#1}}}
\newcommand{\abs}[1]{\lvert #1 \rvert}
\newcommand{\ket}[1]{\lvert #1 \rangle}
\newcommand{\bra}[1]{\langle #1 \rvert}
\newcommand{\braket}[2]{\langle #1 \mid #2 \rangle}

% Common functions
\DeclareMathOperator{\tr}{Tr}
\DeclareMathOperator{\re}{Re}
\DeclareMathOperator{\im}{Im}
\DeclareMathOperator{\atan}{atan}
\DeclareMathOperator{\atanh}{atanh}
\DeclareMathOperator{\asin}{asin}
\DeclareMathOperator{\asinh}{asinh}
\DeclareMathOperator{\acos}{acos}
\DeclareMathOperator{\acosh}{acosh}
\DeclareMathOperator{\Span}{span}
\DeclareMathOperator{\pfaff}{Pf}

% Theorem environments
\theoremstyle{plain}
\newtheorem{defi}{Def.}[chapter]
\newtheorem{thm}{Thm.}[chapter]

% Acronyms
\makeglossaries
\setacronymstyle{long-short}
\def\ac{\gls*}
\newacronym{mcmc}{MCMC}{Markov-Chain Monte-Carlo}
