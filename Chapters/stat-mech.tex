\chapter{Statistical Mechanics}
\label{chap:statMech}
%
\noindent In statistical mechanics we attempt to describe an ensemble of particles that may be interacting
to extract not precise information about what each and every particle is doing, but statistical
information about what most of the particles are doing. One can imagine this process as zooming out
from a detailed view of individual entities and viewing the resulting collection through squinted eyes.
This amounts to treating the collection in a course-grained manner. Through such eyes, only the most significant
behavior is perceptable such that it can be understood and described in a simplified way.

\section{Canonical ensemble and the partition function}

Most of the business of statistical mechanics is about calculating what is known as the partition-
function. Once this function is known, all the heavy lifting is done since most important statistical
quantities can be extracted from it following already established systematic steps.
To calculate the partition function is theoretically very simple: we sum the quantity $e^{-\beta E_i}$
over all the possible states of the system. Every state is a particular configuration of things in
the system and since all things in the system have a certain energy, the energy of a system state is
given by the sum of all these things' individual energy, pluss any energy given by interactions between
then.
If we label
each state of the system with the index $i$, then we can denote the energy of each state $E_i$. The
definition of the partition function $Z$ in the canonical ensemble can then be written
\begin{equation}
    Z = \sum_i e^{-\beta E_i},
    \label{eq:statMech:partDef}
\end{equation}
where $\beta = 1/(\boltz T)$ and $\boltz$ is the Boltzmann constant\footnote{In SI units, the value of the 
Boltzmann constant is given by 
$\boltz \approx \SI{1.380649d-23}{JK^{-1}}$}, given that the number of different possible states is
countable.

As we can see, the essential ingredients needed to calculate the partition function is one: to be able to
enumerate all possible states $i$ of the system, and two: to be able to calculate their corresponding energies $E_i$.
Since we have used the summation sign $\sum_i$ in Eq.~\eqref{eq:statMech:partDef}, we have 
assumed that there exists a countable number of different states. However if there is one thing in the
system that can change in a continuous fashion, which we would measure using the set of real numbers
$\mathbb{R}$ and some unit, then the number of states is infinite and uncountable. In this case we sum over the
different numbers of states by simply integrating over the things that are continuous and the unit
of the partition function becomes the product of the units of the continuous variables (things).
In most cases, it is the position $\v{r}$ and momentum $\v{p}$ of particles in the system that are continuous, hence
the definition of the partition function becomes%
\footnote{The reason why there is a $\sim$ sign in Eq.~\eqref{eq:statMech:partContDef} is that technically there is a factor
of Planck's constant $h$ in the denominator for each $\d{r}\d{p}$ in the integral measure since this makes the partition function
dimensionless and thus consistent with the definition in terms of countable number of states in Eq.~\eqref{eq:statMech:partDef}.}
\begin{equation}
    Z \sim \sum_i\sint\d{^3r}\sint\d{^3p} e^{-\beta E_i(\v{r},\v{p})}.
    \label{eq:statMech:partContDef}
\end{equation}

The canonical partition function is directly related to the Helmholtz free energy $F$ (often only referred to
as the free energy) of the system through a simple exponential
\begin{equation}
    \label{eq:statMech:freeEnergy}
    Z = e^{-\beta F}.
\end{equation}
Because of the exponentials inherent in the definition of $Z$, calculating realistic values often results
in excessively high numbers. This one of the reasons why it is more useful to work with $F$ than $Z$ itself
since inverting Eq.~\eqref{eq:statMech:freeEnergy} $F = -\ln(Z)/\beta$, reducing the value of $Z$ through
a logarithm.

\section{Calculating observables}

An observable in statistical mechanics is a quantity that we can both calculate from the statistical theory,
and (at least in principle) go out and measure in the real world. In quantum mechanics, observables
are restricted to operators that have real expectation-values, as opposed to the complex values it usually
deals with. This is to enforce the connection between observables and measurements, which we intuitively understand
to always be reducible to a series of real values, \ie points on a line.

Since we are interested again in ensembles of many particles we are restricting our attention to statistical
information about this ensemble. To get this information we need some kind of probability distribution of the
particular states of the system. We are imagining that we for each such state (indexed by $j$) can calculate
a real number for the thing (observable) we are interested in measuring. Let's call this observable $O$. Then
$O$ is a statistical variable which takes a particular value $o_j$ in the state $j$ of the system. If we
now let $P_j$ be the probability distribution, i.e. the probability that the system exists in state $j$, then
we know from probability theory that the expectation value of $O$ is
\begin{equation}
    \label{eq:statMech:statMean}
    \avg{O} = \sum_jo_jP_j.
\end{equation}
The probability distribution $P_j$ is in the canonical ensamble given by a Maxwell-distribution
\begin{equation}
    \label{eq:statMech:probDist}
    P_j = e^{-\beta E_j}/Z,
\end{equation}
normalized by the partition function $Z$. Inserting this we get
\begin{equation}
    \label{eq:stateMech:statMeanFull}
    \avg{O} = \sum_j\frac{o_je^{-\beta E_j}}{Z}.
\end{equation}

The observable of specific heat at constant volume $C_v$ is particularly important in the study of phase transitions
since its thermal
behavior can be used to classify these transitions into particular categories. The specific heat is the measure
of how much energy must be transferred to the system for its temperature to change by an infinitesimal amount
given constant volume of the system. Thus it is defined by the equation $C_v = (\frac{\partial E}{\partial T})_V$.
In the canonical ensamble, this quantity is calculated by
\begin{equation}
    \label{eq:statMech:specificHeat}
    C_v = k_\text{B}\beta^2(\avg{E^2}-\avg{E}^2) = -k_\text{B}\beta^2\Big(2\frac{\partial F}{\partial\beta}+\beta\frac{\partial^2 F}{\partial\beta^2}\Big).
\end{equation}
From this form of the specific heat we see that it can be interpreted as a measure of the variance or width of the
distribution of energies, and also that it is related to the second derivative of the free energy. Any discontinuity
in the second derivative of the free energy, thus implies a discontinuity in the specific heat.

\section{Ginzburg-Landau model}

The experimental discovery of superconductivity was a surprise to the scientists at the time. No theoretical model
had so far predicted the properties that the experimentalists were measuring. The theoretical models in use at that
time predicted a decrease in resistivity as the temperature was lowered, but its sudden disappearance was completely
unprecedented and impossible to explain classically in a convincing manner. Thus superconductivity seemed to demand
a radically different understanding of how electrons moved inside atomic structures.

\subsection{Landau Model}

Before such an understanding had been developed, Ginzburg and Landau took a shortcut and came up with a theory
that could describe the phenomenon of superconductivity without knowing its microscopic origin. In other words
they treated superconductivity as a black box and instead of asking what was inside to give the box's output,
he used the output to determine a small set of \emph{material parameters} which could then be used to predict
how the box would react to a large range of stimuli or conditions. The merit of the theory was that
he used symmetry arguments to reduce this set as much as possible.

The \ac{gl} theory of superconductivity is based on Landau's previous work on a theory of general second
order phase-transitions\footnote{2nd order phase-transitions are phase transitions of systems whose free energy
has a discontinuous second order derivative at the transition point, but is continuous for lower orders.
Since the specific heat is given by the second order derivative, then the specific heat is discontinuous in this case.}.
The approach is given by two ideas. The first is simply that the phase transition should be able to be characterized
by the appearance of some kind of measurable
order that can be described by a function $\psi$ which we call the \emph{order parameter}. In the liquid water to ice transition,
it is the position of the molecules that become ordered in a lattice\footnote{The astute reader might have noticed that this
example is a first order phase transition because of the existence of latent heat. Actually first order phase transitions
can also be described by a modified Landau theory, however we will here focus on the second order kind.},
in the magnetization of a metal it is the individual spins that become ordered along a particular direction.

The second idea is that at the phase transition, it is the appearance of this order that should dominate the behaviour
of the system, to the exclusion of all other effects. Thus the system should be described in terms of the order-parameter
and since this is infinitesimally small close to the transition, the free energy can be expanded in a Maclaurin-series
with respect to this parameter as
\begin{equation}
    \label{eq:statMech:LandauMaclaurin}
    F = F_0 + c_1 \psi + c_2 \psi^2 + c_3 \psi^3 + c_4 \psi^4 + c_5 \psi^5 + \ldots
\end{equation}

The real constants $F_0$ and $c_i$ constitute the set of material parameters of the theory and this set can then
be reduced by any symmetries that we suspect should be inherent in the underlying theory. For example, if $\psi$
should represent the order parameter of magnetization of a system of ising spins, which can point either up or
down, then the free energy should be invariant of this global choice, i.e. we need to enforce that the free energy
be invariant with respect to the transformation $\psi \mapsto -\psi$. Then all the constants $c_j$ for odd $j$
vanish.

In the case of superconductivity the order parameter $\Psi$ represents the probability amplitude of the collective
state of the superfluid of Cooper paired electrons such that $\abs{\Psi}$ can be interpreted as the density of such
electron pairs. Since $\Psi$ is complex it has to be combined with its complex conjugate $\Psi^\ast$ in ways that
yield real numbers to produce terms that are valid in the free energy since $F$ itself should be a real number.
Furthermore, the phenomenon of superconductivity is produced as a result of the breaking of $U(1)$ symmetry, so $F$
needs also to be $U(1)$ symmetric, i.e. it has to be invariant under the transformation $\Psi \mapsto e^{i\phi}\Psi$
for $\phi\in\mathbb{R}$. These restrictions result in the free energy
\begin{equation}
    \label{eq:statMech:LandauU1}
    F = F_0 - a\abs{\Psi}^2 + b\abs{\Psi}^4,
\end{equation}
when keeping the lowest order terms that produce a phase transition.

Thermodynamic equilibrium is reached at the minimum of free energy. This restricts $b\geq0$ since negative $b$ yields
a free energy with no definite minimum. The minimum is then found by the condition
\begin{equation}
    \label{eq:statMech:LandauExtreme}
    \frac{\partial F}{\partial \Psi^\ast} = (-a + 2b\abs{\Psi}^2)\Psi = 0,
\end{equation}
which yields the possibilities $\abs{\Psi} = 0$ or $\abs{\Psi} = \sqrt{a/2b}$. The first case gives the energy $F = F_0$, while the second
gives $F = F_0 - a^2/(4b)$. We see that the second case is energetically favorable, but only exists and is different from
the first case when $a>0$. Furthermore, the second case represents the ordered state since in this case the order-parameter $\abs{\Psi} \neq 0$,
in the conventional Landau theory\footnote{Actually this only represents when Cooper-pairs are forming and the real onset
of superconductivity is determined by the point in parameter-space where the gauge-mass becomes non-zero, which is closely
related but not exactly the same as where the density of Cooper-pairs becomes non-zero. The real onset of superconductivity is
thus more related to when the phase of the wave-function settles on a value.}.

It is the thermodynamic parameter of temperature that traditionally determines whether a system is in one phase or another.
Looking at the free energy in Eq.~\eqref{eq:statMech:LandauU1}, the order parameter $\Psi$ is the dynamical variable 
of the theory while the explicit temperature dependence lies in the material parameters $a$ and $b$. Denoting the
the critical temperature where the phase transition happens $T_c$, the dimensionless parameter
$t = (T-T_c)/T_c$ is small close to the critical point which means it can be used to expand the temperature dependence
of the material parameters such that
\begin{equation}
    \label{eq:statMech:matTempExpan}
    \begin{split}
        a(T) &= a_0 + a_1t + \ldots\\
        b(T) &= b_0 + b_1t + \ldots
    \end{split}
\end{equation}

Now we argue for what terms to keep in these expansions. Since $a(T)$ should change sign at $t=0$ based on the discussion of
Eq.~\eqref{eq:statMech:LandauExtreme}, then we only keep odd terms.
Since we need $b(T)>0$ for the theory to be thermodynamically stable it seems that $b_0$ is the important term that needs to be
larger than any negative contributions from the other terms. Keeping only lowest order terms then the expansions reduce to
$a(T) = a_1(T-T_c)/T_c$ and $b(T) = b_0$. Since the ordered state is the solution of the theory when $a>0$ and this ordered
state exists at temperatures $T<T_c$ then $a_1<0$ and the final temperature dependence of $a$ becomes
$a(T) = -\abs{a_1}(T-T_c)/T_c$. From this temperature dependence, it is straightforward to derive critical exponents, the specific
heat etc.

\subsection{Gradient Terms}

The simple Landau theory described above is a type of mean field theory in that there is no spatial dependence in the
solution, and thus gives a simplified picture that can only be valid far away from any defects or boundaries. This simple approach
can be extended to include spatial variation by allowing terms with gradients of the order parameter in the free energy through
a gradient expansion
\begin{equation}
    \label{eq:statMech:gradTerms:gradExpansion}
    F = \int\d{^3r}f(\Psi, \nabla\Psi, \nabla^2\Psi, \nabla^3\Psi, \ldots).
\end{equation}
Keeping only the lowest order in this expansion that is invariant under $U(1)$ symmetry we get the term
$\abs{\nabla\Psi}^2$.

Perhaps the single most important phenomenon of superconductivity from a theoretical stand-point is the fact that it expels
magnetic fields, hence it is clear that any theory that attempts to explain superconductivity needs to have some way the
superconducting order interacts with magnetic fields. The standard way to achieve this is through the recipe of
\emph{minimal coupling}, where the vector potential $\v{A}$ times a constant is subtracted with from any momentum
in the previously neutral theory. Specifically $\v{p}\mapsto \v{p}-q/c\v{A}$ where $q$ is the charge of the particle and
$c$ is the speed of light. Using this trick, then the free energy density becomes
\begin{equation}
    \label{eq:statMech:gradTerms:GLequation}
    f = f_0 - a\abs{\Psi}^2 + b\abs{\Psi}^4 + K\abs{(\nabla+ig\v{A})\Psi}^2,
\end{equation}
by letting $g=q/\hbar c$, which is the form of the free energy in the Ginzburg-Landau theory of conventional $s$-wave superconductivity.
This form of the gradient resulting from minimal coupling is called the covariant derivative and is defined as 
\begin{equation}
    \label{eq:statMech:gradTerms:covariantGradient}
    D_\mu = \partial_\mu + igA_\mu.
\end{equation}

For unconventional symmetries, the form of the gradient terms can vary substantially from that in
Eq.~\eqref{eq:statMech:gradTerms:GLequation}. Provided that the transition can be described by a single component, \ie a
single function $\Psi$, then an unconventional symmetry could still lead to anisotropies in the gradient terms such that
$K$ becomes directionally dependent. An example is single-component pairing in a tetragonal crystal: the rotational symmetry
in the $xy$-plane makes this gradient isotropic, but because of the lack of symmetry in the $z$-direction the gradient
terms take the general form
\begin{equation}
    \label{eq:statMech:gradTerms:tetragonal}
    K_1\sum_{\mu=x,y}|(\partial_\mu+igA_\mu)\Psi)|^2 + K_2|\partial_z+igA_z)\Psi|^2.
\end{equation}

Even more complex gradient terms are possible when the order-parameter consists of multiple components, \ie there are degenerate
states that all give significant contributions to the physics at the phase-transition. A particular case of this is when the
pairing state is an irreducible representation of the crystal symmetry-group that is multi-dimensional\footnote{For a more detailed explanation
of irreducible representations and group theory, see Chapter~\ref{chap:Group}.}. In this case, instead of a single complex function $\Psi$
describing the order, we need several complex functions $\eta_i$. 

A chiral $p$-wave superconductor has a pairing state that
belongs to such a two-dimensional irreducible representation which comes from the $\Gamma_5$ irreducible representation of the
tetragonal symmetry group $D_{4h}$. This representation consist of two components $\eta_x$ and $\eta_y$ that combine
to form the general gradient terms
\begin{equation}
    \label{eq:statMech:gradTerms:chiralPWave}
    \begin{split}
        &K_1\big[|D_x\eta_x|^2 + |D_y\eta_y|^2\big] + K_2\big[|D_x\eta_y|^2 + |D_y\eta_x|^2\big]\\
        & + K_3\big[(D_x\eta_x)^\ast(D_y\eta_y) + \text{c.c.}\big] + K_4\big[(D_x\eta_y)^\ast D_y\eta_x + \text{c.c.}\big]\\
        & + K_5\big[|D_z\eta_x|^2 + |D_z\eta_y|^2\big].
    \end{split}
\end{equation}
This is the general expression of the gradient terms in the model we have investigated in our work and can be found in Ref.~\cite{SigristUeda91}.
