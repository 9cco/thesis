\chapter{Statistical Mechanics}

In statistical mechanics we attempt to describe an ensamble of particles that may be interacting
to extract not precise information about what each and every particle is doing, but statistical
information about what most of the particles are doing. We zoom out and look upon the bunch of
particles and try to answer the question of what is this bunch doing. What is the most significant
behavior of this bunch as a whole.

\section{Canonical ensamble and the partition function}

Most of the business of statistical mechanics is about calculating what is known as the partition-
function. Once this function is known, all the heavy lifting is done since most important statistical
quantities can be extracted from it following already established systematic steps.
To calculate the partition function is theoretically very simple: we sum the quantity $e^{-\beta E_i}$
over all the possible states of the system. Every state is a particular configuration of things in
the system and since all things in the system has a certain energy, if we sum the energy of all the
things in the system for each state, we can say that each state has a certain energy. If we label
each state of the system with the index $i$, then we can denote the energy of each state $E_i$. The
definition of the partition function $Z$ in the canonical ensamble can then be written
\begin{equation}
    Z = \sum_i e^{-\beta E_i},
    \label{eq:statMech:partDef}
\end{equation}
where $\beta = 1/(\boltz T)$ and $\boltz$ is the Boltzmann constant given by 
$\boltz \approx \SI{1.380649d-23}{JK^{-1}}$.

As we can see, the essential ingrediants needed to calculate the partition function is to be able to
enumerate all possible states $i$ of the system and also calculate their corresponding energy $E_i$.
Since we have used the summation sign $\sum_i$ in Eq.~\eqref{eq:statMech:partDef}, we have implicitely
assumed that there exists a finite number of different states. However if there is one thing in the
system that can change in a continuous fashion, which we would measure using the set of real numbers
$\mathbb{R}$ and some unit, then the number of states is infinite. In this case we sum over the
different numbers of states by simply integrating over the things that are continuous and the unit
of the partition function becomes the product of the units of the continuous variables (things).
In most cases, it is the position and momentum of particles in the system that are continuous, hence
the definition of the partition function becomes%
\footnote{The reason why there is a $\sim$ sign in Eq.~\eqref{eq:statMech:partContDef} is that specifically there is a factor
of Planck's constant $h$ in the denominator for each $\d{r}\d{p}$ in the integral measure since this makes the partition function
dimensionless and thus consistent with the definition in terms of finite number of states.}
\begin{equation}
    Z \sim \sint\d{^3r}\sint\d{^3p} e^{-\beta E(\v{r},\v{p})}.
    \label{eq:statMech:partContDef}
\end{equation}

\section{Calculating observables}

An observable in statistical mechanics is anything that we want to measure. In quantum mechanics observables
are restricted to operators that have real eigenvalues which makes sense considering that we don't really
have an intuition for complex numbers, which is the alternative, and thus we want to restrict things we
can observe to things we can understand in terms of a point on a single line (the real axis). 

Since we are interested again in ensambles of many particles we are restricting our attention to statistical
information about this ensamble. To get this information we need some kind of probability distribution of the
particular states of the system. We are imagining that we for each such state (indexed by $j$) can calulate
a real number for the thing (observable) we are interested in measuring. Let's call this observable $O$. Then
$O$ is a statistical variable but takes a particular value $o_j$ in the state $j$ of the system. If we
now let $P_j$ be the probability distribution, i.e. the probability that the system exists in state $j$, then
we know from probability theory that the mean of the observaible is
\begin{equation}
    \label{eq:statMech:statMean}
    \avg{O} = \sum_jo_jP_j.
\end{equation}
The probability distribution $P_j$ is given by
\begin{equation}
    \label{eq:statMech:probDist}
    P_j = e^{-\beta E_j}/Z,
\end{equation}
in terms of the partition function $Z$. Inserting this we get
\begin{equation}
    \label{eq:stateMech:statMeanFull}
    \avg{O} = \sum_j\frac{o_je^{-\beta E_j}}{Z}.
\end{equation}

\section{Ginzburg-Landau model}

The experimental discovery of superconductivity was a surprise to the scientists at the time. No theoretical model
had so far predicted the properties that the experimentalists were measuring. On the contrary, they predicted
very different results no matter how they were twisted and turned, and thus superconductivity seemed to demand
a radically different understanding of how electrons moved in a material.

\subsection{Landau Model}

Before such an understanding had been developed, Landau took a shortcut and came up with a theory
that could describe the phenomenon of superconductivity without knowing its microscopic origin. In other words
he treated superconductivity as a black box and instead of asking what was inside to give the boxes output,
he used the output to determine a small set of \emph{material parameters} which could then be used to predict
how the box would react to a large range of stimuli or conditions. The merit of the theory was that
he used symmetry arguments to reduce this set as much as possible.

The Ginzburg-Landu theory of superconductivity is based on Landau's previous work on a theory for general second
order phase-transitions\footnote{2nd order phase-transitions are phase transitions of systems whose free energy
has a discontinuous second order derivative at the transition point, but is continuous for lower orders.
Since the specific heat is given by the second order derivative, then the specific heat is discontinuous in this case.}
The approach is given by two ideas. The first is simply that the phase transition should be able to be characterized
by the appearence of some kind of measurable
order that can be described by a function $\psi$ which we call the \emph{order parameter}. In the liquid water to ice transition,
it is the position of the molecules that become ordered in a lattice\footnote{The astute reader might have noticed that this
example is a first order phase transition because of the existance of latent heat. Actually first order phase transitions
is also able to be described by a modified Landau theory, however we will here focus on the second order kind.},
in the magnetization of a metal it is the individual spins that become ordered along a particular direction.

The second idea is that at the phase transition, it is the appearence of this order that should dominate the behaviour
of the system, to the exclusion of all other effects. Thus the system should be described in terms of the order-parameter
and since this is infinitesimally small close to the transition, the free energy\footnote{In this case we are talking
about the Helmholtz free energy which is related to the partition function through a logarithm while at the same time
the Legendre transformation of the internal energy of the thermodynamic system.} can be expanded in a Maclaurin-series
with respect to this parameter as
\begin{equation}
    \label{eq:statMech:LandauMaclaurin}
    F = F_0 + c_1 \psi + c_2 \psi^2 + c_3 \psi^3 + c_4 \psi^4 + c_5 \psi^5 + \ldots
\end{equation}

The real constants $F_0$ and $c_i$ constitute the set of material parameters of the theory and this set can then
be reduced by any symmetries that we suspect should be inherent in the underlying theory. For example, if $\psi$
should represent the order parameter of magnetization of a system of ising spins, which can point either up or
down, then the free energy should be invariant of this global choice, i.e. we need to enforce that the free energy
be invariant with respect to the transformation $\psi \mapsto -\psi$. Then all the constants $c_j$ for odd $j$
vanish.

In the case of superconductivity the order parameter $\Psi$ represents the probability amplitude of the collective
state of the superfluid of Cooper paired electrons such that $\abs{\Psi}$ can be interpreted as the density of such
electron pairs. Since $\Psi$ is complex it has to be combined with its complex conjugate $\Psi^\ast$ in ways that
yield real numbers to produce terms that are valid in the free energy since $F$ itself should be a real number.
Furthermore, the phenomenon of superconductivity is produced as a result of the breaking of $U(1)$ symmetry, so $F$
needs also to be $U(1)$ symmetric, i.e. it has to be invariant under the transformation $\Psi \mapsto e^{i\phi}\Psi$
for $\phi\in\mathbb{R}$. These restrictions result in the free energy
\begin{equation}
    \label{eq:statMech:LandauU1}
    F = F_0 - a\abs{\Psi}^2 + b\abs{\Psi}^4,
\end{equation}
when keeping the lowest order terms that produce a phase transition.

Thermodynamic equilibrium is reached at the minimum of free energy. This restricts $b\geq0$ since negative $b$ yields
a free energy with no definite minimum. The minimum is then found by the condition
\begin{equation}
    \label{eq:statMech:LandauExtreme}
    \frac{\partial F}{\partial \Psi^\ast} = (-a + 2b\abs{\Psi}^2)\Psi = 0,
\end{equation}
which yields the possibilities $\abs{\Psi} = 0$ or $\abs{\Psi} = \sqrt{a/2b}$. The first case gives the energy $F = F_0$, while the second
gives $F = F_0 - a^2/(4b)$. We see that the second case is energetically favorable, but only exists and is different from
the first case when $a>0$. Furthermore, the second case represents the ordered state since in this case the order-parameter $\abs{\Psi} \neq 0$,
in the conventional Landau theory\footnote{Actually this only represents when Cooper-pairs are forming and the real onset
of superconductivity is determined by the point in parameter-space where the gauge-mass becomes non-zero, which is closely
related but not exactly the same as where the density of Cooper-pairs becomes non-zero. The real onset of superconductivity is
thus more related to when the phase of the wave-function settles on a value.}.

In superconductivity it is the parameter of temperature which conventionally determines when the system enters the superconducting regime.
Looking at the free energy in Eq.~\eqref{eq:statMech:LandauU1}, the order parameter $\Psi$ is the dynamical variable 
of the theory while the explicit temperature dependence lies in the material parameters $a$ and $b$. Denoting the
temperature at which the phase transition happens --- the critical temperature: $T_c$, the dimensionless parameter
$t = (T-T_c)/T_c$ is small close to the critical point which means it can be used to expand the temperature dependence
of the material parameters such that
\begin{equation}
    \label{eq:statMech:matTempExpan}
    \begin{split}
        a(T) &= a_0 + a_1t + \ldots\\
        b(T) &= b_0 + b_1t + \ldots
    \end{split}
\end{equation}

Now we argue for what terms to keep in these expansions. Since $a(T)$ should change sign at $t=0$ then we only keep odd terms.
Since we need $b(T)>0$ for the theory to be thermodynamically stable it seems that $b_0$ is the important term which need to be
larger than any negative contributions from the other terms. Keeping only lowest order terms then the expansions reduce to
$a(T) = a_1(T-T_c)/T_c$ and $b(T) = b_0$. Since the ordered state is the solution of the theory when $a>0$ and this ordered
state exists at temperatures $T<T_c$ then $a_1<0$ and the final temperature dependence of $a$ becomes
$a(T) = -\abs{a_1}(T-T_c)/T_c$. From this temperature dependence, it is straight forward to derive critical exponents, the specific
heat etc.

\subsection{Gradient Terms}

The simple Landau theory described above is a type of mean field theory in that there is no spatial dependence in the
solution, and thus gives a simplified picture that can only be valid far away from any defects or boundaries. This simple approach
can be extended to include spatial variation by allowing terms with gradients of the order parameter in the free energy through
a gradient expansion
\begin{equation}
    \label{eq:statMech:gradTerms:gradExpansion}
    F = \int\d{^3r}f(\Psi, \nabla\Psi, \nabla^2\Psi, \nabla^3\Psi, \ldots).
\end{equation}
Keeping only the lowest order in this expansion that is invariant under $U(1)$ symmetry we get the term
$\abs{\nabla\Psi}^2$.

Perhaps the single most important phenomenon of superconductivity from a theoretical stand-point is the fact that it expells
magnetic fields, hence it is clear that any theory that attempts to explain superconductivity needs to have some way the
superconducting order interacts with magnetic fields. The standard way to acheive this is through the recipe of
\emph{minimal coupling}, where the vector potential $\v{A}$ times a constant is subtracted with from any momentum
in the previously neutral theory. Specifically $\v{p}\mapsto \v{p}-q/c\v{A}$ where $q$ is the charge of the particle and
$c$ is the speed of light.
\begin{equation}
    \label{eq:statMech:gradTerms:GLequation}
    f = f_0 - a\abs{\Psi}^2 + b\abs{\Psi}^4 + K\abs{(\nabla+ig\v{A})\Psi}^2 - \int_0^{B_a}\v{M}\cdot\d{\v{B}_a}
\end{equation}<++>
