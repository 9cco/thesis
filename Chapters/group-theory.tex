\chapter{Group Theory}
\label{chap:Group}
%
\noindent In this chapter we will introduce the minimum mathematical framework needed to
understand the phrase \say{\emph{This free energy is the $\Gamma^-_{5u}$ irreducible
representation of the $D_{4h}$ symmetry group}}.

A few words about notation. We will use the semicolon `$;$' in equations as notation for the words `such that', \eg when defining sets.
A colon with a trailing space `$\colon$' is used when defining maps where the symbol representing the mapping itself should be on the left while the the sets
being related or how the elements of the sets are related is on the right of the colon. The colon `$:$' is also used as a shortcut for the
words `applied through its representation to' for when group elements are applied to vectors, where the correct representation to use
for this application should be implicitly understood.

The material in this section is based on the material covered in Refs.~\cite{Inui90} and \cite{NegeleOrland98}, specified for the use in
quantum mechanical theories of unconventional superconducting states.

\section{Irreducible representations}

To know what an \ac{ir} is, let's start with what we mean by a reducible representation.
\begin{defi}
    A matrix representation is \textbf{reducible} if there exists a non-trivial invariant subspace of the vector space of the representation.
\end{defi}
The intuition is then that the vector space of the representation is reducible if a ``smaller'' representation is contained within it. Since
there is a smaller vector space within the vector space of the original representation and this vector space is invariant, it is possible
to define another representation on this smaller vector space, \ie\emph{reduce} the original representation. We have now used the word
``invariant'' a couple of times, so let's define what it means more precisely.

\begin{defi}
    Let $D(g)$ be a representation of the group $G$ on the vector space $V$ such that $D(g)\colon V\rightarrow V$. Then
    a subspace $U\subseteq V$ is \textbf{invariant} if 
    \begin{equation}
        \label{eq:Group:Irr:invariantDef}
        \forall g\in G\quad u\in U \implies D(g)u\in U.
    \end{equation}
\end{defi}
In other words: a vector space is invariant if it is not possible for any vector in it to escape using a representation of any group element.
All representations applied to any vector in the invariant subspace must necessarily land in that same subspace from which it started.

\section{BCS Hilbert Space}
\label{sec:Group:BCSHil}
We define the BCS Hilbert space as the Hilbert space upon which BCS-type potentials operate. Specifically this is a reduced form of the two-state
fermionic product Hilbert space $\hil_2 = \hil\otimes\hil$ where $\hil = \Span\{\ket{\v{k},s}\}$ and we only consider states that have opposite momentum.
Thus this Hilbert space is given by 
\begin{equation}
    \label{eq:Group:BCSHil:space}
    \mathcal{B} = \Span\{\ket{\v{k},s_1}\ket{-\v{k},s_2}\},
\end{equation}
and the identity operator in this space can be written
\begin{equation}
    \label{eq:Group:BCSHil:id}
    \hat{\mathbb{1}} = \sum_{\v{k}\,s_1s_2}\ket{\v{k},s_1}\ket{-\v{k},s_2}\bra{-\v{k},s_2}\bra{\v{k},s_1}.
\end{equation}
Acting on the arbitrary vector $\ket{v}\in\hilB$ with this identity operator, we find that in terms of this basis it can be written
\begin{equation}
    \label{eq:Group:BCSHil:arbitrary}
    \ket{v} = \sum_{\v{k}\,s_1s_2}v_{s_1s_2}(\v{k})\ket{\v{k},s_1}\ket{-\v{k},s_2},
\end{equation}
where
\begin{equation}
    \label{eq:Group:BCSHil:arbitrary:coeff}
    v_{s_1s_2}(\v{k}) = \bra{-\v{k},s_2}\braket{\v{k},s_1}{v}.
\end{equation}
The indices $s_1$ and $s_2$ can take on only two values each, namely $s_1,s_2\in\{\up,\dn\}$. In total there are thus $4$ different realizations
of pairs, $s_1s_2$ e.g. $\up\up$ for $v_{s_1s_2}(\v{k})$. Putting these different realizations of $v_{s_1s_2}(\v{k})$ as elements in a $2\times2$ matrix
we get
\begin{equation}
    \label{eq:Group:BCSHil:arb:mat}
    v_{s_1s_2}(\v{k}) =
    \begin{pmatrix}
        v_{\up\up}(\v{k}) & v_{\up\dn}(\v{k})\\
        v_{\dn\up}(\v{k}) & v_{\dn\dn}(\v{k})
    \end{pmatrix}.
\end{equation}
Any $2\times 2$ matrix can be written in the conventional basis of the $4$ Pauli matrices $\sigma^0 = \mathbb{1}_{2\times 2}$, $\sigma^x$, $\sigma^y$,
and $\sigma^z$. This means that we could write the matrix in Eq.~\eqref{eq:Group:BCSHil:arb:mat}
\begin{equation}
    \label{eq:GroupBCSHil:arb:pauliExp}
    v_{s_1s_2}(\v{k}) = v^0_\v{k}\sigma^0_{s_1s_2} + v^i_\v{k}\sigma^i_{s_1s_2}.
\end{equation}
It is however conventional to factor out a Pauli matrix $i\sigma^y$ to the right in the expansion since this results in nice transformation properties
of the coefficients as we shall see.
With the spin-indices expanded in this basis it is conventional to let the function of $\v{k}$
that is in front of $\sigma^0$ be called $\psi_\v{k}$. The three others are conventionally denoted $d_{\v{k},i}$ which are components of what we call
the $\v{d}$-vector\footnote{Note that a state in $\hilB$ that is described by a $\v{d}$-vector does not necessarily mean that it has $d$-wave symmetry,
which is a symmetry of its $\v{k}$-space argument. It tells us that the state is a spin-triplet state.}. Expanded in this conventional basis
then $v_{s_1s_2}(\v{k})$ takes the form
\begin{equation}
    \label{eq:Group:BCSHil:arb:pauli}
    v_{s_1s_2}(\v{k}) = (\psi_\v{k}\sigma^0_{s_1s'} + d_{\v{k},i}\sigma^i_{s_1s'})i\sigma^y_{s's_2},
\end{equation}
and finally the state $\ket{v}$ can be written
\begin{equation}
    \label{eq:Group:BCSHil:arb:ket}
    \ket{v} = \sum_{\v{k}\,s_1s_2}[(\psi_\v{k}\sigma^0 + \v{d}_\v{k}\cdot\v{\sigma})i\sigma^y]_{s_1s_2}\ket{\v{k},s_1}\ket{-\v{k},s_2}.
\end{equation}
Going one step back and writing out the different combinations of $s_1s_2$ in $v_{s_1s_2}(\v{k})$ as a matrix like we did in
Eq.~\eqref{eq:Group:BCSHil:arb:mat}, but now multiplying out the Pauli matrices in Eq.~\eqref{eq:Group:BCSHil:arb:pauli} we get
\begin{equation}
    \label{eq:Group:BCSHil:arb:mat:Pauli}
    \begin{pmatrix}
        v_{\up\up}(\v{k}) & v_{\up\dn}(\v{k})\\
        v_{\dn\up}(\v{k}) & v_{\dn\dn}(\v{k})
    \end{pmatrix} =
    \begin{pmatrix}
        -d_{\v{k},x}+id_{\v{k},y} & \psi_\v{k}+d_{\v{k},z}\\
        -\psi_\v{k}+d_{\v{k},z} & d_{\v{k},x}+id_{\v{k},y}
    \end{pmatrix}.
\end{equation}
This set of linear relations is easily inverted to yield
\begin{align}
    \label{eq:Group:BCSHil:arb:inverseRel:psi}
    \psi_\v{k} &= \frac{1}{2}(v_{\up\dn}(\v{k}) - v_{\dn\up}(\v{k}))\\
    \label{eq:Group:BCSHil:arb:inverseRel:dx}
    d_{\v{k},x} &= \frac{1}{2}(v_{\dn\dn}(\v{k})-v_{\up\up}(\v{k}))\\
    \label{eq:Group:BCSHil:arb:inverseRel:dy}
    d_{\v{k},y} &= -\frac{i}{2}(v_{\up\up}(\v{k})+v_{\dn\dn}(\v{k}))\\
    \label{eq:Group:BCSHil:arb:inverseRel:dz}
    d_{\v{k},z} &= \frac{1}{2}(v_{\up\dn}(\v{k})+v_{\dn\up}(\v{k})).
\end{align}

Since the space $\hilB$ is fermionic we have the symmetry\footnote{We are here assuming that the state is even in (time) frequency. It is also possible
to have superconducting states that are odd in frequency \cite{Pal17}, however we will not treat that possibility here.}
\begin{equation}
    \label{eq:Group:BCSHil:fermSymm}
    \ket{\v{k},s_1}\ket{-\v{k},s_2} = -\ket{-\v{k},s_2}\ket{\v{k},s_1}.
\end{equation}
Using this symmetry transformation on the basis vectors in the expansion of $\ket{v}$ in Eq.~\eqref{eq:Group:BCSHil:arbitrary}, then renaming indices
and finally equating coefficients term by term, we see that for the coefficients of $\ket{v}$, this symmetry takes the form
\begin{equation}
    \label{eq:GroupBCSHil:fermSymmCoeff}
    v_{s_1s_2}(\v{k}) = -v_{s_2s_1}(-\v{k}).
\end{equation}


\section{Application of group elements}

When we are talking about applying some symmetry transformation to a state, this is synonymous with applying a group element to a vector.
Even more specifically, the `applying' part means that we have some natural representation of the group on the vector space we have defined
states on, and we are using the linear transformation of the representation of the group element to act on the state vector. In this
thesis we will use the notation $g:\ket{\psi}$ to refer to this procedure.

Let $g$ be an arbitrary group element in the symmetry group $G$ and $D$ be a representation of $G$ on the $d$-dimensional vector space $V$.
Let $V$ have a basis $\{\v{b}_i\}_{i=1}^d$. The application of a group element to a basis vector is then defined as
\begin{equation}
    \label{eq:Group:Prod:basisApp}
    g:\v{b}_i = \sum_j\v{b}_jD_{ji}(g).
\end{equation}
The application of a group element to any vector in $V$ then is calculated by expanding the vector in the basis and applying the representation
$D$ to each basis vector separately as a linear transformation:
\begin{equation}
    \label{eq:Group:App:linearTransformApp}
    g:\v{v} = \sum_iv_ig:\v{b}_i.
\end{equation}

\subsection{Active vector transformation}

As $g:$ has been defined in Eq.~\eqref{eq:Group:Prod:basisApp} it is defined in a passive perspective where the transformation is happening
to the basis vectors. It is often useful and sometimes more intuitive to consider
the application of $g$ to a vector $\v{v}$ in the basis $\{\v{b}_i\}$ as an application not
on the vectors themselves but on the expansion coefficients $v_i$ of $\v{v}$ in the basis. This is the active view of the transformation.
Inserting Eq.~\eqref{eq:Group:Prod:basisApp} into Eq.~\eqref{eq:Group:App:linearTransformApp} we get
\begin{equation}
    \label{eq:Group:App:activeTransform}
        g:\v{v} = \sum_iv_i\sum_j\v{b}_jD_{ji}(g) = \sum_iv'_i\v{b}_i,
\end{equation}
where we have defined the transformed coefficients
\begin{equation}
    \label{eq:Group:App:transformedCoeff}
    v'_i = \sum_jD_{ij}(g)v_j.
\end{equation}
From this calculation we see that we can consider the application of $g$ as a transformation of the coefficients of the vector as
\begin{equation}
    \label{eq:Group:App:coeffTransformation}
    g:v_i = \sum_jD_{ij}(g)v_j.
\end{equation}
This defines the active transformation of a vector $v$ by a group element $g$.

\subsection{Representation on product spaces}

The product space $V\otimes V$ then has a basis $\{\v{b}_i\v{b}_j\}_{i,j=1}^d$. A derived representation can be constructed from $D$ on this product space
called the product representation $D^{(D\times D)}(g)$. This is defined through its application on the basis by
\begin{equation}
    \label{eq:Group:Prod:prodBasisApp}
    g:\v{b}_i\v{b}_j = \sum_{kl}\v{b}_k\v{b}_l\big[D^{(D\times D)}(g)\big]_{kl,ij} = \sum_{kl}\v{b}_k\v{b}_lD_{ki}(g)D_{lj}(g).
\end{equation}

To apply group theory to physical problems, we need to know how the objects we are working with in the physical theory transform under group elements.
The most important vector space in quantum mechanics is arguably the Hilbert space where particle states are determined by a momentum and spin
quantum number. Each quantum number has its own vector space defined by the basis vectors $\ket{\v{k}}$ and $\ket{s}$ in the Dirac notation. The
combination of both quantum numbers in the description of a particle state then gives a state in the product space of these vector spaces.
A basis for this space is given by the vectors $\ket{\v{k},s} = \ket{\v{k}}\ket{s}$. Given $\v{k}\in\mathbb{R}^d$ and $s\in\{\up,\dn\}$,
these basis vectors transform according to the product representation of the representations on each vector space, given by
\begin{equation}
    \label{eq:Group:App}
    g : \ket{\v{k}',s'} = \sum_{\v{k}s}\ket{\v{k},s}D^{(\v{k}\times s)}_{\v{k}s;\,\v{k}'s'} = \sum_{\v{k}s}\ket{\v{k},s}D_{g\;ss'}\delta_{\v{k},g:\v{k}'},
\end{equation}
under a group element $g$. Here $g\v{k}'$ means application of $g$ to the vector $\v{k}'$ through the standard representation of $g$ in $\mathbb{R}^d$.
$D_{g\;ss'}$ is a representation on the spin-up spin-down vector space given by the matrix
\begin{equation}
    \label{eq:Group:Prod:spinRepMatrix}
    D_{g\;ss'} = \sigma^0_{ss'}\cos(\phi/2) - i\hat{\v{u}}\cdot\v{\sigma}_{ss'}\sin(\phi/2),
\end{equation}
where $\hat{\v{u}}$ is the rotation axis unity vector, while $\phi$ is the angle that defines the proper rotation associated with $g$. $\v{\sigma}$ is
the vector notation for the $3$ Pauli matrices and $\sigma^0_{ss'}=\delta_{ss'}$ \cite{Sergi04,Merzbacher99}.
%, \cite[][Appendix A]{Schober16}.

In the BCS Hilbert space which we discussed in more detail in Section~\ref{sec:Group:BCSHil},
the basis vectors are outer products of the momentum spin basis vectors with opposite momentum: $\{\ket{\v{k},s_1}\ket{-\v{k},s_2}\}$. The product
representation on this vector space then transforms the basis vectors according to
\begin{equation}
    \label{eq:Group:Prod:BCSProdRep}
    g:\ket{\v{k}',s_1'}\ket{-\v{k}',s_2'} = \sum_{\v{k}\,s_1s_2}\ket{\v{k},s_1}\ket{-\v{k},s_2}D^{(D\times D)}_{\v{k}\,s_1s_2;\,\v{k}'\,s_1's_2'}(g),
\end{equation}
where
\begin{equation}
    \label{eq:Group:Prod:BCSProdMatrix}
    D^{(D\times D)}_{\v{k}\,s_1s_2;\,\v{k}'\,s_1's_2'}(g) = D_{g\,s_1s_1'}\delta_{\v{k},g:\v{k}'}D_{g\,s_2s_2'}\delta_{-\v{k},g:-\v{k}'}.
\end{equation}
Since group representations on $\v{k}$ is a linear transformation then $\delta_{-\v{k},g:-\v{k}'} = \delta_{\v{k},g:\v{k}'}$, such that the last
Kronecker delta function becomes superfluous.

\subsection{Representation on $\psi$-$\v{d}$ functions}
\label{sec:Group:pdRep}

The coefficients of the basis expansion of a vector in the BCS Hilbert space were given in the conventional $\psi$-$\v{d}$ notation in 
Eq.~\eqref{eq:Group:BCSHil:arb:ket}. Taking the active view of group transformations we can say that the expansion coefficients of arbitrary states
in the BCS Hilbert space $\hilB$ transforms like the $v_i$ in Eq.~\eqref{eq:Group:App:coeffTransformation} but where now the representation
matrix $D$ is given by the matrix $D^{(D\times D)}_{\v{k}\,s_1s_2;\,\v{k}'\,s_1's_2'}$ above in Eq.~\eqref{eq:Group:Prod:BCSProdMatrix}. Written
out then, the coefficients transform according to
\begin{equation}
    \label{eq:Group:pdRep:coeffTrans}
    g:v_{s_1s_2}(\v{k}) = \sum_{\v{k}'\,s_1's_2'}D^{(D\times D)}_{\v{k}\,s_1s_2;\,\v{k}'\,s_1's_2'}v_{s_1's_2'}(\v{k}').
\end{equation}
Let now $\ket{v}$ be a state that is even in space, meaning that its expansion only consists of coefficients $\psi(\v{k})$ in the $\psi$-$\v{d}$
notation. Then we see from Eq.~\eqref{eq:Group:BCSHil:arb:mat:Pauli} that $\psi(\v{k})$ can be written $\psi(\v{k}) = v_{\up\dn}(\v{k})$.
The transformation properties of $\psi(\v{k})$ are thus given by
\begin{equation}
    \label{eq:Group:pdRep:psiRep}
    \begin{split}
        g:\psi(\v{k}) &= g:v_{\up\dn}(\v{k}) = \sum_{\v{k}'\,s_1's_2'}D^{(D\times D)}_{\v{k}\,\up\dn;\,\v{k}'\,s_1's_2'}\psi(\v{k}')(i\sigma^y)_{s_1's_2'}\\
        &=\psi(g^{-1}:\v{k})\big(i\sigma^y)_{\up\dn} = \psi(g^{-1}:\v{k}).
    \end{split}
\end{equation}
In this calculation we inserted the expression of $D^{(D\times D)}_{\v{k}\,s_1s_2;\,\v{k}'\,s_1's_2'}$ in Eq.~\eqref{eq:Group:Prod:BCSProdMatrix} and used
the equation $D_gi\sigma^y D_g^\trans = i\sigma^y$, where $D_g$ are the spin representation matrices given in Eq.~\eqref{eq:Group:Prod:spinRepMatrix}.

To find the transformation properties of $\v{d}_\v{k}$ the principle is the same as above for $\psi(\v{k})$ but the calculations become more involved.
We assume that the spin-momentum basis expansion of state $\ket{v}$ consists of only odd coefficients so that
$v_{s_1s_2}(\v{k}) = \v{d}_\v{k}\cdot(\v{\sigma}i\sigma^y)_{s_1s_2}$. Inserting this into the active transformation of the coefficients of $\ket{V}$ in
Eq.~\eqref{eq:Group:pdRep:coeffTrans} and also inserting the expression for $D^{(D\times D)}_{\v{k}\,s_1s_2;\,\v{k}'\,s_1's_2'}$ as we did before yields
\begin{equation}
    \label{eq:Group:pdRep:oddCoeffTrans}
    \begin{split}
        g:v_{s_1s_2}(\v{k}) &= \sum_{\v{k}'\,s_1's_2'}\delta_{\v{k},g:\v{k}'}D_{g\,s_1s_1'}D_{g\,s_2s_2}\v{d}_{\v{k}'}\cdot(\v{\sigma}i\sigma^y)_{s_1's_2'}\\
        &= \sum_s\Big(D_g\v{\sigma}\sigma^yD_g^\trans\sigma^y\Big)_{s_1s}\cdot\v{d}_{g^{-1}:\v{k}}\;i\sigma^y_{ss_2}.
    \end{split}
\end{equation}
Since a group transformation (aka. the linear transformation given by a group representation) can not bring a state that was even to be odd or vice versa,
then the resulting state given by the transformed coefficients $g:v_{s_1s_2}(\v{k})$ has to remain odd, and thus they can be expanded in terms of a new
$\v{d}'$ such that 
\begin{equation}
    \label{eq:Group:pdRep:oddCoeffTrans:transExpansion}
    g:v_{s_1s_2}(\v{k}) = \sum_s\v{d}'_{g^{-1}:\v{k}}\cdot\v{\sigma}_{s_1s}i\sigma^y_{ss_2}.
\end{equation}
Having expanded both sides of the transformed coefficients with a common factor $i\sigma^y$ to the right we can equate the remaining $2\times2$ spin
matrices which gives an expression for $\v{d}'_{g^{-1}:\v{k}}\cdot\v{\sigma}$ by comparing Eq.~\eqref{eq:Group:pdRep:oddCoeffTrans} and
Eq.~\eqref{eq:Group:pdRep:oddCoeffTrans:transExpansion}.
Furthermore, using the anti-commutation property $\{\sigma^i,\sigma^j\} = 2\delta_{ij}\sigma^0$ of Pauli matrices we find that
\begin{equation}
    \label{eq:Group:pdRep:projectingD}
    d'_{\v{k},i} = \frac{1}{4}\tr\big(\{\sigma^i, \v{d}'_\v{k}\cdot\v{\sigma}\big\}).
\end{equation}
Inserting the expression for $\v{d}'_{g^{-1}:\v{k}}\cdot\v{\sigma}$ and the full expression of the $SU(2)$ spin-representation matrices $D_g$, which
is found in  Eq.~\eqref{eq:Group:Prod:spinRepMatrix}, yields after some algebra
\begin{equation}
    \label{eq:Group:pdRep:transformedD}
    \begin{split}
        d'_{g^{-1}:\v{k},i} &= \frac{1}{4}\tr\big(\{\sigma^i, \v{d}_{g^{-1}:\v{k}}\cdot D_g\v{\sigma}\sigma^yD_v^\trans\sigma^y\}\big)\\
        &= R_{ij}(\hat{\v{u}},\phi)d_{g^{-1}:\v{k},j},
    \end{split}
\end{equation}
where we have defined the matrix
\begin{equation}
    \label{eq:Group:pdRep:rotMatrixIndices}
    \begin{split}
        &R_{ij}(\hat{\v{u}},\phi) = \delta_{ij}\cos\phi + \hat{u}_i\hat{u}_j(1-\cos\phi)-\epsilon_{ijk}\hat{u}_k\sin\phi\\
        & = 
        \left(\begin{smallmatrix}
            \cos\phi + \hat{u}_x^2(1-\cos\phi) & \hat{u}_x\hat{u}_y(1-\cos\phi)-\hat{u}_z\sin\phi & \hat{u}_x\hat{u}_z(1-\cos\phi)+\hat{u}_y\sin\phi\\
            \hat{u}_y\hat{u}_x(1-\cos\phi)+\hat{u}_z\sin\phi & \cos\phi+\hat{u}_y^2(1-\cos\phi) & \hat{u}_y\hat{u}_z(1-\cos\phi)-\hat{u}_x\sin\phi\\
            \hat{u}_z\hat{u}_x(1-\cos\phi)-\hat{u}_y\sin\phi & \hat{u}_z\hat{u}_y(1-\cos\phi)+\hat{u}_x\sin\phi & \cos\phi+\hat{u}_z^2(1-\cos\phi)
        \end{smallmatrix}\right).
    \end{split}
\end{equation}
This matrix is in fact the rotation matrix of a vector in $\mathbb{R}^3$ by an angle $\phi$ about a unit vector $\hat{\v{u}}$. Since the coefficients
 of an odd state $\ket{v}$ are fully determined by the vector $\v{d}$, their transformation can be regarded just as a transformation of $\v{d}$ itself
 which thus takes the form
\begin{equation}
    \label{eq:Group:pdRep:dTransformation}
    g:d_{\v{k},i} = R_{ij}(\hat{\v{u}},\phi)d_{g^{-1}:\v{k},j},
\end{equation}
where $\hat{\v{u}}$ and $\phi$ give the unit vector and angle respectively, of the proper rotation associated with $g$. The conclusion is thus that
$\v{d}$ transforms as a vector by the proper rotation associated with $g$.


\subsection{Representation on ladder operators}
\label{sec:Group:App:Ladder}

The fermionic creation and annihilation operators $c^\dagger_{\v{k},s}$ and $c_{\v{k},s}$, which we will collectively refer to as
$c^{(\dagger)}_{\v{k},s}$, are second-quantized operators that act on multi-particle
states in a fermionic Fock space \cite{NegeleOrland98}. To properly define how a group element $g$ transforms these operators we thus should
strictly speaking first derive the representation $Q(g)$ of $g$ on $N$-particle states $\ket{\v{k}_1,s_1}\land\ldots\land\ket{\v{k}_N,s_N}$ for
arbitrary $N$ and then use the relation $g:c^{(\dagger)}_{\v{k},s} = Q(g)c^{(\dagger)}_{\v{k},s}Q(g)^{-1}$ to then derive their transformation properties.
However, luckily, a shortcut is possible because the $Q(g)$ representation is connected with how $g$ acts on the single-particle basis
$\{\ket{\v{k},s}\}$ through the relationship
\begin{equation}
    \label{eq:Group:App:Ladder:quantizedRepToSingleParticleBasis}
    Q(g)c^{(\dagger)}_{\v{k},s}Q(g)^{-1} = c^{(\dagger)}(g:\ket{\v{k},s}).
\end{equation}
In this notation we treat the creation and annihilation operators as respectively linear and antiunitary functions of the states that they create or
annihilate. If the matrix-components of the representation of $g$ on the single-particle basis is denoted $D_{\v{k}s;\v{k}'s'}^{(\v{k}\times s)}$ then
we thus have the transformation property
\begin{equation}
    \label{eq:Group:App:Ladder:generalTransformationBehavior}
    g:c^{(\dagger)}_{\v{k},s} = \sum_{\v{k}'s'}c^{(\dagger)}_{\v{k}',s'}\big[D_{\v{k}'s';\v{k}s}^{(\v{k}\times s)}\big]^{(\ast)},
\end{equation}
where the matrix-elements are complex conjugated only if we are transforming an annihilation operator.

Given that $g$ is an element of some point group such that it can be decomposed into a rotation and a parity inversion, then we may use the
representation on $\{\ket{\v{k},s}\}$ given in Eq.~\eqref{eq:Group:App} such that
\begin{equation}
    \label{eq:Group:App:Ladder:pointGroupTransformation}
    g:c^{(\dagger)}_{\v{k},s} = \sum_{s'}c^{(\dagger)}_{g:\v{k},s'}D_{g\;s's}^{(\ast)},
\end{equation}
where the matrix $D_{g\;ss'}$ is defined in Eq.~\eqref{eq:Group:Prod:spinRepMatrix}. As an example, a simple parity transformation $g=\hat{P}$, then
transform the operators according to $\hat{P}:c^{(\dagger)}_{\v{k},s} = c^{(\dagger)}_{-\v{k},s}$.

We will later need the translation transformation $g=\v{R}$ as well. In real space it is defined by $\v{R}:\ket{\v{r},s} = \ket{\v{r}+\v{R},s}$
and given by linear transformation $L(\v{R}) = e^{i\v{R}\cdot\hat{\v{p}}/\hbar}$ where $\hat{\v{p}}$ is the momentum operator%
\footnote{The form of the linear transformation $L(\v{R})$ follows directly from the multivariate Taylor expansion.} $\hat{\v{p}} = \hbar\nabla/i$.
It then follows by a Fourier-transformation that the fermionic operators transform according to
\begin{equation}
    \label{eq:Group:App:Ladder:translation}
    \v{R}:c^{(\dagger)}_{\v{k},s} = \big(e^{-i\v{R}\cdot\v{k}}\big)^{(\ast)}c^{(\dagger)}_{\v{k},s}.
\end{equation}

Finally we discuss the transformation of time-reversal. This operation is traditionally denoted $\Theta$ and consists of flipping the sign of
time, which implies flipping the spin and momentum. Time-reversal is special in that it is an antiunitary, \ie it transforms all linear coefficients
to their complex conjugate. Any representation is
because of this, often factorized into a normal linear part and a complex-conjugation operator $\hat{K}$. From its action on the
spin-momentum basis vectors $\ket{\v{k},s}$, time-reversal transforms the fermionic creation and annihilation operators according to
\begin{equation}
    \label{eq:Group:App:Ladder:timeReversal}
    \Theta:c^{(\dagger)}_{\v{k},s} = \sum_{s'}c^{(\dagger)}_{-\v{k},s'}(-i\sigma_y)_{s's}.
\end{equation}
Note that the matrix $-i\sigma_y$ is not complex conjugated for the fermionic annihilation operators because of the antiunitarity of $\Theta$.


\section{Single-particle Hamiltonian symmetries}

Any fermionic single-particle operator can be written in the spin-momentum basis as
\begin{equation}
    \label{eq:Group:Single:generalHamiltonian}
    \hat{H} = \sum_{\v{k}\v{k}'ss'}H^{ss'}_{\v{k}\v{k}'}c^{\dagger}_{\v{k},s}c_{\v{k}'s'}.
\end{equation}
In this section we will employ common finite symmetry-transformations discussed in Section~\ref{sec:Group:App:Ladder} and see how this reduces the degrees
of freedom in $\hat{H}$ so that it can be written in a more concise form. To begin we reduce the notational burden by expanding the spin-matrix
elements in separate coefficients in the Pauli-matrix basis just as was done for the BCS Hilbert space vector coefficients $v_{s_1s_2}(\v{k})$ in
Eq.~\eqref{eq:GroupBCSHil:arb:pauliExp} such that
\begin{equation}
    \label{eq:Group:Single:pauliMatrices}
    \hat{H} = \sum_{\v{k}\v{k}'ss'}\big[\xi_{\v{k}\v{k}'}\sigma^0 + \v{\gamma}_{\v{k}\v{k}'}\cdot\v{\sigma}\big]_{ss'}c^{\dagger}_{\v{k},s}c_{\v{k}'s'}.
\end{equation}

The transformation of a quantum mechanical operator such as the Hamiltonian is defined by $g:\hat{H} = Q(g)\hat{H}Q(g)^{-1}$ for some many-body
representation $Q(g)$. Inserting the form of $\hat{H}$ in Eq.~\eqref{eq:Group:Single:pauliMatrices} as a single-particle operator in the
spin-momentum basis yields
\begin{equation}
    \label{eq:Group:Single:generalTransformedHamiltonian}
    g:\hat{H} = \sum_{\v{k}\v{k}'ss'}\big[\xi_{\v{k}\v{k}'}\sigma^0 + \v{\gamma}_{\v{k}\v{k}'}\cdot\v{\sigma}\big]_{ss'}(g:c^\dagger_{\v{k},s})(g:c_{\v{k}'s'}),
\end{equation}
after inserting $Q(g)^{-1}Q(g)$ between the creation and annihilation operator. We can now use the transformation properties of these operators presented
in Section~\ref{sec:Group:App:Ladder} to derive the transformation properties of the Hamiltonian.

For a translation invariant system we need a translation invariant Hamiltonian. Using the transformation of $c^{(\dagger)}$ in
Eq.~\eqref{eq:Group:App:Ladder:translation} under a translation $\v{R}$, we get that the coefficients $\xi_{\v{k}\v{k}'}$ and $\v{\gamma}_{\v{k}\v{k}'}$
has to satisfy the equation
\begin{equation}
    \label{eq:Group:Single:Translation:condition}
    (e^{i\v{R}\cdot(\v{k}'-\v{k})}-1)\big[\xi_{\v{k}\v{k}'}\sigma^0 + \v{\gamma}_{\v{k}\v{k}'}\cdot\v{\sigma}\big] = 0.
\end{equation}
Since the sigma-matrices are linearly independent and the translation $\v{R}$ is arbitrary, it follows that both $\xi_{\v{k}\v{k}'}$ and
$\v{\gamma}_{\v{k}\v{k}'}$ must vanish whenever $\v{k}\neq\v{k}'$, \ie these coefficients must be diagonal in $\v{k}$. The Hamiltonian can then be written on the
reduced form
\begin{equation}
    \label{eq:Group:Single:Translation:Hamiltonian}
    \hat{H} = \sum_{\v{k}ss'}\big[\xi_\v{k}\sigma^0 + \v{\gamma}_\v{k}\cdot\v{\sigma}\big]_{ss'}c^{\dagger}_{\v{k},s}c_{\v{k},s'}.
\end{equation}

In the vast majority of cases, the system is Hermitian such that the Hamiltonian is self-adjoint and has real eigenvalues. Enforcing the condition
$\hat{H}^\dagger=\hat{H}$ on the translationally invariant Hamiltonian in Eq.~\eqref{eq:Group:Single:Translation:Hamiltonian} yields the equation
\begin{equation}
    \label{eq:Group:Single:Hermitian:condition}
    (\xi_\v{k}^\ast-\xi_{\v{k}})\sigma^0 + (\v{\gamma}^\ast_\v{k}-\v{\gamma}_\v{k})\cdot\v{\sigma} = 0,
\end{equation}
by using the self-adjoint property of the Pauli-matrices. This implies further by the linear independence of the sigma-matrices that the coefficients $\xi_\v{k}$
and $\v{\gamma}_\v{k}$ are self-conjugate and hence real.

Time-reversal symmetry of the system implies that $\hat{H}$ should be invariant under the transformation $\Theta$, which transforms fermionic creation and annihilation
operators according to Eq.~\eqref{eq:Group:App:Ladder:timeReversal}. Transforming these operators in the translationally invariant Hamiltonian in
Eq.~\eqref{eq:Group:Single:Translation:Hamiltonian} and remembering that $Q(\Theta)$ is an antilinear operator we get the transformed Hamiltonian
\begin{equation}
    \label{eq:Group:Single:TRS:transformedHamiltonian}
    \Theta:\hat{H} = \sum_{\v{k}ss'}\big[\xi_{-\v{k}}^\ast\sigma^0 - \v{\gamma}_{-\v{k}}^\ast\cdot\v{\sigma}\big]_{ss'}c^\dagger_{\v{k},s}c_{\v{k},s'}.
\end{equation}
For the Hamiltonian to be invariant under time-reversal symmetry, the coefficients thus have to satisfy $\xi_\v{k} = \xi_{-\v{k}}^\ast$ and
$\v{\gamma}_\v{k} = -\v{\gamma}_{-\v{k}}^\ast$.

Finally we look at the transformation of the Hamiltonian under point-group elements $g$. Using the transformation-property of the fermionic creation and
annihilation operators in Eq.~\eqref{eq:Group:App:Ladder:pointGroupTransformation}, then the translation-invariant Hamiltonian in
Eq.~\eqref{eq:Group:Single:Translation:Hamiltonian} transforms according to
\begin{equation}
    \label{eq:Group:Single:Point:transformedHamiltonian}
    g:\hat{H} = \sum_{\v{k}ss'}\big[\xi_{g^{-1}:\v{k}}\sigma^0 + \v{\gamma}_{g^{-1}:\v{k}}\cdot D_g\v{\sigma}D_g^\dagger\big]_{ss'}c^\dagger_{\v{k},s}c_{\v{k},s'}.
\end{equation}
Using the form of the spin-rotation matrix $D_g$ in Eq.~\eqref{eq:Group:Prod:spinRepMatrix} we find that
\begin{equation}
    \label{eq:Group:Single:Point:sigmaTransform}
    \v{\gamma}_{g^{-1}:\v{k}}\cdot D_g\v{\sigma}D_g^\dagger = \v{\gamma}'_{g^{-1}:\v{k}}\cdot\v{\sigma},
\end{equation}
for the transformed vector
\begin{equation}
    \label{eq:Group:Single:point:transformedVector}
    \v{\gamma}'_{\v{k}} = R(\hat{\v{u}},\phi)\v{\gamma}_\v{k},
\end{equation}
where $R(\hat{\v{u}},\phi)$ is the $3\times3$ matrix representation of the proper-rotation associated with $g$ given in
Eq.~\eqref{eq:Group:pdRep:rotMatrixIndices}. Symmetry of $\hat{H}$ under point-group transformation thus implies the symmetry relations $\xi_{g:\v{k}} = \xi_{\v{k}}$
and $R(\hat{\v{u}},\phi)\v{\gamma}_{g^{-1}:\v{k}} = \v{\gamma}_\v{k}$ on the coefficients of the Hamiltonian.


\section{Projection Operators}
\label{sec:Group:Pro}

Let us assume that we are in a vector space $V$ that can be divided into possibly several different \ac{ir}s $D^{(\alpha)}$ of some
symmetry group $G$. Further, let the basis vectors of these \ac{ir}s be denoted by $\v{b}^{(\alpha)}_m$ where $m$ thus counts the number
of basis vectors in each \irr Then an arbitrary vector $\v{f}\in V$ can be written in terms of these basis vectors as
\begin{equation}
    \label{eq:Group:Irr:Pro:basisDecomp}
    \v{f} = \sum_\alpha\sum_m c_m^{(\alpha)}\v{b}^{(\alpha)}_m.
\end{equation}

A projection operator can be used to extract any combination of constant $c_m^{(\alpha)}$ multiplied by a basis vector $\v{b}^{(\alpha)}_n$, where
$m$ and $n$ can in general be different. Denoting the projection operator that picks out the $m$th constant multiplied by the $l$th basis vector
in the \irr $\beta$ of the expansion of $\v{f}$: $P^{(\beta)}_{lm}$, then
\begin{equation}
    \label{eq:Group:Irr:Pro:generalProApplication}
    P^{(\beta)}_{lm}\v{f} = c_m^{(\beta)}\v{b}_l^{(\beta)}.
\end{equation}
This is extremely useful in finding a basis for the \ac{ir}s.
To achieve this, the projection operator is defined as
\begin{equation}
    \label{eq:Group:Irr:Pro:proOpDef}
    P^{(\beta)}_{l,m} = \frac{d_\beta}{\abs{G}}\sum_{g\in G}D_{lm}^{(\beta)}(g)^\ast g:,
\end{equation}
where $d_\beta$ is the dimension of \irr $\beta$, $D_{lm}^{(\beta)}(g)$ is the $lm$ element of the matrix representation of the group element $g$ and
finally we have used the notation $g:$ to denote application on vectors by the relevant representation. An example is the application of $g$ to the
basis vectors $\v{b}_m^{(\alpha)}$. Since the relevant representation of $g$ in this case is the \ac{ir} for which $\v{b}_m^{(\alpha)}$ is
a basis vector, the application becomes
\begin{equation}
    \label{eq:Group:Irr:Pro:gApplication}
    g : b_m^{(\alpha)} = \sum_nb_n^{(\alpha)}D_{nm}^{(\alpha)}(g).
\end{equation}

Usually, the full generality of the
projection operators $P^{(\beta)}_{l,m}$ isn't needed and it suffices to consider the diagonal projection operators
$P^{(\beta)}_{l,l} \equiv P^{(\beta)}_l$ or indeed their sum, in which case the resulting operator can be written only in terms of the \irr characters
$\chi^{(\alpha)}(g)$ since
\begin{equation}
    \label{eq:Group:Irr:Pro:charPro}
    P^{(\beta)}\equiv\sum_lP^{(\beta)}_l = \frac{d_\beta}{\abs{G}}\sum_{g\in G}\sum_lD_{ll}^{(\beta)}(g)^\ast g: = \frac{d_\beta}{\abs{G}}\sum_{g\in G}\chi^{(\beta)}(g)^\ast g:.
\end{equation}


\section{Symmetries of the Square Lattice}
\label{sec:Group:Symm}

The symmetry group of the square lattice is denoted $C_{4v}$ in the Sch\"onflies notation. It contains $8$ elements in total:
\begin{itemize}
    \item[$e$:] The identity element (do nothing),
    \item[$C_4$:] Rotation by $90^\circ$ in the positive direction (ccw),
    \item[$C_4^{-1}$:] Rotation by $90^\circ$ in the negative direction (cw),
    \item[$C_4^2$:] Rotation by $180^\circ$,
    \item[$\sigma_x$:] Mirror about the $zy$-plane,
    \item[$\sigma_y$:] Mirror about the $zx$-plane,
    \item[$\sigma_{d_1}$:] Mirror about the downwards diagonal plane\footnote{We are assuming the western bias of left-to-right movement
        here. More precisely it is the plane containing the $z$ axis and the axis $y=-x$},
    \item[$\sigma_{d_2}$:] Mirror about the upwards diagonal plane.
\end{itemize}
This results in the group multiplication table in Table~\ref{tab:Group:Symm:multTab}. We can check that this is correct by
performing the group transformations in the top row followed by the one in the left column and seeing that this results
in the group transformations where these two intersect. As an example consider the vector $(x,y)^\trans$. Transforming this
by the $90^\circ$ counterclockwise rotation $C_4$ we get $(-y,x)^\trans$. Then mirroring this result about the $yz$-plane
yields $(y,x)^\trans$. We now realize that this is the same as mirroring the original vector about the axis $y=x$, hence
$\sigma_xC_4=\sigma_{d_2}$ as the multiplication table says.
\begin{table}
    \centering
    \caption{Group multiplication table of the group $C_{4v}$.}
    \begin{tabular}{c|cccc|cccc}
        & $e$ & $C_4$ &$C_4^2$ & $C_4^{-1}$ & $\sigma_x$ & $\sigma_y$ & $\sigma_{d_1}$ & $\sigma_{d_2}$\\ \hline
        $e$ & $e$ & $C_4$ & $C_4^2$ & $C_4^{-1}$ & $\sigma_x$ & $\sigma_y$ & $\sigma_{d_1}$ & $\sigma_{d_2}$\\
        $C_4$ & $C_4$ & $C_4^2$ & $C_4^{-1}$ & $e$ & $\sigma_{d_1}$ & $\sigma_{d_2}$ & $\sigma_y$ & $\sigma_x$\\
        $C_4^2$ & $C_4^2$ & $C_4^{-1}$ & $e$ & $C_4$ & $\sigma_y$ & $\sigma_x$ & $\sigma_{d_2}$ & $\sigma_{d_1}$\\
        $C_4^{-1}$ & $C_4^{-1}$ & $e$ & $C_4$ & $C_4^2$ & $\sigma_{d_2}$ & $\sigma_{d_1}$ & $\sigma_x$ & $\sigma_y$\\ \hline
        $\sigma_x$ & $\sigma_x$ & $\sigma_{d_2}$ & $\sigma_y$ & $\sigma_{d_1}$ & $e$ & $C_4^2$ & $C_4^{-1}$ & $C_4$\\
        $\sigma_y$ & $\sigma_y$ & $\sigma_{d_1}$ & $\sigma_x$ & $\sigma_{d_2}$ & $C_4^2$ & $e$ & $C_4$ & $C_4^{-1}$\\
        $\sigma_{d_1}$ & $\sigma_{d_1}$ & $\sigma_x$ & $\sigma_{d_2}$ & $\sigma_y$ & $C_4$ & $C_4^{-1}$ & $e$ & $C_4^2$\\
        $\sigma_{d_2}$ & $\sigma_{d_2}$ & $\sigma_y$ & $\sigma_{d_1}$ & $\sigma_x$ & $C_4^{-1}$ & $C_4$ & $C_4^2$ & $e$
    \end{tabular}
    \label{tab:Group:Symm:multTab}
\end{table}

\subsection{Conjugation classes}

The conjugation classes of a group is the sets of group elements that is conjugate to each other, meaning that there exists a group element $g$ such
that $gAg^{-1} = B$ between conjugate elements $A$ and $B$. Since conjugation is an equivalence relation it subdivides the group elements
exactly into conjugation classes. The conjugation classes of the group $C_{4v}$ are $e = \{e\}$, $2C_4 = \{C_4,C_4^{-1}\}$, $C_4^2 = \{C_4^2\}$,
$2\sigma_d = \{\sigma_{d_1},\sigma_{d_2}\}$ and $2\sigma_v = \{\sigma_x,\sigma_y\}$. The character $\chi^\Gamma(g)$ of a representation $\Gamma$ is the trace of the representation
matrix $D^\Gamma(g)$ of a certain group element $g$. Since the trace is cyclic, then for conjugate group elements $A$ and $B$
\begin{equation}
    \label{eq:Group:Symm:conjugateCharacter}
    \begin{split}
        \chi^\Gamma(B) &= \tr\Big(D^\Gamma(g)D^\Gamma(A)D^\Gamma(g^{-1})\Big)\\
        &= \tr\Big(D^\Gamma(g^{-1})D^\Gamma(g)D^\Gamma(A)\Big)\\
        &= \tr\Big(D^\Gamma(g^{-1}g)D^\Gamma(A)\Big) = \chi^\Gamma(A).
    \end{split}
\end{equation}
This means that the representation of all group elements in a certain conjugation class has the same character. 

It is useful to list the characters of the different conjugation classes in a table of the different \ac{ir}s of a group.
This is because the number of conjugation classes of a finite group is the same as the number of \ac{ir}s of that group.
This table is known as the character table of the group. The character table of
the group $C_{4v}$ is shown in Table~\ref{tab:Group:Symm:characterTable}. This table can be derived without knowing the details of the irreducible
representations but instead using character relations from basic group theory\footnote{Many of these relations are derived from the great orthogonality
theorem which can be found \eg in \cite{Inui90}.}. 

\subsection{Irreducible representations}

The dimensionality of the \irr can be found in Table~\ref{tab:Group:Symm:characterTable} by looking up the first column, i.e. the column giving
the character of the conjugation class $\{e\}$.
Since the group element $e$ maps to the identity transformation in all representations then its
trace gives the dimension of the representation. From the table we see that all the \ac{ir}s are $1$-dimensional except for
$\Gamma_5$ which is $2$ dimensional.
\begin{table}
    \centering
    \caption{Character table of the group $C_{4v}$.}
    \begin{tabular}{c|ccccc}
        $C_{4v}$ & $e$ & $2C_4$ & $C_4^2$ & $2\sigma_v$ & $2\sigma_d$\\ \hline
        $\Gamma_1$ & $1$ & $1$ & $1$ & $1$ & $1$\\
        $\Gamma_2$ & $1$ & $1$ & $1$ & $-1$ & $-1$\\
        $\Gamma_3$ & $1$ & $-1$ & $1$ & $1$ & $-1$\\
        $\Gamma_4$ & $1$ & $-1$ & $1$ & $-1$ & $1$\\
        $\Gamma_5$ & $2$ & $0$ & $-2$ & $0$ & $0$
    \end{tabular}
    \label{tab:Group:Symm:characterTable}
\end{table}
All the $1$-dimensional \ac{ir} matrices are then completely determined by the character table since they are just given by
the characters themselves, \ie $D^{\Gamma_2}(\sigma_x) = -1$.

To find a $2$-dimensional representation of $C_{4v}$ we can imagine how a normal $2$D vector $(x,y)^\trans\in\mathbb{R}^2$ behaves under its transformations.
We take again the example of a counterclockwise rotation by $90^\circ$ which transforms a vector
\begin{equation}
    \label{eq:Group:Symm:C4transform}
    C_4:\begin{pmatrix}
        x\\
        y
    \end{pmatrix} = 
    \begin{pmatrix}
        -y\\
        x
    \end{pmatrix} = 
    \begin{pmatrix}
        0 & -1\\
        1 & 0
    \end{pmatrix}
    \begin{pmatrix}
        x\\
        y
    \end{pmatrix}.
\end{equation}
Obviously, the matrix 
\begin{equation}
    \label{eq:Group:Symm:C4G5D}
    D^{(\Gamma_5)}(C_4) = 
    \begin{pmatrix}
        0 & -1\\
        1 & 0
    \end{pmatrix},
\end{equation}
is the representation matrix of a two-dimensional representation of $C_4$. Continuing in this way for all the group transformations yields the matrices
\begin{subequations}
    \label{eq:Group:Symm:G5D}
\begin{align}
    D^{(\Gamma_5)}(e) &=
    \begin{pmatrix}
        1 & 0\\
        0 & 1
    \end{pmatrix}, & D^{(\Gamma_5)}(C_4) &=
    \begin{pmatrix}
        0 & -1\\
        1 & 0
    \end{pmatrix},\\
    D^{(\Gamma_5)}(C_4^{-1}) &=
    \begin{pmatrix}
        0 & 1\\
        -1 & 0
    \end{pmatrix}, & D^{(\Gamma_5)}(C_4^2) &= 
    \begin{pmatrix}
        -1 & 0\\
        0 & -1
    \end{pmatrix},\\
    D^{(\Gamma_5)}(\sigma_x) &= 
    \begin{pmatrix}
        -1 & 0\\
        0 & 1
    \end{pmatrix}, & D^{(\Gamma_5)}(\sigma_y) &=
    \begin{pmatrix}
        1 & 0\\
        0 & -1
    \end{pmatrix},\\
    D^{(\Gamma_5)}(\sigma_{d_1}) &=
    \begin{pmatrix}
        0 & -1\\
        -1 & 0
    \end{pmatrix}, & D^{(\Gamma_5)}(\sigma_{d_2}) &=
    \begin{pmatrix}
        0 & 1\\
        1 & 0
    \end{pmatrix}.
\end{align}
\end{subequations}
Taking the trace of these matrices, we see that this representations characters are the same as the ones for the \irr $\Gamma_5$ in the character table
(Table~\ref{tab:Group:Symm:characterTable}). This implies that $\sum_g|\chi^{\Gamma_5u}(g)|^2 = |C_{4_v}|$ which implies in turn that the representation
given by the matrices in Eq.~\eqref{eq:Group:Symm:G5D} is irreducible. Thus this is indeed the $\Gamma_5$ \ac{ir} as advertised,
and we have completed the description of the representation matrices of all the \ac{ir}s of $C_{4v}$.

\subsection{Representation on odd BCS functions $\v{d}_\v{k}$}

When discussing the representation of general group elements on states in the BCS Hilbert space in Section~\ref{sec:Group:pdRep} we learned from
Eq.~\eqref{eq:Group:pdRep:dTransformation} that the coefficient $\v{d}_\v{k}\in\mathbb{R}^3$ of odd states transforms by a proper rotation $R(\hat{\v{u}},\phi)$.
A proper rotation of a group element $g$ is the rotation obtained when writing $g$ as a rotation followed by an inversion $P$ or identity. We thus
obtain the representation matrices of the $3$D-representation of group elements $g\in C_{4v}$ directly from the $3$D rotation matrix $R(\hat{\v{u}},\phi)$ by
envisioning the combination of a rotation and $C_{4}^2=P$ or $e$ that leads to $g$. Because of this, we denote the representation matrices of this
representation $R(g)$. As an example $\sigma_x = C_4^2R(\hat{\v{x}},\pi)$, such that the representation $R(\sigma_x) = R(\hat{\v{x}},\pi)$. Written
out in its full matrix form this yields the representation matrices
%
\begin{subequations}
    \label{eq:Group:Symm:dRepresentation}
    \begin{align}
        \label{eq:Group:Symm:dRepresentation:C4}
        R(C_4) &= 
        \begin{pmatrix}
            0 & -1 & 0\\
            1 & 0 & 0\\
            0 & 0 & 1
        \end{pmatrix}, & R(C_4^{-1}) &= 
        \begin{pmatrix}
            0 & 1 & 0\\
            -1 & 0 & 0\\
            0 & 0 & 1
        \end{pmatrix},\\
        R(e) &= 
        \begin{pmatrix}
            1 & 0 & 0\\
            0 & 1 & 0\\
            0 & 0 & 1
        \end{pmatrix}, & R(C_4^2) &=
        \begin{pmatrix}
            -1 & 0 & 0\\
            0 & -1 & 0\\
            0 & 0 & 1
        \end{pmatrix},\\
        R(\sigma_x) &= 
        \begin{pmatrix}
            1 & 0 & 0\\
            0 & -1 & 0\\
            0 & 0 & -1
        \end{pmatrix}, & R(\sigma_y) &=
        \begin{pmatrix}
            -1 & 0 & 0\\
            0 & 1 & 0\\
            0 & 0 & -1
        \end{pmatrix},\\
        R(\sigma_{d_1}) &=
        \begin{pmatrix}
            0 & 1 & 0\\
            1 & 0 & 0\\
            0 & 0 & -1
        \end{pmatrix}, & R(\sigma_{d_2}) &=
        \begin{pmatrix}
            0 & -1 & 0\\
            -1 & 0 & 0\\
            0 & 0 & -1
        \end{pmatrix}.
    \end{align}
\end{subequations}

\section{Square Lattice Harmonics}

Since $\psi(\v{k})$ and $\v{d}_\v{k}$ are invariant with respect to translation by any reciprocal lattice vector $\v{Q}$: $\psi(\v{k}+\v{Q}) = \psi(\v{k})$,
they can be expanded in a discrete Fourier transform over the real lattice, such that
\begin{equation}
    \label{eq:Group:SLH:psiFourier}
    \psi(\v{k}) = \frac{1}{\sqrt{N}}\sum_\v{R}\psi_\v{R}\cos\v{R}\cdot\v{k},
\end{equation}
and
\begin{equation}
    \label{eq:Group:SLH:dFourier}
    \v{d}_\v{k} = \frac{1}{\sqrt{N}}\sum_\v{R}\v{d}_\v{R}\sin\v{R}\cdot\v{k},
\end{equation}
where the exponential of the Fourier transform has been reduced to trigonometric functions by the parity of the functions.

We are interested in the basis vectors $\ket{\Gamma,q,m}$ of the representations of the symmetry group $C_{4v}$ of the $2D$ square lattice. In this
ket notation, $\Gamma$ gives the irrep., $m$ enumerates the basis vectors in case the \irr is multi-dimensional, while $q$ gives the version of
the \irr in case the space of possible $\ket{v}$ permits multiple versions of the same \irr In the active
view of group transformations, the question of finding the basis vectors translates to finding the basis functions of the functions $\psi(\v{k})$
and $\v{d}_\v{k}$ for even and odd bases respectively. In Section~\ref{sec:Group:pdRep} we
saw how these functions transformed under group transformations. In Section~\ref{sec:Group:Pro} we saw how the projection operators could be used
to extract individual basis vectors. We will now use these operators on the Fourier expansions of $\psi(\v{k})$ and $\v{d}_\v{k}$ to extract possible
basis functions given the symmetry group of the square lattice.

\subsection{Even basis functions}

We remember from Eq.~\eqref{eq:Group:pdRep:psiRep} that the function $\psi(\v{k})$ transforms as $g:\psi(\v{k}) = \psi(g^{-1}:\v{k})$ under a group
transformation $g$. Operating on the Fourier expansion of $\psi(\v{k})$ in Eq.~\eqref{eq:Group:SLH:psiFourier} with the projection operator
defined in Eq.~\eqref{eq:Group:Irr:Pro:proOpDef} by an arbitrary \irr $\Gamma$ yields
\begin{equation}
    \label{eq:Group:SLH:even:Ppsi}
    \begin{split}
        P_{l,l}^{(\Gamma)}\psi(\v{k}) &= \frac{d_\Gamma}{8\sqrt{N}}\sum_\v{R}\psi_\v{R}\Big[\big(D_{ll}^{(\Gamma)}(e)+D_{ll}^{(\Gamma)}(C_4^2)\big)\cos(\v{R}\cdot\v{k})\\
        + & \big(D_{ll}^{(\Gamma)}(C_4) + D_{ll}^{(\Gamma)}(C_4^{-1})\big)\cos(R_xk_y-R_yk_x)\\
        + & \big(D_{ll}^{(\Gamma)}(\sigma_x) + D_{ll}^{(\Gamma)}(\sigma_y)\big)\cos(R_xk_x-R_yk_y)\\
        + & \big(D_{ll}^{(\Gamma)}(\sigma_{d_1}) + D_{ll}^{(\Gamma)}(\sigma_{d_2})\big)\cos(R_xk_y+R_yk_x).
    \end{split}
\end{equation}
Since $\v{k}\in\mathbb{R}^2$ we have in this calculation used the natural $2D$ representation given by the matrices in Eq.~\eqref{eq:Group:Symm:G5D}
of the group elements in $C_{4v}$ to calculate the expressions $g^{-1}:\v{k}$ inside the cosine functions. Inserting the matrix element of the different
\ac{ir}s of $C_{4v}$ which we discussed in Section~\ref{sec:Group:Symm} we get the projected functions 
%
\begin{equation}
    \label{eq:Group:SLH:even:projections}
    \begin{split}
        P_{1,1}^{(\Gamma_1)}\psi(\v{k}) &\propto \sum_\v{R}\psi_\v{R}\big[\cos R_xk_x\cos R_yk_y + \cos R_xk_y\cos R_yk_x\big],\\
        P_{1,1}^{(\Gamma_2)}\psi(\v{k}) &\propto \sum_\v{R}\psi_\v{R}\big[\sin R_xk_y\sin R_yk_x - \sin R_xk_x\sin R_yk_y\big],\\
        P_{1,1}^{(\Gamma_3)}\psi(\v{k}) &\propto \sum_\v{R}\psi_\v{R}\big[\cos R_xk_x\cos R_yk_y - \cos R_xk_y\cos R_yk_x\big],\\
        P_{1,1}^{(\Gamma_4)}\psi(\v{k}) &\propto \sum_\v{R}\psi_\v{R}\big[\sin R_xk_x\sin R_yk_y + \sin R_xk_y\sin R_yk_x\big],\\
        P_{1,1}^{(\Gamma_5)}\psi(\v{k}) &\propto 0,\\
        P_{2,2}^{(\Gamma_5)}\psi(\v{k}) &\propto 0.
    \end{split}
\end{equation}
%
Since the projection operators $P_{ll}^{(\Gamma)}$ projects onto the subspace of vectors belonging to the \ac{ir} $\Gamma$, if we
let the different basis functions of the irreps. $\Gamma$ be denoted $\psi^{(\Gamma),q,m}(\v{k})$, then we expect the projection to produce the result
\begin{equation}
    \label{eq:Group:SLH:even:expectedProjection}
    P_{l,l}^{(\Gamma)}\psi(\v{k}) = \sum_{q}c_{q,l}\psi^{(\Gamma),q,l}(\v{k}).
\end{equation}
Here $q$ again enumerates the version of the basis of $\Gamma$ possible in the space of different $\psi(\v{k})$, and $c_{q,l}$ are the coefficients of
$\psi(\v{k})$ in the basis of the irrep. basis functions. Comparing Eqs.~\eqref{eq:Group:SLH:even:expectedProjection} and \eqref{eq:Group:SLH:even:projections},
we see that different sets of basis vectors can be obtained for the \ac{ir}s by including different order terms in the $\v{R}$-sum,
\ie different lattice neighbour sites. It is also worth noting that the \ac{ir} $\Gamma_5$ does not exist in the space of possible $\psi(\v{k})$,
since it is an odd representation.

Including on-site, nearest neighbor and next-nearest neighbor sites in the $\v{R}$ sum of the projected arbitrary function $\psi(\v{k})$ on the $\Gamma_1$ subspace in
Eq.~\eqref{eq:Group:SLH:even:expectedProjection} we see that any such function can be constructed from the three basis functions
\begin{subequations}
    \label{eq:Group:SLH:even:G1basis}
    \begin{align}
        \label{eq:Group:SLH:even:G1basis:1}
        \psi^{(\Gamma_1),1}(\v{k}) &= \frac{1}{2\pi},\\
        \label{eq:Group:SLH:even:G1basis:2}
        \psi^{(\Gamma_1),2}(\v{k}) &= \frac{1}{2\pi}(\cos k_x + \cos k_y),\\
        \label{eq:Group:SLH:even:G1basis:3}
        \psi^{(\Gamma_1),3}(\v{k}) &= \frac{1}{\pi}\cos k_x \cos k_y.
    \end{align}
\end{subequations}
Each of these functions give a complete basis function set of the $\Gamma_1$ \ac{ir} which can be checked by calculating how they transform under
group elements $g\in C_{4v}$. In this case, since this is the trivial $\Gamma_1$ representation, the functions are symmetric under all group elements $g$ which
produces the character $1$ for all conjugation classes (compare with the first row in Table~\ref{tab:Group:Symm:characterTable}). 

These basis functions are automatically
mutually orthogonal since they belong to different \ac{ir} version subspaces and their normalization coefficients have been chosen such that they are normal
on the $1^\text{st}$ Brillouin zone, \ie
\begin{equation}
    \label{eq:Group:SLH:even:G1orthonormal}
    \int_{-\pi}^\pi\int_{-\pi}^\pi\!\textrm{d}k_x\textrm{d}k_y\;\psi^{(\Gamma_1),q}(\v{k})^\ast\psi^{(\Gamma_1),q'}(\v{k}) = \delta_{qq'}.
\end{equation}

Inserting up to next-nearest neighbor sites for $\v{R}$ in the projected functions under the remaining \ac{ir}s we find the orthonormal basis 
functions
\begin{subequations}
    \label{eq:Group:SLH:even:bases}
    \begin{align}
        \label{eq:Group:SLH:even:bases:G3}
        \psi^{(\Gamma_3)}(\v{k}) &= \frac{1}{2\pi}(\cos k_x - \cos k_y),\\
        \label{eq:Group:SLH:even:bases:G4}
        \psi^{(\Gamma_4)}(\v{k}) &= \frac{1}{\pi}\sin k_x\sin k_y,
    \end{align}
\end{subequations}
of the representations $\Gamma_3$ and $\Gamma_4$ respectively. The $\Gamma_3$ basis is found by expansion of $\v{R}$ to nearest neighbor, while the one for $\Gamma_4$
is found at the next-nearest neighbor. To get a basis vector for the representation $\Gamma_2$ we would need to expand beyond the next-nearest neighbor site.

These basis functions are also known as square lattice harmonics. The set of square lattice harmonic functions includes the set of functions found when expanding
$\v{R}$ to arbitrary sites. As we have seen, they can be grouped and found through consideration of the \ac{ir}s of the symmetry group of the
square lattice. We have so far only considered even-in-$\v{k}$ basis functions. In the next section we will complete our discussion of square lattice harmonics
with the inclusion of odd functions.


\subsection{Odd basis functions}

Any state made out of odd basis functions is fully determined by the coefficients $\v{d}_\v{k}$. These coefficients transform as $g:\v{d}_\v{k} = R(g)\v{d}_{g^{-1}:\v{k}}$
under group elements $g\in C_{4v}$, see Eq.~\eqref{eq:Group:pdRep:dTransformation}, where $R(g)$ are the representation matrices in Eq.~\eqref{eq:Group:Symm:dRepresentation}.
Acting on the Fourier expansion of the arbitrary function $\v{d}_\v{k}$ in Eq.~\eqref{eq:Group:SLH:dFourier} with the projection operators Eq.~\eqref{eq:Group:Irr:Pro:proOpDef}
down on the subspace of the \ac{ir} $\Gamma_5$ of the symmetry group $C_{4v}$ yields the results
\begin{subequations}
    \label{eq:Group:SLH:odd:projections}
    \begin{align}
        \label{eq:Group:SLH:odd:projections:1}
        P_{1,1}^{(\Gamma_5)}\v{d}_\v{k} &= \frac{\hat{\v{z}}}{\sqrt{N}}\sum_\v{R}d_{\v{R},z}\cos R_xk_x \sin R_yk_y,\\
        \label{eq:Group:SLH:odd:projections:2}
        P_{2,2}^{(\Gamma_5)}\v{d}_\v{k} &= \frac{\hat{\v{z}}}{\sqrt{N}}\sum_\v{R}d_{\v{R},z}\sin R_xk_y\cos R_yk_y,
    \end{align}
\end{subequations}
for the two basis functions of $\Gamma_5$. As in the spin-singlet case we get different versions of the $\Gamma_5$ basis vectors depending on the order of our expansion in $\v{R}$.
Expanding to nearest neighbor sites and normalizing such that the states are orthonormal produces the basis functions
\begin{subequations}
    \label{eq:Group:SLH:odd:G5basis}
    \begin{align}
        \label{eq:Group:SLH:odd:G5basis:1}
        \v{d}^{(\Gamma_5),1}_\v{k} &= -\frac{\hat{\v{z}}}{2\pi}\sin k_y,\\
        \label{eq:Group:SLH:odd:G5basis:2}
        \v{d}^{(\Gamma_5),2}_\v{k} &= \frac{\hat{\v{z}}}{2\pi}\sin k_x,
    \end{align}
\end{subequations}
of the two-dimensional \ac{ir} $\Gamma_5$.

\section{Decomposition of the Potential}
\label{sec:Group:Potential}

Let at first $\hat{V}$ be a general two-body operator that acts on an $N$-particle state which is a vector in $\hil_N = \otimes_{i=1}^N\hil$. The
single particle Hilbert space $\hil$ in question is quantified by momentum and spin such that $\hil=\Span\{\ket{\v{k},s}\}$. Denoting specific combinations
of $\v{k}$ and $s$ as $\alpha$ as a shorthand for the moment, then $\hat{V}$ acts on basis vectors in $\hil_N$ as
\begin{equation}
    \label{eq:Group:Potential:twoBodyInteraction}
    \hat{V}\ket{\alpha_1}\ldots\ket{\alpha_N} = \sum_{1\leq i<j\leq N}\hat{V}_{ij}\ket{\alpha_1}\ldots\ket{\alpha_N},
\end{equation}
by definition of being a two-body operator. Here $\hat{V}_{ij}$ is an operator that only acts on the $i$th and $j$th ket. Even though $\hat{V}$ acts on
$\hil_N$, because of how it can be written in terms of $\hat{V}_{ij}$ and this only acts on two states at a time, it follows that $\hat{V}$ is
completely determined by its action on the reduced Hilbert space $\hil_2$. This implies that $\hat{V}$ is fully described by its matrix elements
\begin{equation}
    \label{eq:Group:Potential:genTwoBodyMatrixElem}
    \bra{\alpha}\bra{\alpha'}\hat{V}\ket{\beta}\ket{\beta'}.
\end{equation}
Inserting back the $\ket{\v{k},s}$ notation, these matrix elements are referred to as
\begin{equation}
    \label{eq:Group:Potential:momSpinPotentialMatrixElems}
    V_{\v{k}_1\v{k}_2\v{k}_3\v{k}_4;\,s_1s_2s_3s_4} = \bra{\v{k}_1s_1}\bra{\v{k}_2s_2}\hat{V}\ket{\v{k}_4s_4}\ket{\v{k}_3s_3}.
\end{equation}
When $\hat{V}$ is a BCS operator acting on the BCS Hilbert space described in the previous section, these matrix elements are denoted
\begin{equation}
    \label{eq:Group:Potential:bcsTwoBodyMatrixElem}
    V_{\v{k}\v{k}';\,s_1s_2s_3s_4} = \bra{\v{k}s_1}\bra{-\v{k}s_2}\hat{V}\ket{\v{k}'s_4}\ket{-\v{k}'s_3}.
\end{equation}

Since $\hat{V}$ is Hermitian, it must be diagonalizable in some basis of eigenfunctions. Barring accidental degeneracy, a basis for a $d$-degenerate
eigenvalue is also a basis for an \ac{ir} of the symmetry group $G$ of the Hamiltonian. In the case of accidental degeneracy
then this $d$-dimensional vector space consists of several non-intersecting subspaces where each subspace is a basis for a (possibly different) \irr Note
that this does not mean that (barring accidental degeneracy) there exists one separate eigenvalue for each \irr of $G$ since there might be
several different eigenvalues with different eigenspace bases but where all of them are bases for the same \irr Regardless of these details,
the connection between \ac{ir}s and the eigenvalues of $\hat{V}$ is a great help in finding the bases for which it is
diagonal.

We let the basis for a $d_\Gamma$-dimensional \irr $\Gamma$ be denoted $\{\ket{\Gamma,q_\Gamma,m}\}_{m=1}^{d_\Gamma}$, where $\hat{V}$ has an eigenvalue $V_{\Gamma,q_\Gamma}$
for the vectors in this basis and $q_\Gamma$ is an index enumerating the different versions of bases of $\Gamma$ that $\hat{V}$ might have
in its set of eigenspace bases. Since $\hat{V}$ then is diagonal in this set of bases then
\begin{equation}
    \label{eq:Group:Potential:potIrrepDecomp}
    \hat{V} = \sum_{\Gamma q_\Gamma}V_{\Gamma,q_\Gamma}\sum_{m=1}^{d_\Gamma}\ket{\Gamma,q_\Gamma,m}\bra{\Gamma,q_\Gamma,m}.
\end{equation}
Because of the potential for accidental degeneracy we can not guarantee that $V_{\Gamma,q_\Gamma}\neq V_{\Gamma',q_{\Gamma'}}$ for different $\Gamma$ and $\Gamma'$.
Inserting this expression for $\hat{V}$ into the matrix elements in Eq.~\eqref{eq:Group:Potential:bcsTwoBodyMatrixElem} lets us write them in terms of
\ac{ir} basis vectors in the momentum spin function representation:
\begin{equation}
    \label{eq:Group:Potential:irrepPotMatrixElem}
    V_{\v{k}\v{k}';\,s_1s_2s_3s_4} = \sum_\Gamma V_{\Gamma,q_\Gamma}\sum_{m=1}^{d_\Gamma}\Psi_{s_1s_2}^{\Gamma,q_\Gamma}(\v{k})\Psi_{s_3s_4}^{\Gamma,q_\Gamma}(-\v{k}')^\dagger,
\end{equation}
where
\begin{equation}
    \label{eq:Group:Potential:irrepPotMatrixElem:coeff}
    \Psi_{s_1s_2}^{\Gamma,q_\Gamma}(\v{k}) = \bra{\v{k},s_1}\braket{-\v{k},s_2}{\Gamma,q_\Gamma,m}.
\end{equation}

We can separate the set of different \ac{ir} bases into bases that have vectors that transform either symmetrically or anti-symmetrically
with respect to the group element of space inversion $P$. We call the representations of such bases even or odd representations. Even representations are those that map
$P$ to the identity operator $\mathbb{1}$ and as a consequence have $\Psi^{\Gamma,q_\Gamma,m}_{s_1s_2}(-\v{k}) = \Psi^{\Gamma,q_\Gamma,m}_{s_1s_2}(\v{k})$. Writing
the spin-indices in these functions in terms of Pauli matrices by using the expansion in Eq.~\eqref{eq:Group:BCSHil:arb:pauli} and using the fermionic symmetry then
even representations $a$ have
\begin{equation}
    \label{eq:Group:Potential:evenIrrepFunctions}
    \Psi^{a,q_a,m}_{s_1s_2}(\v{k}) = \psi^{a,q_a,m}_\v{k}i\sigma^y_{s_1s_2}.
\end{equation}
Odd representations $b$ map $P$ to the inversion operator $I$ such that $\Psi^{b,q_b,m}_{s_1s_2}(-\v{k}) = -\Psi^{b,q_b,m}_{s_1s_2}(\v{k})$. Expanding in Pauli
matrices then yields
\begin{equation}
    \label{eq:Group:Potential:oddIrrepFunctions}
    \Psi^{b,q_b,m}_{s_1s_2}(\v{k}) = \v{d}^{b,q_b,m}_{\v{k}}\cdot(\v{\sigma}i\sigma^y)_{s_1s_2}.
\end{equation}
Separating the sum over \ac{ir}s $\Gamma$ into sums over even ($a$) and odd ($b$) representations in the potential operator matrix elements in
Eq.~\eqref{eq:Group:Potential:irrepPotMatrixElem}, we arrive at the fully expanded expression
\begin{equation}
    \label{eq:Group:Potential:fullyExpandedPotMatrixElem}
    \begin{split}
        &V_{\v{k}\v{k}';\,s_1s_2s_3s_4} = \sum_{aq_a}V_{a,q_a}\sum_{m=1}^{d_a}\psi_\v{k}^{a,q_a,m}i\sigma^y_{s_1s_2}\big(\psi_{-\v{k}'}^{a,q_a,m}i\sigma^y_{s_3s_4}\big)^\dagger\\
        & + \sum_{bq_b}V_{b,q_b}\sum_{m=1}^{d_b}\big(\v{d}^{b,q_b,m}_{\v{k}}\cdot\v{\sigma}i\sigma^y\big)_{s_1s_2}\Big[\big(\v{d}^{b,q_b,m}_{-\v{k}'}\cdot\v{\sigma}i\sigma^y\big)_{s_3s_4}\Big]^\dagger.
    \end{split}
\end{equation}
In this use of the dagger notation, the adjoint acts on both of the matrix indices such that $\v{d}_{-\v{k}}^\dagger = \v{d}_\v{k}^\ast$ and $\sigma_{s_1s_2}^\dagger = \sigma_{s_2s_1}^\ast$.
