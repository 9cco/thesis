\chapter{Vortices in superconductors}
\label{sec:Vor}
% Brain-log:
% Topological defects
% Cause of phase-transitions; vortex blow-out
% Abikosov vortices

In conventional type-I superconductors, the Meissner effect prevents any magnetic field from penetrating the superconductor when it is in the
superconducting state. In a type-II superconductor, the transition between the normal- and superconducting state is more gradual than in the type-I case
due to an intermediate transitional state where topological defects in the superconducting field  becomes stable allowing quanta of magnetic field
to pass through the material. The transitional value of the external field strength below which no magnetic field penetrates the superconductor is
called $B_{c1}$. The upper transitional field strength above which the material stops being superconducting altogether is called $B_{c2}$. The state
with regions of topological defects through which magnetic field quanta can penetrate, which are interspersed in a sea of
superconducing state then exists between these values. It is important to note that the Meissner effect is still present in this transitional state - preventing magnetic field
lines from penetrating the superconducting state, however at topological defects, the material switches to the normal state and thus
allows magnetic field lines to penetrate at these points. The final continuous transition to the normal state at $B_{c2}$ is then caused by the
proliferation of vortex-loops, sending the whole material to the normal state.
%TODO: cite.

The regions of normal state containing a topological defect of the superconducting state and through which magnetic field quanta can penetrate are known
as superconducting vortices because they are surrounded by a circulating superconducting current. This current is set up by the presence of the magnetic field
and shilds the rest of the superconducting condensate from its influence.

Whether a superconductor is considered type-I or type-II is conventionally given by the magnetic field penetration length $\lambda$ and the superconducting
coherence length $\xi$ which come together to form the \ac{gl} parameter $\kappa = \lambda/\xi$.
These parameters come out of the description of the superconducting state given by the \ac{gl} theory of a single-component complex field minimally coupled
to a gauge field.
If $\kappa\gg1$ then we say we have a strongly
type-II superconductor, while is $\kappa\ll1$ the superconductor is strongly type-I. The transitional value between type-I and type-II have a theoretical
mean-field value of $\kappa = 1/\sqrt{2}$, however numerical calculations has given it the value $\kappa = (0.76\pm0.04)/\sqrt{2}$ all within the conventional
\ac{gl} formalism.

In a type-II conventional superconductor without any structural defects, as we increase the field strength, we introduce more vortices into the material
in order to carry the required number of magnetic field quanta. At first these vortices behave like a liquid where they mutually repel eachother if they 
get close. As more vortices are introduced to the system, the inter-portex repulsion leads to them forming a two-dimensional lattice with equidistant lattice-spacing.
Since the triagonal lattice is the lattice with the highest packing fraction, i.e. the lattice that have the highest density of sites at a given lattice spacing,
the lattice formed will be triagonal. Such a triagonal (hexagonal) lattice of single quanta vortices is known as the Abrikosov lattice since it consists
of single quantua vortices which are known as Abrikosov vortices.

\section{Vorticity observables}

% Local vorticity
% Total vorticity when going around the system and why
%    we have restrictions on the filling fraction.

A condensate described by a complex field $\psi$ with phase $\theta$ can have topological defects given by discontinuities in the field $\theta$ due to its compact
nature ($\theta\in[0,2\pi)$). These topologicals
defects then lead to singularities in the field $\nabla\theta$ which allows a nonzero value of $\nabla\times\nabla\theta$ at these points. Integrating over the surface
$S$ system with surface normal vector $\hat{s}$ and using Stokes theorem then yields
\begin{equation}
    \label{eq:Vor:Obs:vorticityIntegral}
    \int_S\!\mathrm{d}^2r (\nabla\times\nabla\theta)\cdot\hat{s} = \oint_{\partial S}\nabla\theta\cdot\mathrm{d}\v{r} = 2\pi N_v,\quad N_v\in\mathbb{Z},
\end{equation}
where $\partial S$ is a path around the boundary of $S$ traversed counter-clockwise. The last equality comes from the observation that
$\partial S$ is far away from the singularity such that $\nabla\theta$ is continuous along the path and $N_v$ thus counts the number of times the vector $\nabla\theta$
rotates counter-clockwise back to its initial position. $N_v$ can then be interpreted as the total vorticity of the field $\theta$ over the surface $S$.
Since $N_v$ is the total vorticity, then from Eq.~\eqref{eq:Vor:Obs:vorticityIntegral} we see that
\begin{equation}
    \label{eq:Vor:Obs:localVorticity}
    \v{n}_v = \frac{\nabla\times\nabla\theta}{2\pi} 
\end{equation}
must be interpreted as a vector of local vorticity density.

If the system $\psi$ and $\theta$ described contains a gauge field that is coupled to them, then any meaningful observable needs to be gauge-invariant.
We clearly see that the expression in Eq.~\eqref{eq:Vor:Obs:localVorticity} is gauge-dependent by sending $\theta\to\theta+\phi$. To make a gauge-invariant observable under
the gauge-transformation in Eq.~\eqref{eq:LM:Derivatives:gaugeTransformation}, we see that we need to modify the definition to
\begin{equation}
    \label{eq:Vor:Obs:localVorticity}
    \v{n}_v = \frac{\nabla\times(\nabla\theta + g\v{A})}{2\pi}.
\end{equation}
This expression then defines $\v{n}_v$ as a gauge invariant vector of local vorticity density of the compact field $\theta$.

In lattice models we want to discretize the vorticity density in Eq.~\eqref{eq:Vor:Obs:localVorticity} in order to effectively calculate it in Monte-Carlo simulations
of the lattice model. In such a discrete model we have to take care to re-compactify the quantity $\nabla\theta+g\v{A}$ to only be defined on some interval of length $2\pi$.
Using the discretization mapping of $\partial_\mu$ and $A_\mu(\v{r})$ from Eq.~\eqref{eq:LM:Field:Fluc:discretization}, we want
$\Delta_\mu\theta + gA_{\v{r},\mu}\in[-\pi,\pi)$. Defining the operator
\begin{equation}
    \label{eq:Vor:Obs:moduloOperator}
    \hat{C}_\pi\, x = \modulo(x+\pi,2\pi)-\pi,
\end{equation}
the discretized vorticity density can be written
\begin{equation}
    \label{eq:Vor:Obs:discreteLocalVorticity}
    \begin{split}
        \v{n}_{v,\v{r}} &= \frac{\hat{e}_\mu\epsilon_{\mu\nu\lambda}\Delta_\nu\hat{C}_\pi(\Delta_\lambda\theta_\v{r} + gA_{\v{r},\lambda})}{2\pi a^2}\\
        &= \frac{1}{2\pi a^2}\sum_\mu \hat{e}_\mu\sum_{\boxdot_\mu}\hat{C}_\pi(\Delta_\lambda\theta_\v{r} + gA_{\v{r},\lambda}).
    \end{split}
\end{equation}
Implicit summation over repeated indices is used on the first line while on the second, the components of the vector $\v{n}_{v,\v{r}}$ are written as plaquette-sums,
which are sums of direction dependent quantities along a path $\boxdot_\mu$, which is described below Eq.~\eqref{eq:LM:Field:Fluc:plaquetteSum} and illustrated in
Figure~\ref{fig:LM:Field:Fluc:plaquetteSum}. In the plaquette-sum the directional quantity is always chosen along the path and the path is traversed according to
the right hand rule with normal vector $\hat{e}_\mu$.

\section{Double quanta vorticies}

\section{Vortex lattices}

% Abikosov lattice of maximum packing fraction.
% Structure function
% Peak histogram

\section{Vortex lattice transitions}
