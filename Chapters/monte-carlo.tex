\chapter{Monte-Carlo Techniques}

In this chapter we discuss some techniques useful in Monte-Carlo simulations
of systems in statistical-physics.
In such systems these techniques wil be used to calculate thermal averages using random numbers. Let
$Z$ denote the partition function and $\mathcal{H}$ the Hamiltonian of the system. Then the thermal
average of an observable $\mathcal{O}$ is defined as
\begin{equation}
    \label{eq:Monte:thermalAverage}
    \langle\mathcal{O}\rangle = \frac{1}{Z}\sum_\psi\mathcal{O}(\psi)e^{-\beta\mathcal{H}(\psi)},
\end{equation}
where $\psi$ denotes states of the system and we thus sum over all possible states. In the case of
a quantum system, then this sum turns into a multi-dimensional integral over quantum coherent states.
What Monte-Carlo techniques then provides is a way of using random numbers in calculating
Eq.~\eqref{eq:Monte:thermalAverage} without actually summing over all the states. We do this by drawing random states
$\psi_i$ from a carefully selected probability distribution and using statistics to estimate how close the resulting
thermal average is likely to be to the true thermal average.

\section{Markov-Chain Monte-Carlo method}

\section{Metropolis-Hastings method}

\section{Importance sampling of observables}

\section{Thermalization procedures}

\section{Parallell tempering}

\section{Reweighting}
