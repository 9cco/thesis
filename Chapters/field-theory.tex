\chapter{Field Theory Methods}

In this chapter we will give a short introduction to the use of grassmann variables
and complex numbers in the calculation of the field-integrals on the
statistical-mechanical partition function and how they can be used to transform the
expression for the action through the Hubbard-Stratonovich transformation.

\section{Pfaffian}

\section{Hubbard-Stratonovich transformation}

The Hubbard-Stratonovich transformation is a transformation in the fields of a theory where a new complex field is introduced
in order to covert a term that is square and thus non-linear in an existing field variable, into a linear term in this
variable that is coupled to the new field. This is particularly useful when the existing field is Fermionic and thus
a Gra\ss mann variable since it makes it possible to consider low energy exitations of the theory using \eg a saddle-point
approximation. It is however important to point out that the transformation itself is not in any way approximative, but is
an exact transformation that maintains all information in the original theory.

In technical terms, the Hubbard-Stratonovich transformation can be viewed simply as the solution of a complex multivariate integral.
Let $A$ have a strictly positive Hermitian part and $\v{J}$ be a vector of
coefficients that could contain Gra\ss mann variables or complex variables. Then
\begin{equation}
    \label{eq:Field:HS:complexIntegral}
    e^{\v{J}^\dagger A \v{J}} = \det A^{-1}\int_{\mathbb{C}}\prod_{i}\bigg[\frac{\mathrm{d}z_i^\ast\mathrm{d}z_i}{2\pi i}\bigg]e^{-\v{z}^\dagger A^{-1}\v{z} + \v{z}^\dagger \v{J} + \v{J}^\dagger\v{z}},
\end{equation}
exchanges a quadratic term in $\v{J}$ with a integration over the complex $\v{z}$ variable.
Since $\v{J}$ usually represents some field in a field theory, the new $\v{z}$ is called the auxhillary field or the conjugate
field because of its linear coupling to $\v{J}$.
In the less general case that $A$ is an Hermitian matrix, this formula is proved simply by completing the square, then
diagonalizing $A$ by a unitary tranformation and calculating the resulting integrals by the formula 
$\int_{\mathbb{C}}\!\mathrm{d}z^\ast\mathrm{d}z\,e^{-azz^\ast} = 2\pi i /a$. 

From Eq.~\eqref{eq:Field:HS:complexIntegral} we see that
what we have to do to perform the Hubbard-Stratonovich transformation is first to make a choice for what to interpret as part
of the matrix $A$ and what to interpret as part of $\v{J}$. We then have to check that this definition of $A$ leads to its
Hermitian part having only positive eigenvalues. Finally we need to know an analytical expression for its inverse.
It is usually the first step that is the most difficult since this dictates the low energy excitation a subsequent saddle point
approximation or a stationary phase approximation will produce.
Typically we are interested in transforming a Fermionic interaction potential of the form
\begin{equation}
    \label{eq:Field:HS:typicalInteactionPotential}
    V = \frac{1}{2}\sum_{\alpha\beta\gamma\delta}V_{\alpha\beta\gamma\delta}\xi^\ast_\alpha\xi^\ast_\beta\xi_\delta\xi_\gamma,
\end{equation}
where $\xi_\alpha$ are Gra\ss mann variables, which can be sketched in the way of single-vertex diagram in Figure~\ref{fig:Field:HS:twoBodyInteraction}.
\begin{figure}[h]
    \begin{center}

\feynmandiagram [horizontal = i1 to f1] {
  i1 [particle=$\delta$] -- [fermion] a [dot] -- [anti fermion] i2 [particle=$\gamma$],
  f1 [particle=$\beta$] -- [anti fermion] a -- [fermion] f2 [particle=$\alpha$],
};

    \end{center}
    \caption{Generic two-body interaction.}
    \label{fig:Field:HS:twoBodyInteraction}
\end{figure}
The HS-transformation is classified into being done in a specific \emph{channel} depending on which pair of Gra\ss mann variables are
considered to be part of $\v{J}$ and consequently $\v{J}^\dagger$. The direct channel\footnote{Also known as the density-density channel.} is given by 
the identification $J_i \sim \xi_\alpha^\ast\xi_\gamma$,
the Cooper channel\footnote{Also known as the particle-particle channel.} is defined by the identification $J_i \sim \xi_\delta\xi_\gamma$ while the
exchange channel is given by the identification $J_i \sim \xi_\alpha^\ast\xi_\delta$. Depending on exactly how $\v{J}$ is chosen, the Gaussian
integral in Eq.~\eqref{eq:Field:HS:complexIntegral} might have to be modified. For example in the case of the direct- and exchange-channel, the
exponential argument on the left side will have the form $\v{J}^\trans A\v{J}$ which necessitates the Gaussian integral identity
\begin{equation}
    \label{eq:Field:HS:realIntegral}
    e^{-\frac{1}{2}\v{J}^\trans A\v{J}} = \sqrt{\det A^{-1}}\int_{\mathbb{R}}\!\prod_i\bigg[\frac{\mathrm{d}x_i}{\sqrt{2\pi}}\bigg]e^{-\frac{1}{2}\v{x}^\trans A^{-1}\v{x} - i\v{J}^\trans\v{x}},
\end{equation}
where the auxhillary field $\v{x}$ now is a real conjugate field.


\subsection{Transformation in symmetry channels}

In the Cooper-channel of the Hubbard-Stratonovich transformation, the complex field $\v{z}$ is conjugate to some combination
of pairs of annihilation operators $\hat{c}_\delta \hat{c}_\gamma$ (or their corresponding Gra\ss mann variables).
The symmetry of the specific combination in turn then determines the symmetry of any low energy field theory obtained
through a subsequent stationary phase approximation. By diagonalizing the interaction potential $\hat{V}$ into its particular
irreducible representations as we did in Section~\ref{sec:Group:Potential} then a Hubbard-Stratonovich transformation
in a specific symmetry channel is done by identifying $\v{J}$ with irrep. basis functions. 

Lets take the case of a BCS theory of superconductivity where the interaction can be written in terms of odd
basis functions $d^{(b),m}_{s_1s_2}(\v{k})$ such that the diagonalized interaction takes the form
\begin{equation}
    \label{eq:Field:HS:Symm:diagonalizedV}
    \hat{V} = \sum d^{(b),m}_{s_1s_2}(\v{k})^\ast v_{(b)}d^{(b),m}_{s_1's_2'}(\v{k}') c_{\frac{\v{q}}{2}+\v{k}\,s_1}^\dagger c_{\frac{\v{q}}{2}-\v{k}\,s_2}^\dagger c_{\frac{\v{q}}{2}-\v{k}'s_2'}c_{\frac{\v{q}}{2}+\v{k}'s_1'},
\end{equation}
where $\sum$ indicates the sum over the indices, $\v{k}, \v{k}', \v{q}, s_1, s_2, s_1', s_2', b$ and $m$. Here $b$ specifies the irreducible representation
while $m$ enumerates the representation basis.
Identifying
\begin{equation}
    \label{eq:Field:HS:Symm:identification}
    \hat{J}_\v{q}^{(b),m} = \sum_{\v{k}s_1s_2}d^{(b),m}_{s_1s_2}(\v{k})\hat{c}_{\frac{\v{q}}{2}-\v{k},s_1}\hat{c}_{\frac{\v{q}}{2}+\v{k},s_2},
\end{equation}
the interaction potential is simply written
\begin{equation}
    \label{eq:Field:HS:Symm:HSpreparedV}
    \hat{V} = \sum_{\v{q},b,m}\hat{J}_\v{q}^{(b),m\,\dagger}v^{(b)}\hat{J}_\v{q}^{(b),m}.
\end{equation}

\section{Extended Nambu notation}

\section{Log-expansion}
