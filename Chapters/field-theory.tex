\chapter{Field Theory Methods}

In this chapter we will give a short introduction to the use of grassmann variables
and complex numbers in the calculation of the field-integrals on the
statistical-mechanical partition function and how they can be used to transform the
expression for the action through the Hubbard-Stratonovich transformation.

A field theoretic expression for the quantum mechanical partition function $\mathcal{Z}$
is obtained by using a coherent state basis. A coherent state is the eigen-state of an annihilation operator, thus it
produces an eigenvalue when operated on by the annihilation operator. Letting $\hat{H}$ be the quantum mechanical
Hamiltonian of the system for which we are interested in calculating the parition function, $\mu$ the the chemical
potential and $\hat{N}$ is the number operator then the partition function
\begin{equation}
    \label{eq:Field:partitionFunction}
    \mathcal{Z} = \tr(e^{-\beta(\hat{H}-\mu\hat{N})}).
\end{equation}
Inserting a basis of coherent states $\{\ket{\xi}\}$ when calculating the trace, we obtain a functional integral over
the coherent state eigenvalues $\xi_\alpha$ and $\xi^\ast_\alpha$ which substitute the annihilation and creation operators
$c_\alpha$ and $c_\alpha^\dagger$ respectively. Here $\alpha$ symbolize the collection of quantum-numbers needed to specify
a state. The functional integral then takes the form
\begin{equation}
    \label{eq:Field:fieldPartition}
    \mathcal{Z} = \int\!\mathcal{D}[\xi^\ast\,\xi]e^{-\int_0^\beta\!\mathrm{d}\tau\sum_\alpha [\xi^\ast_\alpha(\partial_\tau-\mu)\xi_\alpha + H(\xi_\alpha,\xi_\alpha^\ast)]}.
\end{equation}
The integration variable $\tau$ is the imaginary time and a $\tau$ dependence is implicit in the notation such that $\xi_\alpha = \xi_\alpha(\tau)$.
This path-integral notation is a shorthand for a more involved expression where the imaginary time-dependence of $\tau$ is
split into a collection of time-indexed coherent state eigenfunctions $\xi_{\alpha,\tau}$ and the integration measure
is a product over these indices and the quantum-state indices $\alpha$. For further detail we refer to Ref.~\cite{NegeleOrland98}
which we will follow for a large part of this chapter.

\section{Calculation of Fermionic Field Integrals}

Because of the anti-commuting property of the fermion annihilation operators, any coherent state\footnote{An eigenstate
of the annihilation operator} has to have eigenvalues that anti-commute as well. This leads to the partition function
written in the convenient basis of the coherent states being constructed with Gra\ss mann numbers as the central
variables.

\subsection{Gra\ss mann algebras}
A Gra\ss mann algebra is constructed on a set of generators $\{\xi_\alpha\}$ such that a specific product
of the generators $\xi_{\alpha_1}\xi_{\alpha_2}\cdots\xi_{\alpha_n}$ together with a complex coefficient $\phi$ constitute a
number in the algebra and the generators anti-commute such that $\xi_\alpha\xi_\beta = -\xi_\beta\xi_\alpha$. On such
an algebra, differentiation can be defined such that
\begin{equation}
    \label{eq:Field:Ferm:Grass:diff}
    \frac{\mathrm{d}}{\mathrm{d}\xi_{\alpha_m}}\;\phi\,\xi_{\alpha_1}\cdots\xi_{\alpha_n} = (-1)^m\phi\,\xi_{\alpha_1}\cdots\xi_{\alpha_{m-1}}\xi_{\alpha_{m+1}}\cdots\xi_{\alpha_n},
\end{equation}
provided the generator $\xi_{\alpha_m}$ is in the number and $0$ otherwise. The factors of $(-1)$ comes from anti-commuting
the generator $\xi_{\alpha_m}$ such that it is next to the differentiation operator. In Gra\ss mann algebra, integration
can be (perhaps a little non-intuitively) be defined such that it acts in the same way as differentiation, i.e. generators
have to be anti-commuted until they are next to the symbolic differential symbol $\mathrm{d}\xi_{\alpha}$, and then
\begin{equation}
    \label{eq:Field:Ferm:Grass:int}
    \int\!\mathrm{d}\xi\;\xi = 1,
\end{equation}
while
\begin{equation}
    \label{eq:Field:Ferm:Grass:int2}
    \int\!\mathrm{d}\xi\;1 = 0.
\end{equation}
On an algebra consisting of $2n$ generators we define conjugation as a map from the first half of the generators $\{\xi_{\alpha_i}\}_{i=1}^n$
to the other half $\{\xi_{\alpha_i}^\ast\}_{i=1}^n$ and in such a way that when applied to a particular number
\begin{equation}
    \label{eq:Field:Ferm:Grass:conj}
    (\phi\xi_\alpha\xi_\beta)^\ast = \phi^\ast\xi_\beta^\ast\xi_\alpha^\ast.
\end{equation}

\subsection{Nambu Spinor}

In the Nambu notation we group spin-dependent Gra\ss mann numbers $\xi_\up$ and $\xi_\dn^\ast$, which correspond to the
annihilation- and creation-operators $\hat{c}_\up^\dagger$ and $\hat{c}_\dn$, in a vector called a Nambu spinor
\begin{equation}
    \label{eq:Field:Ferm:Nambu:vector}
    \v{\xi} = 
    \begin{pmatrix}
        \xi_\up\\
        \xi_\dn^\ast
    \end{pmatrix}.
\end{equation}
A sesquilinear form can then be created with this vector and its adjoint such that
\begin{equation}
    \label{eq:Field:Ferm:Nambu:terms}
    \v{\xi}^\dagger S\v{\xi} = S_{11}\xi_\up^\ast\xi_\up + S_{22}\xi_\dn^\ast\xi_\dn + S_{12}\xi_\up^\ast\xi_\dn^\ast + S_{21}\xi_\up\xi_\dn.
\end{equation}
This allows any action that contains spin-dependent terms of the form of the right hand side of Eq.~\eqref{eq:Field:Ferm:Nambu:terms} to
be put on sesquilinear form. Assuming this is the case, then the partition function in the field-integral representation takes the form
\begin{equation}
    \label{eq:Field:Ferm:Nambu:partitionFunction}
    \mathcal{Z} = \int\!\mathcal{D}[\xi^\ast\,\xi]\;e^{-\int_0^\beta\!\mathrm{d}\tau\;\v{\xi}^\dagger_\gamma S_{\gamma\delta}\v{\xi}_\delta}.
\end{equation}
In this equation, the indices $\gamma$ and $\delta$ is an arbitrary collection of quantum numbers needed to specify a state other than spin,
for example they could be momentum indices $\gamma = \v{k}, \delta = \v{k}'$, and summation over these repeated indices is implicitly understood.

Splitting the integral over $\tau$ into $M$ imaginary time-slices and expanding the path integral measure into a product of individual integrals
over specific quantum numbered and time-sliced Gra\ss mann variables such that 
\begin{equation}
    \label{eq:Field:Ferm:Nambu:fieldMeasure}
    \int\!\mathcal{D}[\xi^\ast\,\xi] \;\propto\; \lim_{M\rightarrow\infty}\int\prod_{\tau=1}^M\prod_\alpha\mathrm{d}\xi^\ast_{\alpha,\tau}\mathrm{d}\xi_{\alpha,\tau},
\end{equation}
the path-integral in Eq.~\eqref{eq:Field:Ferm:Nambu:partitionFunction}
can be evaluated by the Gaussian Gra\ss mann integral identity
\begin{equation}
    \label{eq:Field:Ferm:Nambu:quadraticGrassmannInt}
    \int\prod_i(\mathrm{d}\xi_i^\ast\mathrm{d}\xi_i)\;e^{-\xi_i^\ast S_{ij}\xi_j} = \det S,
\end{equation}
for which a derivation is found in Ref.~\cite{NegeleOrland98}. This identity holds for any Hermitian matrix $S$, even if it is not positive definite.
The result is then that the partition function in Eq.~\eqref{eq:Field:Ferm:Nambu:partitionFunction}
becomes $\mathcal{Z} = \det S$. To calculate this determinant one has to consider the matrix $S$ as also a matrix with
time-slice indices. This is perhaps most easily accomplished using the Matsubara formalism in which the $\tau$ dependence is substituted with a dependence on Matsubara
frequencies through a Fourier-like transform.

\subsection{Extended Nambu Spinor}

From Eq.~\eqref{eq:Field:Ferm:Nambu:terms} we see that the Nambu spinor sesquilinear product fails to accomodate terms in a Hamiltonian that mix creation and annihilation
operators of differing spins, e.g. a term $\propto \hat{c}_\up^\dagger\hat{c}_\dn$. 
In general, a quadratic Hamiltonian can contain any combination of spin-indices of the form $\hat{c}_{s_1}\hat{c}_{s_2}$, $\hat{c}_{s_1}^\dagger\hat{c}_{s_2}$,
$\hat{c}_{s_1}\hat{c}_{s_2}^\dagger$ and
$\hat{c}_{s_1}^\dagger\hat{c}_{s_2}^\dagger$. This gives in total $16$ different combinations and to accomodate them all we thus need a $4\times4$ matrix.
Exchanging to Gra\ss mann numbers we define the vector
\begin{equation}
    \label{eq:Field:Ferm:ExNambu:spinor}
    \v{\xi}_\gamma = 
    \begin{pmatrix}
        \xi_{\gamma,\up}^\ast\\
        \xi_{\gamma,\up}\\
        \xi_{\gamma,\dn}^\ast\\
        \xi_{\gamma,\dn}
    \end{pmatrix},
\end{equation}
where all quantum numbers except spin is included in the index $\gamma$.
Writing the elements of this vector $(\v{\xi}_\gamma)_i = \tilde{\xi}_{\gamma,i}$ regardless of whether it is a conjugate or not, 
we can write all quadratic terms of a Hamiltonian on the bilinear form
\begin{equation}
    \label{eq:Field:Ferm:ExNambu:bilinear}
    \v{\xi}_\gamma^\trans S_{\gamma\delta}\v{\xi}_\delta = \tilde{\xi}_{\gamma,i}S_{\gamma i; \delta j}\tilde{\xi}_{\delta, j},
\end{equation}
where $S_{\gamma\delta}$ is a $4\times4$ anti-symmetric\footnote{To see why this matrix can always be said to be anti-symmetric lets first simplify the notation
and write the bilinear product as $\tilde{\xi}_iS_{ij}\tilde{\xi}_j$. Then the matrix $S = (S + S^\trans)/2 + (S - S^\trans)/2$, such that we can write it as a
symmetric matrix ${\mathcal{S} = (S+S^\trans)/2}$ and an anti-symmetric matrix $\mathcal{A} = (S-S^\trans)/2$. Considering only the symmetric part of the bilinear
form we get
\begin{equation*}
    \begin{split}
        \tilde{\xi}_i\mathcal{S}_{ij}\tilde{xi}_j = -\tilde{\xi}_j\mathcal{S}_{ij}\tilde{\xi}_i = -\tilde{\xi}_i\mathcal{S}_{ji}\tilde{\xi}_j = -\tilde{\xi}_i\mathcal{S}_{ij}\tilde{\xi}_j.
    \end{split}
\end{equation*}
Hence $\tilde{\xi}_i\mathcal{S}_{ij}\tilde{\xi}_j = 0$ and all that remains is the antisymmetric bilinear form.}
matrix, and $S_{\gamma i; \delta j}$ denotes its elements. Let there be $n$ number of different quantum numbers,
now including spin. Then there must be $2n$ different Gra\ss mann generators $\tilde{\xi}_{\gamma,i}$. All of these are integrated over in the discrete version
of the partition function field integral
\begin{equation}
    \label{eq:Field:Ferm:ExNambu:partitionFunction}
    \mathcal{Z} = \int\!\mathcal{D}[\xi^\ast\,\xi]\;e^{-\int_0^\beta\!\mathrm{d}\tau\;\tilde{\xi}_{\gamma,i}S_{\gamma i; \delta j}\tilde{\xi}_{\delta, j}}.
\end{equation}
Even though this superficially looks like the field integral in Eq.~\eqref{eq:Field:Ferm:Nambu:partitionFunction}, we now have a bilinear and not a sesquilinear
form, and $S$ is now a $2n\times2n$ matrix and not an $n\times n$ matrix. This means that we can not use the integral in Eq.~\eqref{eq:Field:Ferm:Nambu:quadraticGrassmannInt}
to evaluate the integral but instead have to rely on the more general Gaussian Gra\ss mann integral
\begin{equation}
    \label{eq:Field:Ferm:ExNambu:pfaffianIntegral}
    \int\prod_{i}(\mathrm{d}\tilde{\xi}_i)\;e^{-\frac{1}{2}\tilde{\xi}_iS_{ij}\tilde{\xi}_j} = \pfaff(S),
\end{equation}
which applies for any anti-symmetric matrix $S$. The right hand side is called the Pfaffian $\pfaff(S)$ of the matrix $S$ and is defined for any anti-symmetric
matrix to be given by
\begin{equation}
    \label{eq:Field:Ferm:ExNambu:pfaffianDef}
    \pfaff[S] = \frac{1}{2^nn!}\sum_{P\in S_n}(-1)^PS_{P_1P_2}\cdots S_{P_{n-1}P_n},
\end{equation}
where $P$ is a permutation in the finite group $S_n$ of all possible permutations of $n$ numbers. This matrix function is related to the determinant
by the relation $\pfaff(S)^2 = \det(S)$. 

Applying the integral identity in Eq.~\eqref{eq:Field:Ferm:ExNambu:pfaffianIntegral} to the partition function%
%
\footnote{In relating the discrete version of
Eq.~\eqref{eq:Field:Ferm:ExNambu:pfaffianIntegral} to \eqref{eq:Field:Ferm:ExNambu:partitionFunction} we have to make sure that the spinor elements $\tilde{\xi}_i$
are defined in terms of $\xi_i$ and $\xi_i^\ast$ in such a way as to get a correspondence to the sequence of Gra\ss mann
generators $\mathrm{d}\xi^\ast_i\mathrm{d}\xi_i$ in the measure to avoid any sign errors. One solution
is to set $\xi_i^\ast = \tilde{\xi}_{2i-1}$ and $\xi_i = \tilde{\xi}_{2i}$ as we have done in Eq.~\eqref{eq:Field:Ferm:ExNambu:spinor}. With this definition,
then the measure $\int\prod_i\mathrm{d}\xi_i^\ast\mathrm{d}\xi_i$, which results from the discretized version of the field-integral measure, becomes
equal to $\int\prod_{i=1}^{2n}\mathrm{d}\tilde{\xi}_i$ such that Eq.~\eqref{eq:Field:Ferm:ExNambu:pfaffianIntegral} can be directly applied.%
} %
%
in Eq.~\eqref{eq:Field:Ferm:ExNambu:partitionFunction}
after applying the proper discretization of the imaginary time then yields the result
\begin{equation}
    \label{eq:Field:Ferm:ExNambu:partitionFinal}
    \mathcal{Z} = \int\!\mathcal{D}[\xi^\ast\,\xi]\;e^{-\int_0^\beta\!\mathrm{d}\tau\;\tilde{\xi}_{\gamma,i}S_{\gamma i; \delta j}\tilde{\xi}_{\delta, j}} = \sqrt{\det(S)}.
\end{equation}
We have chosen the positive result in $\pfaff(S) = \pm\sqrt{\det(S)}$ since the partition function $\mathcal{Z}$ needs to be positive on physical grounds.
The matrix $S$ to be taken the derminant of on the right hand side of Eq.~\eqref{eq:Field:Ferm:ExNambu:partitionFinal} is the full matrix one gets after
discretizing the imaginary-time in slices which we usually have done through the Matsubara-frequency formalism.

\section{Matsubara formalism}



\section{Hubbard-Stratonovich transformation}

The Hubbard-Stratonovich transformation is a transformation in the fields of a theory where a new complex field is introduced
in order to covert a term that is square and thus non-linear in an existing field variable, into a linear term in this
variable that is coupled to the new field. This is particularly useful when the existing field is Fermionic and thus
a Gra\ss mann variable since it makes it possible to consider low energy exitations of the theory using \eg a saddle-point
approximation. It is however important to point out that the transformation itself is not in any way approximative, but is
an exact transformation that maintains all information in the original theory.

In technical terms, the Hubbard-Stratonovich transformation can be viewed simply as the solution of a complex multivariate integral.
Let $A$ have a strictly positive Hermitian part and $\v{J}$ be a vector of
coefficients that could contain Gra\ss mann variables or complex variables. Then
\begin{equation}
    \label{eq:Field:HS:complexIntegral}
    e^{\v{J}^\dagger A \v{J}} = \det A^{-1}\int_{\mathbb{C}}\prod_{i}\bigg[\frac{\mathrm{d}z_i^\ast\mathrm{d}z_i}{2\pi i}\bigg]e^{-(\v{z}^\dagger A^{-1}\v{z} + \v{z}^\dagger \v{J} + \v{J}^\dagger\v{z})},
\end{equation}
exchanges a quadratic term in $\v{J}$ with a integration over the complex $\v{z}$ variable.
Since $\v{J}$ usually represents some field in a field theory, the new $\v{z}$ is called the auxhillary field or the conjugate
field because of its linear coupling to $\v{J}$.
In the less general case that $A$ is an Hermitian matrix, this formula is proved simply by completing the square, then
diagonalizing $A$ by a unitary tranformation and calculating the resulting integrals by the formula 
$\int_{\mathbb{C}}\!\mathrm{d}z^\ast\mathrm{d}z\,e^{-azz^\ast} = 2\pi i /a$. 

From Eq.~\eqref{eq:Field:HS:complexIntegral} we see that
what we have to do to perform the Hubbard-Stratonovich transformation is first to make a choice for what to interpret as part
of the matrix $A$ and what to interpret as part of $\v{J}$. We then have to check that this definition of $A$ leads to its
Hermitian part having only positive eigenvalues. Finally we need to know an analytical expression for its inverse.
It is usually the first step that is the most difficult since this dictates the low energy excitation a subsequent saddle point
approximation or a stationary phase approximation will produce.
Typically we are interested in transforming a Fermionic interaction potential of the form
\begin{equation}
    \label{eq:Field:HS:typicalInteactionPotential}
    V = \frac{1}{2}\sum_{\alpha\beta\gamma\delta}V_{\alpha\beta\gamma\delta}\xi^\ast_\alpha\xi^\ast_\beta\xi_\delta\xi_\gamma,
\end{equation}
where $\xi_\alpha$ are Gra\ss mann variables, which can be sketched in the way of single-vertex diagram in Figure~\ref{fig:Field:HS:twoBodyInteraction}.
\begin{figure}[h]
    \begin{center}

\feynmandiagram [horizontal = i1 to f1] {
  i1 [particle=$\delta$] -- [fermion] a [dot] -- [anti fermion] i2 [particle=$\gamma$],
  f1 [particle=$\beta$] -- [anti fermion] a -- [fermion] f2 [particle=$\alpha$],
};

    \end{center}
    \caption{Generic two-body interaction.}
    \label{fig:Field:HS:twoBodyInteraction}
\end{figure}
The HS-transformation is classified into being done in a specific \emph{channel} depending on which pair of Gra\ss mann variables are
considered to be part of $\v{J}$ and consequently $\v{J}^\dagger$. The direct channel\footnote{Also known as the density-density channel.} is given by 
the identification $J_i \sim \xi_\alpha^\ast\xi_\gamma$,
the Cooper channel\footnote{Also known as the particle-particle channel.} is defined by the identification $J_i \sim \xi_\delta\xi_\gamma$ while the
exchange channel is given by the identification $J_i \sim \xi_\alpha^\ast\xi_\delta$. Depending on exactly how $\v{J}$ is chosen, the Gaussian
integral in Eq.~\eqref{eq:Field:HS:complexIntegral} might have to be modified. For example in the case of the direct- and exchange-channel, the
exponential argument on the left side will have the form $\v{J}^\trans A\v{J}$ which necessitates the Gaussian integral identity
\begin{equation}
    \label{eq:Field:HS:realIntegral}
    e^{-\frac{1}{2}\v{J}^\trans A\v{J}} = \sqrt{\det A^{-1}}\int_{\mathbb{R}}\!\prod_i\bigg[\frac{\mathrm{d}x_i}{\sqrt{2\pi}}\bigg]e^{-\frac{1}{2}\v{x}^\trans A^{-1}\v{x} - i\v{J}^\trans\v{x}},
\end{equation}
where the auxhillary field $\v{x}$ now is a real conjugate field.


\subsection{Transformation in symmetry channels}

In the Cooper-channel of the Hubbard-Stratonovich transformation, the complex field $\v{z}$ is conjugate to some combination
of pairs of annihilation operators $\hat{c}_\delta \hat{c}_\gamma$ (or their corresponding Gra\ss mann variables).
The symmetry of the specific combination in turn then determines the symmetry of any low energy field theory obtained
through a subsequent stationary phase approximation. By diagonalizing the interaction potential $\hat{V}$ into its particular
irreducible representations as we did in Section~\ref{sec:Group:Potential} then a Hubbard-Stratonovich transformation
in a specific symmetry channel is done by identifying $\v{J}$ with irrep. basis functions. 

Lets take the case of a BCS theory of superconductivity where the interaction can be written in terms of
basis functions $d^{(b),m}_{s_1s_2}(\v{k})$ such that the diagonalized interaction takes the form
\begin{equation}
    \label{eq:Field:HS:Symm:diagonalizedV}
    \hat{V} = \sum d^{(b),m}_{s_1s_2}(\v{k})^\ast v_{(b)}d^{(b),m}_{s_1's_2'}(\v{k}') c_{\frac{\v{q}}{2}+\v{k}\,s_1}^\dagger c_{\frac{\v{q}}{2}-\v{k}\,s_2}^\dagger c_{\frac{\v{q}}{2}-\v{k}'s_2'}c_{\frac{\v{q}}{2}+\v{k}'s_1'},
\end{equation}
where $\sum$ indicates the sum over the indices, $\v{k}, \v{k}', \v{q}, s_1, s_2, s_1', s_2', b$ and $m$. Here $b$ specifies the irreducible representation
while $m$ enumerates the representation basis.
Identifying
\begin{equation}
    \label{eq:Field:HS:Symm:identification}
    \hat{J}_\v{q}^{(b_m)} = \sum_{\v{k}s_1s_2}d^{(b),m}_{s_1s_2}(\v{k})\hat{c}_{\frac{\v{q}}{2}-\v{k},s_1}\hat{c}_{\frac{\v{q}}{2}+\v{k},s_2},
\end{equation}
the interaction potential is simply written
\begin{equation}
    \label{eq:Field:HS:Symm:HSpreparedV}
    \hat{V} = \sum_{\v{q},b,m}\hat{J}_\v{q}^{(b_m)\,\dagger}v^{(b)}\hat{J}_\v{q}^{(b_m)}.
\end{equation}
In the path-integral representation of the partition function, the annihilation operators become Gra\ss mann variables which we denote
by writing $J$ in stead of $\hat{J}$ such that the contribution from the interaction potential results in the exponential
\begin{equation}
    \label{eq:Field:HS:Symm:potentialExp}
    \mathcal{Z}_I = e^{-\int_0^\beta\!\mathrm{d}\tau\sum_{\v{q},b,m}J_\v{q}^{(b_m)\,\dagger}v^{(b)}J_\v{q}^{(b_m)}}.
\end{equation}
Now it is straight forward to use the Hubbard-Statonovich formula
\begin{equation}
    \label{eq:Field:HS:Symm:fieldHSTransform}
    e^{\int_0^\beta\!\mathrm{d}\tau\sum_{ij} J_i^\ast A_{ij}J_{j}} = \int\!\mathcal{D}[\eta_i^\ast\eta_i]e^{-\int_0^\beta\!\mathrm{d}\tau\big(\eta_i^\ast A^{-1}_{ij}\eta_j + J_i^\ast\eta_i+J_i\eta_i^\ast\big)},
\end{equation}
which is a path integral version of Eq.~\eqref{eq:Field:HS:complexIntegral}, to transform each pair of irreducible representation basis vectors to
individual conjugate fields. In the notation of Eq.~\eqref{eq:Field:HS:Symm:fieldHSTransform} implicit summation over repeated indices is used and each index $i$ is a collection $i = (b,m,\v{q})$ of indices.
Comparing Eq.~\eqref{eq:Field:HS:Symm:fieldHSTransform} and \eqref{eq:Field:HS:Symm:potentialExp} we gather that
\begin{equation}
    \label{eq:Field:HS:Symm:Matrix}
    A_{ij} = A_{b,m,\v{q};\,b',m',\v{q}'} = -\delta_{\v{q}\v{q}'}\delta_{mm'}\delta_{bb'}v^{(b)},
\end{equation}
which is trivially Hermetic and positive definite provided $v^{(b)}<0$. In this case we say that the irreducible representation $b$ is an
attractive channel. $A$ is in this case also trivially invertible with $A_{ij}^{-1} = -\delta_{ij}/v^{(b)}$.
Writing out all the indices we finally arrive at the Hubbard-Statonovich transformation of the interaction potential in individually
attractive symmetry channels
\begin{equation}
    \label{eq:Field:HS:Symm:HSTransformedPotentialExp}
    \mathcal{Z}_I = \int\!\mathcal{D}[\eta_\v{q}^{(b_m)\,\ast}\eta_\v{q}^{(b_m)}]e^{\int_0^\beta\!\mathrm{d}\tau\!\sum\limits_{\v{q}bm}\big[\frac{|\eta_\v{q}^{(b_m)}|}{v^{(b)}} - \big(J_\v{q}^{(b_m)\,\ast}\eta_\v{q}^{(b_m)} + J_\v{q}^{(b_m)}\eta_\v{q}^{(b_m)}\big)\big]},
\end{equation}
where
\begin{equation}
    \label{eq:Field:HS:Symm:JRedef}
    J^{(b_m)}_\v{q} = \sum_{\v{k}\,s_1s_2}d_{s_1s_2}^{(b),m}(\v{k})\xi_{\frac{\v{q}}{2}-\v{k},s_1}\xi_{\frac{\v{q}}{2}+\v{k},s_2}
\end{equation}
in terms of Gra\ss mann variables $\xi$. We note that this derivation does not assume either odd or even basis functions for the
irreducible representations and works just as well for either.

\section{Log-expansion}
