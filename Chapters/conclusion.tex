\chapter{Outlook}\noindent
%
%
In this thesis we have given an introduction to some of the fundamental techniques we employed in our theoretical investigations
into the nature of unconventional superconductivity with $p$-wave pairing symmetry. These investigations have resulted in three papers.

In Paper I we used a group-theoretical approach to
motivate the form of the effective interaction potential between electrons whose low energy excitations could be described
in terms of Cooper-pairs with $p$-wave symmetry. This potential was then used as a basis for deriving the effective free energy
for such a superconductor when it was influenced by explicite spin-orbit interaction. We found that the effective free energy
had the expected form given by its group-theoretical constraints, but that previous assumptions about its coefficients needed
revision because of the effect of spin-orbit coupling.

In Paper II we used large-scale Monte-Carlo simulations to investigate the vortex matter of a $p$-wave superconductor with a free
energy similar to the one derived in Paper I. We found a transition between a square vortex-lattice of single-quanta vortices
to an hexagonal vortex lattice consisting of double quanta vortices as the temperature was lowered in a finite field parallel
to the crystallographic $c$-axis.

In Paper III we investigated this same superconductor when exposed to zero external magnetic field and found an Ising phase transition
in the neutral sector of the theory. This transition did not separate from the phase-transition of the charged sector in contrast
to other models of two-component superconductors. The reason for the connection between the charged and neutral modes seemed to
be because of their group-theoretical nature as components of a single
irreducible representation. This identification implies equal stiffness to both components and an explicit coupling through mixed gradient
and mixed-component terms in the effective free energy.

Looking towards the future, it now seems less likely that the unconventional superconductor \ce{Sr2RuO4} should be theoretically
modeled as a $p+ip$ superconductor in spite of the strong evidence for its spontaneous time-reversal-symmetry breaking nature
\cite{Luke98,Xia06,Grinenko20} and the good
agreement between theory and experiment for the qualitative nature of its vortex lattices \cite{Ray14,AsleGaraud16}. Rather, the
prevailing view based on the current evidence has shifted to suggesting a degeneracy between a $d_{x^2-y^2}$ and a $g_{xy(x^2-y^2)}$ 
superconducting state \cite{Kivelson20,Ghosh21}. It might in this regard be interesting to use Monte-Carlo simulations to
investigate the vortex lattice behavior of such a superconductor and see how it matches with experiments on \ce{Sr2RuO4}.

In the exploration of such a novel symmetry state it might be beneficial to utilize more modern forms of Monte-Carlo analysis
such as those offered by the advances in machine-learning to yield convincing results in an efficient manner
\cite{Bojesen18,Nagai20,Bedolla20}.

Building on the results of Paper I, it would be of interest to investigate numerically how the kinetic dimensionless phenomenological
parameters in the Ginzburg-Landau model depend on the microscopic parameters, especially spin-orbit coupling strength and
spin-orbit coupling spin $z$-component.
