\chapter{Introduction}\noindent
%
% Should be engaging and easy to understand. Make reader want to read the rest of the thesis.
%
Research into superconductors holds a vital key in the development of technologies that can reduce global emissions of greenhouse gasses
and thus prevent large economic as well as human losses due to the effects of the climate crisis.
In IPCC's special report, they state that in order to have no or limited overshoot in global temperature from the goal of \SI{1.5}{\degreeCelsius},
the global net anthropogenic emissions of \ce{CO2} need to decline by $45\%$ compared to such emission levels in 2010, and this has to happen
by 2030. The emission levels must then continue to decline, reaching net zero around 2050 \cite{Allen18}.
In order for the member nations of the Paris Agreement to meet this goal, the NDC Synthesis report \cite{NDC21} highlights the need for further
increase in the nations' contributions compared to those that are currently declared. Measures mentioned by member nations for mitigating the
release of greenhouse gasses include renewable energy generation, electrification of the transport sector and more efficient electrical grids.
Because of the non-traditional properties of superconductors, such materials could potentially be of great benefit in further strengthening
such mitigation strategies.

In aircraft travel, designs for hybrid electric aircraft such as NASA's N3-X are underway. Analysis shows that fully utilizing high temperature
superconductors in the propulsion system could provide as much as $3.5$ times higher power-to-weight ratio than previous designs due to
superconductors' high current-densities \cite{Corduan20}.

In Norway there has recently been a debate about the development of wind turbine parks close to population centers and in vulnerable natural
habitats. Moving the wind power production from land to sea solves some of the debated issues but needs effective turbines. Including high
temperature superconductors into the design of such off-shore wind-turbines is beneficial for much the same reasons as for aircraft design:
the high power-density makes for a compact, lightweight and efficient construction \cite{Cheng21, Liu18}.

Other examples of future applications of superconductors include their use in more efficient power grids \cite{Tixador19, Stemmle14}, sustaining the high
magnetic fields needed for nuclear fusion \cite{Hartwig20, Whyte19} and for the operation of a particle collider more powerful than the
LHC \cite{Mentink18}, faster, more efficient electronics for digital logic and memory devices and more robust quantum computers \cite{Bommer19}.

Superconductors also currently have numerous important applications. In the Ch\=u\=o Shinkansen magnetic levitation line, which is currently
under construction, the interaction between superconducting coils in the train and copper coils on both sides of the track provides levitation and
guidance of the train at high speeds \cite{Bernstein20}. Superconductors are essential for generating the high strength magnetic fields needed
in MRI imaging. They are also used in other medical settings such as measurements of the electrical currents in the heart (magnetocardiography),
in measuring the concentration of iron stored in the liver (biomagnetic liver susceptometry) and cancer treatments through their role in 
particle accelerators \cite{Alonso12}.

All of this is the product of fundamental research into the electronic properties of metals and other materials that has shown that for some
of them, at a critical temperature $T_c$, the electrical resistivity of the material suddenly vanishes and any external magnetic field is expelled. These
are the two main properties that we associate with the phase of superconductivity. Zero resistivity means that electricity can travel through the
material without losing any energy, in contrast to a normal conductor where energy is usually lost through heat.
The expulsion of magnetic fields is called the Meissner effect and is in a sense the more fundamental
of the two properties. On a microscopic level it is due to electrons forming paired states that share certain features in such a way that different pairs
can behave as one. Because a macroscopic number of states share these features, the quantum mechanical nature of such states which is usually
only significant for tiny particles, becomes apparent through these non-classical macroscopically measurable effects.


\section{A brief history of superconductivity}

Superconductivity was first discovered in mercury at $T_c\approx\SI{-268.99}{\degreeCelsius}$ by Heike Kamerlingh Onnes in the Netherlands in 1911 \cite{Onnes}.
The Meissner effect was then discovered in 1933 by W. Meissner and R. Ochsenfeld \cite{Meissner33}. These discoveries happened without any previous
theoretical prediction or explanation. Theoretical description was then gradually developed, first by a simple thermodynamic two-fluid model of electron densities
by Garter and Casimir and then in 1935 by the phenomenological theory of the electromagnetic properties by H. and F. London \cite{London35}. WWII
came and went and then a significant improvement on the London-model was published by V. L. Ginzburg and L. D. Landau in 1950 \cite{Ginzburg50}, which built on Landau's
previous description \cite{Landau37} of a second order phase-transition by an order-parameter quantity. Based on this theory, Abrikosov introduced
the concept of a type-II superconductor in 1952, which has negative surface energy and a mixed phase at non-zero magnetic field \cite{Abrikosov52}.

An attempt at a microscopic theory was given by Fr\"olich in 1950 based on electron-phonon interaction \cite{Frolich50}. Even though the perturbation theory he derived
failed to predict important superconducting properties such as the Meissner effect, his Hamiltonian later became well known as a
fruitful starting point for the application of field theoretic methods.
In 1953 Pippard introduced a second length scale, the coherence length $\xi$, through a non-local modification of the London-model \cite{Pippard53}.
This length scale was a measure of the width of the interface between normal and superconducting regions. Although not a theory of superconductivity
itself, Landau's Fermi-liquid theory which came in 1956 would prove crucial in the development of a microscopic theory and describes the electronic
properties of many metals that at lower temperature become superconducting \cite{Landau56}. A complete microscopic theory of superconductivity was
published by J. Bardeen, L. N. Cooper, and J. R. Schrieffer in 1957 \cite{Bardeen57,BCS}. The BCS-theory was based on the idea that Fermi-liquid
quasiparticles with opposite momentum could form an attractive interaction through an intermediate interaction with a phonon. This would then lead
to the formation of pairs that could form a condensate, and which implied an energy gap $\Delta$ between the energies of paired electrons
and energies of normal quasiparticle states in the Fermi-sea. This year Abrikosov also published his prediction of the existence of a lattice of
vortices in the mixed state of type-II superconductors \cite{Abrikosov56}.

A separate form of a microscopic theory appeared in 1958 by N. N. Bogoliubov in a series of papers \cite{Bogoliubov58I, Bogoliubov58III, Bogoliubov58}.
This methodology of solving the Fr\"olich Hamiltonian was presented in a book \cite{deGennes66} by P. G. de Gennes and has since become known as
the Bogoliubov-de Gennes or BdG equations.

The diagrammatic methods developed for high-energy physics was first applied by Gor'kov to the
problem of superconductivity in 1958 when he calculated Green's functions based on the ideas of BCS-theory that reproduced
its results \cite{Gorkov58}. He then in 1959 used these methods to prove that the Ginzburg-Landau theory follows from the BCS theory in the limit $T\to T_c$ \cite{Gorkov59}.
The application of field theoretic methods was extensively developed by the work of Nambu, published in 1960, where he introduced the Nambu-spinor
for calculating the Gor'kov Green's functions. A perturbation theory for these Green's functions were calculated by \'Eliashberg in the same year
following a similar approach as Nambu, which later become known as the \'Eliashberg theory \cite{Eliashberg60}.

The understanding of the effects of impurities got an important contribution in 1959 by what is known as Anderson's theorem \cite{Anderson59}.
It says that any instability of the Fermi-surface that does not lift the Kramer degeneracy of time-reversed paired quasi-particles do not affect
the mean-field transition temperature \cite{Balatsky06}.
The idea of an energy gap in the excitation spectrum, which was integral to the BCS theory, was given strong experimental backing by the tunneling
experiments of I. Giæver in 1960 \cite{Giaever60}.
Such experiments were given a theoretical understanding by B. D. Josephson in 1962, through what is now known as the Josephson effect
\cite{Josephson62}.

From the framework of the Gor'kov Green's functions, a set of transport equations were derived for type-II superconductors in 1968 by
Eilenberger \cite{Eilenberger68}. These equations were further simplified for the case of a dirty superconductor by Usadel in 1970 \cite{Usadel70}.

From the perspective of our work, the discovery of a new phase in \ce{He^3} by Osheroff \etal\ in 1972 was of particular importance. Although a superfluid
and not a superconducting phase, the $A$-phase of this system has an unconventional anisotropic pairing symmetry which was famously described by Leggett in his
1975 review article \cite{Leggett75}. This is the same symmetry that we have considered in our work.

Superconductivity was found in the first heavy-fermion system \ce{CeCu2Si2} in 1979 by Steglich \etal\ \cite{Steglich79}. The heavy-fermion superconductors are systems where the superconducting
state consists of quasiparticles that are fermions with large effective masses and where the superconducting order is of an
unconventional character. For a review see Ref.~\cite{White15}.

The first high-$T_c$ superconductor was discovered in the form of\\ \ce{La_{2-x}Ba_xCuO4} by Bednorz and M\"uller in 1986 \cite{Bednorz86}.
This was followed up one
year later by the discovery of \ce{YBa2Cu3O_{7-x}} by M. K. Wu \etal\ \cite{Wu87}. These discoveries ushered in an era of superconductivity
research dominated by cuprates --- ceramic compounds consisting of metal oxides between planes of \ce{CuO2}. These are truly unconventional
superconductors in that their pairing symmetry is demonstrably non-isotropic. It was in 1987 proposed by V. J. Emery that antiferromagnetic spin-fluctuations
could cause such an anisotropic pairing \cite{Emery87}. This theory of the cuperate superconductive mechanism was then extensively studied
by P. Monthoux, D. Pines and D. J. Scalapino \cite{Monthoux91, Monthoux92, Monthoux94}, among many others in the early 90s, however
a consensus on its validity is yet to be reached due to its seeming inconsistency with normal state properties of the materials \cite{Keimer15}.
Through a group-theoretical approach, a vast array
of unconventional symmetries and their Ginzburg-Landau theories and physical properties were enumerated by Sigrist and Ueda in 1991 \cite{SigristUeda91}.
By which of these symmetries the superconducting state of cuprates could be described was in the early 90s a topic of much discussion, however
due in part to strong evidence from phase-sensitive SQUID measurements of \ce{YBCO} by Wollmann \etal\ in 1993 \cite{Wollman93},
it was by 2000 firmly established as a $d_{x^2-y^2}$ symmetry \cite{Tsuei00}.

In 1994, superconductivity was discovered in the perovskite structure of \ce{Sr2RuO4} by Y. Maeno \etal\ \cite{Maeno94}. This proved that copper
was not a necessary ingredient for superconductivity in layered perovskite crystal structures and would be the starting point of a still-standing
debate about its pairing symmetry which served as the immediate backdrop to our own research.

One of the phenomena that needs a theoretical explanation for a full understanding of superconductivity in the cuperate family of high-$T_c$ superconductors
is the pseudogap phase. Above the transition temperature, but below a characteristic temperature $T^\ast$ there is a hitherto undiscovered phase in such compounds where the electronic density of
states near the Fermi-surface continues to be suppressed by an energy gap $\Delta_\text{PG}$. This phase was named the pseudogap phase by Ding \etal
in 1996 \cite{Ding96}, and its origin continues to be a hotly debated topic.

2008 marked the beginning of the ``iron age'' of superconductivity research by the discovery of the first iron-based superconductor \ce{La[O_{1-x}F_x]FeAs} by
Y. Kamihara \etal\ \cite{Kamihara08}. The materials in this family of superconductors, called iron-pnictides, feature high $T_c$ and several other exotic
properties including nematic order \cite{Fernandes14}. For a review see Ref.~\cite{Paglione10}.

Lastly one could argue that we now have entered a ``hydrogen age'', as an increasing number of hydrogen-rich compounds are approaching room-temperature
superconductivity when they are placed under insane pressures \cite{Semenok20, Snider20}. It could also be argued that we are currently in a
``topological age'' as topological edge states in superconductors constitute a field under intense study \cite{Wang20} that potentially have far-reaching
consequences through their immediate application to quantum computing. Others again, would surely argue that we are in a ``graphene age'' as 
novel forms of superconductivity have been observed in twisted layers of graphene \cite{Park21}.
Which age we are in, I suspect, depends on what field the researcher
you are asking works on, and a clear answer will have to be postponed until seen through the coarse-grained eyes of history.


\section{About this work}

In the last section we saw how the research into the phenomena of superconductivity has blossomed into a myriad of different directions and sub-fields. Through
all this research, an implicit motivation has been the search of one day finding a theory or a specific system of a material that is superconducting at
room temperature. This has become a goal, similar to how the alchemists searched for the philosopher's stone, that we still have not quite reached,
but whose continued pursuit itself has borne numerous fruits. 

As for this thesis, we have focused on a branch of unconventional superconductivity that pertains to the description of phases with
$k_x\pm ik_y$ chiral $p$-wave pairing symmetry. The $p$ in $p$-wave implies that the pairing states internal angular momentum has $l=1$, \ie it has a $1$st order (linear) dependence
on its internal angular momentum\footnote{The letter-convention stands for ``principal'' and comes historically from the study of atomic emission spectra 
that result when electrons jump between different orbitals.}. As we mentioned, this is the same kind of state that describes the real world system of the
\ce{He^3} superfluid $A$-phase. In that context it is often referred to as the ABM-state after Anderson Brinkman and Morel who first described this pairing
state in the context of the BCS-theory \cite{Anderson61} in 1961, and then demonstrated how this state was stable in the $A$-phase
of \ce{He^3} in 1973 \cite{Anderson73}. It was for a long time thought that the unconventional superconducting phase of the perovskite compound \ce{Sr2RuO4}
had such a pairing symmetry, with one of the chief reasons being that it clearly features spontaneously broken time-reversal symmetry. This
formed in part the motivation for much of our work.

The determination of pairing symmetries is an important step in understanding what type of superconductivity is in a system because it leads to distinct
experimental consequences. Roughly speaking one can think of the pairing symmetry as determining the $\v{k}$ dependence of the gap function
\begin{equation}
    \label{eq:Intro:GapFunc}
    \Delta(\v{k}) = \sum_{m}\eta_mb_m(\v{k}),
\end{equation}
where $b_m(\v{k})$ are basis functions that depend on the point group symmetry of the system and $\v{k}$ determines a point on the Brillouin zone. Depending
on these basis functions then, $\Delta(\v{k})$ could have points or lines in the Brillouin zone where it vanishes, so-called point- or line-nodes. The 
existence of such nodes implies distinct signatures such as quadratic low temperature-dependence of the specific heat. Other examples of experimental signatures
of unconventional symmetry include
temperature-independent Knight-shift, magnetic field dependent Kerr-angle rotation and unconventional symmetry of lattices of magnetic vortices. For an introduction to superconductors with
unconventional pairing symmetries we highly recommend the lecture notes by Sigrist in Ref.~\cite{Sigrist05} and \cite{Sigrist09}.

Importantly, one may derive the form of the Ginzburg-Landau theory of the superconductor based solely on the pairing-symmetry as was done by Sigrist and Ueda
for a large number of different symmetries in 1991 \cite{SigristUeda91}. The Ginzburg-Landau theory is a
phenomenological theory, meaning that it explains the effective phenomenon observable in superconductors without necessarily knowing all the microscopic
details. As such, any correct microscopic theory should then reduce to the phenomenological Ginzburg-Landau theory in the limit of $T\to T_c$. As we mentioned,
Gor'kov first did this for a conventional $s$-wave superconductor in 1959 through a Green's function approach \cite{Gorkov59}. One fruitful
starting point for such a microscopic theory is the Hubbard-Hamiltonian
\begin{equation}
    \label{eq:Intro:HubbardHamiltonian}
    \hat{H} = \sum_{ijss'}H_{ij;ss'}\hat{c}_{is}^\dagger\hat{c}_{js'} + \mathcal{O}(\hat{c}^4),
\end{equation}
which itself can be viewed as an effective theory of the underlying many-body
quantum mechanics. In a Hubbard-theory, electrons can occupy sites in an atomic crystal lattice and hop from one site to another. Any long-range interaction
such as the Coulomb interaction is then written in terms
of how electrons at different sites interact through nearest neighbor terms, next-nearest neighbor terms, etc. In Chapter~\ref{chap:Field} we will describe some of the
tools useful in deriving an effective Ginzburg-Landau theory from such a microscopic starting point. We then in Chapter~\ref{chap:Group} present the
group-theory needed for deriving the requirements on such a microscopic theory, for this to result in a sought-after pairing-symmetry.

In the last chapter of Chapter~\ref{sec:Vor}, we present some tools useful when investigating magnetic vortices and vortex-lattices in chiral $p$-wave
superconductors.

As was said by Leggett in his Nobel lecture, there are very few things that can be proved rigorously in condensed matter physics, by which he referred to analytic
arguments \cite{Leggett03}. To get around this difficulty, we have used Monte-Carlo techniques that relies on the power of computers to simulate physical
consequences of a theoretical Ginzburg-Landau model. Such techniques have a long history in our group of being successfully
able to simulate superconductive systems. For a few examples, see \cite{Nguyen99PRB, Smiseth05, Smorgrav05, Bojesen14, Galteland15}. Although similar techniques
have been used for centuries, their modern form
was first developed in the context of nuclear research in the Manhattan project. After the researchers had become familiar with the Monte Carlo casino in Monaco,
they named it after the casino because of the technique's reliance on random or pseudo-random numbers to
calculate multidimensional integrals \cite{Metropolis87}.

In our use of Monte-Carlo techniques, which we present in Chapter~\ref{chap:Monte}, they are used
to calculate thermal averages of statistical-mechanical
observables. The rough procedure is that first, the theory under investigation is discretized to a corresponding lattice model as described in Chapter~\ref{sec:LM},
such that the probability of any configuration of fields on this numerical lattice can be computed. The theory then implies a probability distribution for how
likely certain field-configurations are to materialize in a real system.
Then, from a predetermined starting configuration, small changes are made incrementally to the numerical configuration in such
a way as to yield numerous statistical samples of field configurations that follow the theoretical probability distribution after a sufficient number of
incremental changes have been done. These samples of field configurations finally are used to calculate statistical averages, which corresponds to taking
thermal averages in statistical mechanics of the observables we are interested in. 

We begin this thesis in Chapter~\ref{chap:statMech} with a brief review
of some relevant aspects of statistical mechanics that should refresh what is meant by thermal averages of observables, how these are tied to probability
distributions of system configurations, and a brief introduction to Landau and Ginzburg-Landau theory as it pertains to phase transitions.

%\subsection{View of history of technology as progressing through knowledge of materials}
%
%\subsection{Discovery of superconductivity}
%
%\subsection{Initial theoretical description}
%
%% vortex BKT transition, Mermin-Wagner theorem, Elitzurs theorem
%% BdG framework, Keldysh formalism.
%
%\subsection{Experimental discovery of high-$T_c$ superconductivity}
%
%% Mott insulator, Kondo systems
%% Superconducting polymers and organic superconductors.
%% The discovery of superconductivity in ceramic compounds (J. Georg Bednorz, K. Alexander Muller)
%% Discovery of superfluid He-3 (David M. Lee and Douglas D. Oscheroff and Robert C. Richardson)
%
%\subsection{Unconventional vs. conventional superconductivity}
%
%% Line nodes and point-nodes, consequence on specific heat
%% Alternative mechanisms: Friedel oscillations, electron spin-fluctuations.
%% Timer reversal symmetry breaking materials: muon spin-rotation, spontaneous Hall effect, Magneto-optic effect, Polar Kerr effect
%% Spin-orbit coupling and DMI (Deyalovshinzki-Moria interaction)
%% Odd-frequency superconductivity.
%% Andreev boundary states.
%% Shiba bound states.
%% Majorana bound states.
%% Pomeranchuk phase / instability. Charge 
%% FFLO
%% Topological insulator - sc. transition ref. Eirik / Stefan.
%% Spin-triplet vs spin-singlet superconductors
%% GMR giant magneto-resistance effect.
%
%
%\subsection{Theoretical developments influence on other domains}
%% Spontaneous sc breaking, Higgs theory and Goldstone boson, Elitzurs theorem
%
%\subsection{Development of experimental techniques}
%
%% TEM, NMR, Kerr-angle, sputtering, Knight-shift, SQUID
%% ARPES and spectral function. Spin-ARPES.
%% Molecular beam epitaxy.
%
%
%\subsection{Contemporary research areas}
%
%The theory Andreev bound states is very interesting \cite{Escrig18}.
%
%Noncentrosymmetric superconductors such as \ce{BiPd} has been found to have topologically non-trivial properties
%due to Dirac points under the Fermi-level and spin-polarized surface states \cite{Neupane16}
%
%Room-temperature high-pressure superconductivity has been shown to be possible by the use of various hydrogen rich material
%\cite{Semenok20, Snider20} and nickelates \cite{Si20}.
%
%ARPES is a recently developed experimental techniques that allows the differentiation between spins when mapping electrons on the
%Fermi-surface. This has recently been used on cuprates, showing the effect of spin-orbit coupling in such systems \cite{Gotlieb18}.
%
%% Magic angle superconductivity, i.e. twisted layer graphene
%Superconductivity has been observed for materials consisting of twisted sheets of graphene. This kind of superconductivity has been dubbed
%twisted-layer superconductivity and has been found both for bi-layers as well as three layers \cite{Park21}.
%
%Write an \textsc{introduction} to the field {\prefixFont field} here. 
