\chapter{Introduction}\noindent

% Should be engaging and easy to understand. Make reader want to read the rest of the thesis.

Research into superconductors holds a vital key in the development of technologies that can reduce global emissions of greenhouse gasses
and thus prevent large economic as well as human losses due to the effects of the climate crisis.
In IPCC's special report, they state that in order to have no or limited overshoot in global temperature from the goal of \SI{1.5}{\degreeCelsius},
the global net anthropogenic emissions of \ce{CO2} needs to decline by $45\%$ compared to such emission levels in 2010, and this has to happen
by 2030. The emission levels must then continue to decline, reaching net zero around 2050 \cite{Allen18}.
In order for the member nations of the Paris Agreement to meet this goal, the NDC Synthesis report \cite{NDC21} highlights the need for further
increase in the nations contributions compared to those that are currently declared. Measures mentioned by member nations for mitigating the
release of greenhouse gasses include renewable energy generation, electrification of the transport sector and more efficient electrical grids.
Because of the non-traditional properties of superconductors, such materials could potentially be of great benefit in further strengthening
such mitigation strategies.

In aircraft travel, designs for hybrid electrical aircraft such as NASA's N3-X are underway. Analysis shows that fully utlizing high temperature
superconductors in the propulsion system could provide as much as $3.5$ times higher power-to-weight ratio than previous designs due to
superconductors high current-densities \cite{Corduan20}.

In Norway there has recently been a debate about the development of wind turbine parks close to population centers and in vulnerable natural
habitats. Moving the wind power production from land to sea solves some of the debated issues but needs effective turbines. Including high
temperature superconductors into the design of such off-shore wind-turbines is beneficial for much the same reasons as for aircraft design:
the high power density makes for a compact, lightweight and efficient construction \cite{Cheng21, Liu18}.

Other examples of future applications of superconductors include their use in more efficient power grids \cite{Tixador19, Stemmle14}, sustaining the high
magnetic fields needed for nuclear fusion \cite{Hartwig20, Whyte19} and for the operation of a particle collider more powerful than the
LHC \cite{Mentink18}, faster, more efficient electronics for digital logic and memory devices and more robust quantum computers \cite{Bommer19}.

Superconductors also currently have numerous important applications. In the Ch\=u\=o Shinkansen magnetic levitation line, which is currently
under construction, the interaction between superconducting coils in the train and copper coils on both sides of the track provides levitation and
guidance of the train at high speeds \cite{Bernstein20}. Superconductors are essential for generating the high strength magnetic fields needed
in MRI imaging. They are also used in other medical settings such as measurements of the electrical currents in the heart (magnetocardiography),
in measuring the concentration of iron stored in the liver (biomagnetic liver susceptometry) and cancer treatments through their role in 
particle accellerators \cite{Alonso12}.

All of this is the product of fundamental research into the electronic properties of metals and other materials that has shown that for some
of them, at some critical temperature $T_c$, the electrical resistivity of the material suddenly vanishes and any external magnetic field is expelled. This
is what we call the phase of superconductivity. Zero resistivity means that electricity can travel through the material without losing any energy, for
example through heating of the conductor. The explusion of magnetic fields is called the Meissner effect and is in a sense the more fundamental
of the two aspects. On a microscopic level it is due to pairs of electrons forming states that share certain features in such a way that they
can behave as one. Because a macroscopic number of states share these features, then the quantum mechanical nature of such states which is usually
only significant for tiny particles, becomes apparent through these non-classical macrosopically measureable effects.


\section{Historical development of superconductivity - a glimpse}

Superconductivity was first discovered in mercury at $T_c\approx\SI{-268.99}{\degreeCelsius}$ by Heike Kamerlingh Onnes in the Netherlands in 1911 \cite{Onnes}.
The Meissner effect was then discovered in 1933 by W. Meissner and R. Ochenfeld \cite{Meissner33}. These discoveries happened without any previous
theoretical prediction or explanation. Such theories were gradually developed, first by a simple thermodynamic two-fluid model of electron densities
by Garter and Casimir and then in 1935 by the phenomenological theory of the electromagnetic properties by H. and F. London \cite{London35}. WWII
came and went and then a significant improvement on the London-model was published by V. L. Ginzburg and L. D. Landau in 1950, which built on Landau's
previous description of a second order phase-transition by an order-parameter quantity \cite{Ginzburg50}. Based on this theory, Abrikosov introduced
the concept of a type-II superconductor in 1952, which has negative surface energy and a mixed phase at non-zero magnetic field \cite{Abrikosov52}.
In 1953 Pippard introduced a second length scale, the coherence length $\xi$, through a non-local modification to the London-model \cite{Pippard53}.
This length scale was a measure of the width of the interface between normal and superconducting regions. Although not a theory of superconductivity
itself, Landau's fermi liquid theory which came in 1956 would prove crucial in the development of a microscopic theory and describes the electronic
properties of many metals that at lower temperature become superconducting \cite{Landau56}. A complete microscopic theory of superconductivity was
published by J. Bardeen, L. N. Cooper and J. R. Schrieffer in 1957 \cite{Bardeen57}. The BCS-theory was based on the idea that fermi-liquid
quasiparticles with opposite momentum could form an attractive interaction through an intermediate interaction with a phonon. This would then lead
to the formation of pairs that could form a condensate, and which implied an energy gap $\Delta$ between the energies of paired electrons
and energies of normal quasiparticle states in the Fermi-sea. 

The diagrammatic methods developed for quantum field theory was first applied to the
problem of superconductivity by Gor'kov in 1958 \cite{Gorkov58} when he calculated Green's functions based on the ideas of BCS-theory that reproduced
its results. He then in 1959 used these methods to prove that the Ginzburg-Landau theory follows from the BCS theory \cite{Gorkov59}.
The application of field theoretic methods was further developed by the work of Nambu in 1960, where he introduced the Nambu-spinor for calculating the
Gor'kov Green's functions.

The idea of an energy gap in the excitation spectrum, which was integral to the BCS theory, was given strong experimentall backing by the tunneling
experiments of I. Giæver in 1960 \cite{Giaever60}.
Such experiments were given a theoretical understanding by B. D. Josephson in 1962, through what is now known as the Josephson effect
\cite{Josephson62}. 

Much as the alchemists of old who had a goal of synthesizing the philosopher's stone, so too have condensed matter scientists had a dream
of one day discovering a practical material that is superconducting at room-temperature. The alchemists never acheived their goal but their
pursuit paved the way for a plethora of important discoveries that laid the groud work for the periodic table, chemistry and science
as we know it. In the same way, room-temperature superconductivity at normal pressures has still not been discovered, but this pursuit
has yielded an enormously diverse field of different kinds of superconductivity with different applications \cite{Moran05}.

\subsection{View of history of technology as progressing through knowledge of materials}

\subsection{Discovery of superconductivity}

\subsection{Initial theoretical description}

% London-theory
% Landau and Ginzburg-Landau theory
% Abrikosov theory
% Anderson's theorem
% Theory of disorder and impurities (Abrikosov-Gorkov, Hirschfeld)
% vortex BKT transition, Mermin-Wagner theorem, Elitzurs theorem
% BCS theory, Eliashberg theory, BdG framework, Usadel equation, Keldysh formalism.

\subsection{Experimental discovery of high-$T_c$ superconductivity}

% Cuperates
% Doping dome and quantum critical point. Anti-ferromagnetism.
% Hevy fermion systems.
% Mott insulator, Kondo systems
% Superconducting polymers and organic superconductors.
% Pseudogap, nernst phase and nematic order.
% The discovery of superconductivity in ceramic compounds (J. Georg Bednorz, K. Alexander Muller)
% Discovery of superfluid He-3 (David M. Lee and Douglas D. Oscheroff and Robert C. Richardson)

\subsection{Unconventional vs. conventional superconductivity}

% Line nodes and point-nodes, consequence on specific heat
% Alternative mechanisms: Friedel oscillations, electron spin-fluctuations.
% Timer reversal symmetry breaking materials: muon spin-rotation, spontaneous Hall effect, Magneto-optic effect, Polar Kerr effect
% Spin-orbit coupling and DMI (Deyalovshinzki-Moria interaction)
% Odd-frequency superconductivity.
% Andreev boundary states.
% Shiba bound states.
% Majorana bound states.
% Pomeranchuk phase / instability. Charge 
% FFLO
% Topological insulator - sc. transition ref. Eirik / Stefan.
% Spin-triplet vs spin-singlet superconductors
% GMR giant magneto-resistance effect.


\subsection{Theoretical developments influence on other domains}
% Spontaneous sc breaking, Higgs theory and Goldstone boson, Elitzurs theorem

\subsection{Development of experimental techniques}

% TEM, NMR, Kerr-angle, sputtering, Knight-shift, SQUID
% ARPES and spectral function. Spin-ARPES.
% Molecular beam epitaxy.


\subsection{Contemporary research areas}

Noncentrosymmetric superconductors such as \ce{BiPd} has been found to have topologically non-trivial properties
due to Dirac points under the Fermi-level and spin-polarized surface states \cite{Neupane16}

Room-temperature high-pressure superconductivity has been shown to be possible by the use of various hydrogen rich material
\cite{Semenok20, Snider20} and nickelates \cite{Si20}.

ARPES is a recently developed experimental techniques that allows the differentiation between spins when mapping electrons on the
Fermi-surface. This has recently been used on cuperates, showing the effect of spin-orbit coupling in such systems \cite{Gotlieb18}.

% Magic angle superconductivity, i.e. twisted layer graphene
Superconductivity has been observed for materials consisting of twisted sheets of graphene. This kind of superconductivity has been dubbed
twisted-layer superconductivity and has been found both for bi-layers as well as three layers \cite{Park21}.

Write an \textsc{introduction} to the field {\prefixFont field} here. 
