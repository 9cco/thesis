\chapter{Introduction}\noindent

% Should be engaging and easy to understand. Make reader want to read the rest of the thesis.

In the world today we see an ever increasing need to not waste, but use energy in the form of electricity as
efficiently as possible. There is also a continuous drive to create ever more advanced computers to process
the ever increasing amounts of data that we produce and to help us calculate and come up with further innovation.
At a fundamental level, the limit of how much advancement we can make in these areas is dependent on the
materials available to build solutions out of and our understanding of how these materials behave. In the stone
age, the material understanding allowed technology to be made out of wood, bone and stone. In the bronze age,
technology superiour to this could be made because we had discovered how to extract a metal from its natural
stony oxide. When it comes
to solutions using electricity then the electrical properties of materials is naturally especially important. The
current age has been called the information age, and most of the infrastructure upon which this information is
stored and communicated through is possible because of our understanding of the behaviour of electricity,
or more precicely electrons in materials.
Electricity is understood as a stream of electrons and thus we often talk in this respect of materials
electronic properties.

In $1911$, the electronic property called superconductivity was discovered in Netherlands by Heike Onnes \cite{Onnes}. Mercury
was cooled to \SI{-268.99}{\degreeCelsius} and then Dr.~Onnes measured the resistivity, which is a measure of how much energy
is used to push electrons through a material. He found that the resistivity suddenly dissapeared at this temperature and lower.
It was expected that the resistivity would become lower with temperature but the extreme suddenness of the dissapearance was
surprising. Not only did resistivity dissapear so that electrons could flow completely freely through the material as if there
was no material there at all, but in $1933$ it was discovered that at this low temperature, its inside was devoid of any
magnetic field, even if a magnetic field had previously penetrated the material at the higher temperatures. When a material
changes its properties in such a sudden manner with respect to a controlling variable, in this case temperature, we call it
a phase transition. Just as when the mechanical properties of water changes from making it easily fill the shape of a cup
in its liquid form, to making it being stuck on top and impossible to drink after it has frozen into
ice, so the electronic properties of mercury changes in a fundamental way as well during a phase transition. The phase
transition that happens to water when it freezes into ice can be described by classical mechanics which is built on Newtons
theories, however because of the Meissner effect, classical mechanics is not sufficient to describe the phase-transition that
happens to mercury at \SI{-268.99}{\degreeCelsius}. For that we need quantum mechanics. Thus the new low-temperature phase
of mercury is not called a perfectly conducting phase, but a superconducting phase.

\section{Climate Change and its solutions}

Arguably one of modern industrialized civilization's largest future challenges is finding a way to reduce the amount of
climate gases like carbon dioxide in the atmosphere. 

Write an \textsc{introduction} to the field {\prefixFont field} here. 
