\chapter*{Acknowledgements}\noindent

First I would like to thank my supervisor Asle Sudb{\o}  for his willingness to take on a PhD student
that had little in regards to previous knowledge of condensed matter systems, for providing me with a steady stream of insights
and ideas for interesting research topics and for his long patience with my own slow and sometimes meandering
progress towards publishable results. 

I would like to thank the rest of the people in our research group
who I have come to know along the way; Troels Arnfred Bojesen who I met for the first time during my
visit to the March Meeting in New Orleans in 2017, who shared my fascination for Japan and whom have been
of invaluable assistance in understanding and troubleshooting the intricacies of Monte-Carlo algorithms
and techniques. Stephan Rex, who apart from inspiring me to start running, helped me in times when I was
stuck on mathematical technicalities and proved an excellent travelling companion. Peder Notto Galteland, who
helped ease my introduction into the social circles of the theory-section as it stood back
in 2015. Henning Goa Hugdal, whose gentle
demeanor and generosity made him always approachable and provided a soothing presence in times of need.
Even Thingstad, whose vast depth of knowledge in all things physics I both benefited from, and which inspired
me greatly, who was an amazing partner in our task of inspiring the younger generation, and whose friendship I
hold dearly. Håvard Homleid Haugen who has been an excellent collaborator with a keen eye for programming,
and intuitive understanding, with whom I've had numerous very stimulating discussions about physics and in general.
Jonas Blomberg Ghini, whose outlook on life seems so much like my own and who I regret not getting to know sooner.
And lastly, Eirik Erlandsen
whose brilliant and vigorous gregariousness has been a source of tremendous amounts of laughter and joy.

In the department of physics at NTNU there are many others who deserve recognition and praise for their role in creating
a friendly and interesting social environment. To mention everyone by name in such a long PhD as I have had would
prove excessive and frankly a boring read. Therefore I will not attempt to make an exhaustive list but mention
only a select few. 

Of these, first I want to extend a thank you to all the post-docs and doctoral students were already there in the ``theory corridor'' and in the
(old) lunchroom when I started my PhD. You were all
role-models and people I respected greatly. People such as Sol, Alireza and Roberto. Andr\'e with whom I shared an interest in sci-fi, Eirik
who I already knew from our masters program, Eirik Torbjørn Bakken who gave me a connection to the experimentalists at NTNU and Manu Lineares
with whom I shared an office. During my stay at the March Meeting I travelled with Dag-Vidar who proved splendid and
stimulating company.
When I started my PhD it was my honor to start at the same time as the other doctoral students Sverre, Therese, Jabir, and Vetle.
You have perhaps shaped my stay most of all and deserve special thanks for all our shared memories of running, climbing,
conversations, painting and music. In this connection I also want to mention Marina, who we almost counted as a theorist
for all her gifts of company during our lunch-breaks and outside of work.

The next generation of students included Martin and Øyvind. Two brilliant people: Martin with his many cat-stories and
perceptive humor, and Øyvind with his thoughtfulness and warm smile. Also, I have to mention of-course Jeroen and his student
Arnau who I feel have been there almost from the beginnning. Jeroen who contributed greatly to forming a sense
of community among the doctoral students of what has become QuSpin, lifting the intelligence-level in all conversations
he joined and making me aware of resources of learning I had previously missed. Arnau, who I came to know better gradually as the
years passed as having a great sense of humor and taste, and importantly: a willingness to join me for Beatz! Another
co-conspirator in that endaviour has been Akash, a man I learned a great deal from not only about physics and its
thinly veiled politics but also in areas of life, philosophy and sociology, one who has been integral in the culture
of QuSpin and who it's been an absolute pleasure knowing.

I want to give a big thanks to Frode, my other office-mate and good friend whose many discussion on physical fitness
have been inspiring and educational. I want to thank my good friends Matthias and Maximillian for many intelligent
conversations over good food and drink, and foraging expeditions into the forests of Trondheim.

In the last generation of QuSpin students I would be remiss not to mention Atousa, Lina, Marion, Payel, Jonas and Longfei
who have provided laughter, good company and engaing conversations, and whom I hope will have a
nice and productive future in QuSpin in spite of the limitations that the pandemic enforces.

My fantastic family deserve recognition above everyone as the ones who have shown me unwavering support and love all
through my life and this journey in spite of our long standing geographical separation and my own need for seclusion
when I immerse myself in my studies. To my mom and dad. To know that your door always is open should I fall, allows me
the courage to continue walking. To my two brothers who endured the consequences of all my uncertainties, thank you for
always letting our unity overshadow our differences. To my last sibling who in many ways reminds me much of myself,
thank you for all your warm hugs and for showing me your strength. To my paternal grandfather and late grandmother
who from an early age helped encourage my interest in science and maternal grandmother whose wisdom and love supersedes even
Her long age.

Finally a huge thank you to my girlfriend, who perhaps more than anyone has had a front row in all the ups and downs
in this long journey. I am so so grateful for all you have tought me, healed me and all the love you have given and continue to give.
