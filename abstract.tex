\chapter*{Abstract}\noindent
%
The symmetry of the superfluid $A$-phase of \ce{He^3} has previously been suggested to describe the unconventional superconducting state
of \ce{Sr2RuO4} which would make this material a chiral $p$-wave superconductor. In this thesis we discuss tools, results and 
techniques useful in the theoretical description of superconductors with this symmetry.

In Paper I we use field-integral techniques to investigate the effects of spin-orbit coupling on the coefficients of the phenomenological
Ginzburg-Landau theory of chiral $p$-wave superconductors. We find that these coefficients have a non-linear anisotropic dependence on the
spin-orbit coupling strength and direction in spin-space. This dependence necessitates two independent phenomenological parameters for
the mixed gradient terms and the mixed component terms respectively, even in the weak-field limit when written using
dimensionless variables.

In Paper II we use large-scale Monte-Carlo simulations to investigate the vortex-matter of a superconducting system that can be modelled
by a Ginzburg-Landau theory with chiral $p$-wave symmetry such as the one investigated in Paper I, but now in the limit of vanishing
spin-orbit coupling. We find that a square vortex lattice consisting of single-quanta vortices is stable at high temperatures close to
$T_c(B)$. The single-quanta vortices merge into double-quantum vortices at lower temperature which together then stabilizes a triangular
vortex lattice.

In Paper III we investigate a $Z_2$ Ising transition resulting from spontaneously broken time-reversal symmetry in the neutral sector
of chiral $p$-wave symmetric superconductors subjected to zero external field. We find that this transition is irrevocably tied to
the superconducting transition for all realistic values of the phenomenological parameters in our model.

% Ideas: What is the first order expansion in the spin-orbit coupling strength?
% Investigate numerically how the dimensionless parameters of the ginzburg-landua theory depends on the spin-orbit coupling strength given
% representative values for other parameters. Will an increasing spin-orbit coupling strength move the theory away from the A-phase
% towards the B-phase or C-phase?
